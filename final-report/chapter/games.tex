\chapter{Boolean circuit puzzles and games}

In this chapter, we begin to explore landscape of puzzle-and-game complexity
classes---specifically, the \emph{polynomial hierarchy}---through a series of
games played on boolean circuits.

\section{The \Problem{Circuit Value} problem}

To set the stage, we start with an ``easy'' problem, known as the
\Problem{Circuit Value} problem, or \Problem{CircVal} for short:

\begin{definition}{\(\Problem{Circuit Value}=\Problem{CircVal}\)}{}

  Let \(C\) be a given boolean circuit with all input wires/variables
  specified. What is the final output value of \(C\)? As a decision problem:
  \(C∈\Problem{Circuit Value}\) if it outputs \True, and \(C∉\Problem{Circuit
  Value}\) if it outputs \False.

\end{definition}

It is well-known that \(\Problem{Circuit Value}∈\P\) (i.e., it is actually
``easy'').  We give one version of a proof below.

\begin{proof}
  We give a polynomial-time algorithm solving \Problem{Circuit Value} below.
  (Note that this is not the most efficient algorithm doing so; we choose it
  here only for its simplicity.)

  \begin{algorithm}{}{}
    \begin{algorithmic}
      \Given{\(C\), a boolean circuit with all inputs fully specified}
      \LComment{Call a wire \emph{finished} if it has been assigned a boolean
        value. Initially, all the input wires are finished, since their values
      were given, and all intermediate and output wires are unfinished.}%
      \While{final output wire is not finished}%
      \ForEach{unfinished logic gate \(g\) in \(C\)}%
      \If{all input wires of \(g\) are finished}%
      \State{compute and assign the output value of \(g\) and to its output wire}%
      \EndIf%
      \EndFor%
      \EndWhile%
      \State \Return value assigned to final output wire%
    \end{algorithmic}
  \end{algorithm}

  We argue that this algorithm terminates in polynomial time.  On each
  iteration of the ``while'' loop, at least one logic gate is guaranteed to
  have all of its inputs done, since there are no cyclic dependencies in the
  circuit.  Thus each iteration of the ``while'' loop finishes at least one
  additional wire.  Therefore, the number of ``while'' iterations is at most
  the number of wires in the circuit, and the work done within each iteration
  is also polynomial with respect to the size of the circuit, so the overall
  algorithm terminates in polynomial time.
\end{proof}

To kickstart the puzzles-and-games perspective, we think of \Problem{Circuit
Value}---and actually, every problem in \P---as a game with \(0\) turns: the
player does nothing, and an (efficient) algorithm automatically decides whether
the player wins or loses.

This seems like a silly (arguably boring) idea.  But, as we see in the next few
sections, this approach allows us to generalize \Problem{Circuit Value} into
very powerful puzzles and games.

\section{The \Problem{Circuit Satisfiability} puzzle}

By \emph{puzzle}, we really mean \(1\)-turn games: games in which the player
makes a sequence of ``moves'' on a given ``game board'', and an (efficient)
algorithm then determines whether the player's moves constitute a win.
Formulated as decision problems, the computational puzzle is the yes/no
question:
\begin{center}
  Does the player have a winning strategy?
\end{center}

For example, consider the puzzle-ification of \Problem{CircVal}, where the
circuit's inputs are no longer specified but rather chosen by the player (this
is the ``move'' made by the player).  In this sense, to say that the player has
a winning strategy means that there exists a winning move the player can
make---that is, there exists a combination of inputs causing the circuit to
output \True.

Call this problem \Problem{Circuit Satisfiability}, or \Problem{CircSat} for
short:

\begin{definition}{\(\Problem{Circuit Satisfiability}=\Problem{CircSat}\)}{}

  Let \(C\), a boolean circuit, be given. Does there exist a combination of
  boolean input values to \(C\) causing it to output \True?

\end{definition}

Briefly: how (computationally) difficult is \Problem{CircSat}?  As it turns
out, nobody knows for sure, but it seems \emph{quite} difficult.  Loosely
speaking, all known algorithms for solving \Problem{CircSat} amount to brute
force with optimizations that enhance performance on ``practical'', real-world
inputs but do not save them from performing poorly in the worst case.
Tentatively, then, most computer scientists suspect that
\(\Problem{CircSat}∉\P\)---i.e., there is no polynomial-time solution for
\Problem{CircSat}.

Anyway, back to puzzles.  \Problem{CircSat} is one example of how a \(0\)-turn
game such as \Problem{CircVal} may be generalized into a \(1\)-turn game---a
puzzle.  How can we do this in general?

In the example of \Problem{CircSat}, we do this by making the player supplement
the input to the the \(0\)-turn analog, \Problem{CircVal}.  This approach is
readily generalized.  Given some input \(X\) (the ``game board''), construct a
\(1\)-turn game in which the player specifies a supplementary input \(Y\);
victory is decided by whether the pair of inputs \((X,Y)\) meets the \(0\)-turn
winning condition.  As before, the decision problem asks whether the player can
win---i.e., whether there exists \(Y\) such that the player wins.

The complexity class of problems constructed in this manner is called \NP:

\begin{definition}{\NP}{}

  \NP{} is the class of decision problems \(Π\) (the puzzle problem) for which
  there exists another problem \(Π'∈\P\) (the \(0\)-turn analog) such that, for
  each input \(X∈\Set{\True,\False}^*\), \(X∈Π\) if and only if there exists
  \(Y∈\Set{\True,\False}^*\) such that \((X,Y)∈Π'\).

  TODO say \& comment that \(Y\) needs to be polynomially bounded by length of
  \(X\)

\end{definition}



% SURVEY on pnp opinion: https://dl.acm.org/doi/10.1145/564585.564599
% https://www.researchgate.net/publication/292393040_The_PNP_poll

% TODO maybe cite an up-to-date result about how good the bound is, but whatever

% useful citation about best-known SAT bounds: https://cstheory.stackexchange.com/questions/1060/best-upper-bounds-on-sat

% https://www.sciencedirect.com/science/article/pii/S0304397501001748?via%3Dihub
% 3sat solvable in 1.5^n?




description of circuit satisfiability puzzle, guess-and-check solution

definition of NP

\section{Two-player circuit games}

two-turn version.


n-turns, defined inductively.
