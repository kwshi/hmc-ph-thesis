\chapter{Two fun games (and hopefully more)}

In the last chapter, we set the polynomial-hierarchy stage, focusing on the
circuit games \(\CircSatₖ\) as canonical examples of \SigmaP k-complete
problems.  In this chapter, we expand that landscape by exploring two other
(collections of) \SigmaP k-complete games: one played on graph colorings,
another played on set coverings.

It is only due to the time constraints on this thesis that we stop at two:
ultimately, I hope to convey, through the examples presented in this chapter,
the sense that there are many, many \SigmaP k-complete games out there, all of
which intuitive extensions of classic, well-known \NP-complete puzzles.

\section{Graph coloring games}

First, we introduce some preliminary definitions about graphs and colorings.  A
graph is a network of \emph{vertices} connected by \emph{edges}.  Formally:

\begin{definition}{(undirected) graphs}{}

  A \Term{graph} \(Γ\) is a pair \((\Vertices(Γ),\Edges(Γ))\) consisting of:
  \begin{itemize}[nosep]
    \item a finite set of \Term{vertices} \(\Vertices(Γ)\), and
    \item a finite set of \Term{edges} \(\Edges(Γ)⊆\Vertices(Γ)×\Vertices(Γ)\),
      which represent connections between pairs of vertices.
  \end{itemize}

  For our purposes, edges have no directionality.  That is, when specifying an
  edge, the ordering of vertices doesn't matter: \((u,v)\) specifies the same
  edge as \((v,u)\).

  We say that two vertices \(u,v∈\Vertices(Γ)\) are \Term{neighbors}, or that
  they neighbor each other, if \((u,v)∈\Edges(Γ)\).

\end{definition}

% TODO example giraph (fun)

The graph coloring games we explore in this thesis are about assigning colors
to vertices on a graph.  We call such an assignment a \emph{vertex coloring}.
Specifically, for sake of simplicity, we restrict our attention to colorings
that involve only three colors.  The main rule constraining these color
assignments is that neighboring vertices must always be colored distinctly---we
call this the \emph{properness} condition.  These terms are defined precisely
below.

\begin{definition}{vertex 3-colorings, properness}{}%

  Let \(Γ\) be a graph. A \Term{vertex 3-coloring} of \(Γ\) is a map
  \(κ\colon\Vertices(Γ)→\Colors\), which assigns to each vertex one of three
  colors.  In this thesis, we generally just say ``coloring'' to refer to
  ``vertex 3-colorings'', except when specified otherwise.

  A vertex coloring \(κ\) is a \Term{proper} coloring if, for every edge
  \((u,v)∈\Edges(Γ)\), \(κ(u)≠κ(v)\)---i.e., no neighboring vertices share the
  same color.  To simplify discourse, we also call a particular
  edge/neighboring-pair \((u,v)∈\Edges(Γ)\) is \Term{proper} if \(κ(u)≠κ(v)\).
  Thus a proper coloring is one where all edges are proper; an improper
  coloring contains at least one improper edge.

\end{definition}

Having established the basic terminology, we now introduce the graph
(3-)coloring games.

\subsection{The \(0\)-turn game}

The goal of graph coloring games is to assign colors to all vertices so that
the resulting coloring is proper.  To this end, the \(0\)-turn
winning-condition problem is that of checking properness of colorings, called
the \Problem{3-Coloring Properness} problem, or \ColProp{} for short:

\begin{problem}{\Problem{3-Coloring Properness} / \ColProp}{}

  Given a graph and a 3-coloring \(κ\) of it (specified by listing out each
  vertex and its assigned color), determine whether \(κ\) is a proper coloring.

  \tcblower
  \ColProp=\SetBuilder{(\text{graph \(Γ\)},\text{coloring \(κ\)})}{
    ∀(u,v)∈\Edges(Γ)\Q κ(u)≠κ(v)
  }
\end{problem}

In order for \ColProp{} to be usable as a basis for polynomial-hierarchy games,
we must first ensure that it itself is in \P.  Indeed, it is:

\begin{theorem}{\(\ColProp∈\P\)}{}

  \(\ColProp\) is solvable in polynomial time.

\end{theorem}

\begin{proof}

  We describe below a straightforward polynomial-time algorithm computing
  \ColProp.  It iterates through the edges and simply verifies the properness
  condition on each pair of neighbors:

  \begin{algorithm}{a polynomial-time \ColProp{} solver}{}
    \begin{algorithmic}
      \Given{a graph \(Γ\) and a coloring \(κ\colon\Vertices(Γ)→\Colors\)}%
      \ForEach{\((u,v)∈\Edges(Γ)\)}%
      \If{\(κ(u)=κ(v)\)}%
      \LComment{\((u,v)\) is improper!}
      \State{\Return no}%
      \EndIf%
      \EndFor%
      \LComment{all edges have been checked, and no improper ones were found,
      so the coloring is proper}
      \State{\Return yes}%
    \end{algorithmic}
  \end{algorithm}

  The number of edges is, by definition, bounded by the size of the graph, so
  the number of ``for each'' iterations is polynomial. Within each iteration,
  the \(κ(u)=κ(v)\) check runs within polynomial time, so the overall algorithm
  runs in polynomial time as well.  \qedhere

\end{proof}

\subsection{The \(k\)-turn games}

A graph coloring game is played on an initially uncolored graph \(Γ\).  In a
\(k\)-turn game, the graph's vertices are partitioned into \(k\) groups,
\(V₁,V₂,\dotsc,Vₖ\), and players alternate turns assigning colors to the
vertices in each group.  If, on any turn, a player introduces an improper edge
in the (partial) coloring, the other player wins.  If, after all turns, no
improper edges have been introduced---that is, the resulting coloring is
proper---then the \emph{last} player wins.

\begin{definition}{partial (vertex 3-)colorings}{}

  Let \(Γ\) be a graph.  A \Term{partial (vertex 3-)coloring} is a map
  \(κ\colon\Vertices(Γ)→\ColorsOpt\), which \emph{optionally} assigns a color
  to each vertex in \(Γ\) (\None{} means no color is assigned).

  Except where ambiguous, we will simply say ``partial coloring'' to mean
  ``partial vertex 3-coloring''.

  A partial coloring \(κ\) is \Term{proper} if, among the vertices it
  \emph{does} assign a color, there are no improper edges.  That is, for all
  \((u,v)∈\Edges(Γ)\), if both \(κ(u)\) and \(κ(v)\) are not \None, then
  \(κ(u)≠κ(v)\).

\end{definition}








%Given a graph \(G\) along with a \(3\)-coloring on \(G\), is the coloring
%proper?  We can solve this problem by simply checking, for each edge, whether
%the two vertices on that edge have different colors.  The run-time of this
%solution is \(\O(e)\) and therefore polynomially-bounded in the size of \(G\).
%Thus the problem of \emph{checking} whether a given \(3\)-coloring is proper is
%in \P.
%
%\subsection{The \(3\)-coloring puzzle}
%
%The puzzle-ification of this problem comes in the following form:
%\begin{definition}[\Problem{3col}]%
%  Given a graph \(G\), is there a way to properly \(3\)-color the vertices of
%  \(G\)?
%  \[
%    \Problem{3col} = \SetBuilder* G {
%      ∃\,\text{coloring \(C = (c_1,\dotsc,c_n) ∈ \Set{0,1,2}^n\)} \quad \text{\(C\) is proper}
%    }
%  \]
%\end{definition}
%It is straightforward to see from its definition and the fact that
%properness-checking is in \P{} that \(\Problem{3col} ∈ \NP\).
%
%The natural question to ask is: is it also \NP-complete?  After all, earlier,
%we could confidently expect that \NP-completeness from \emph{Boolean} \CircSat{}
%because of the universality of Boolean logic, but, at a glance, it isn't
%obvious that graphs and proper colorings are somehow ``fundamental'' to
%computation as Booleans are.  But, in fact, that is exactly the case:
%\begin{theorem}
%  \Problem{3col} is \NP-complete.
%\end{theorem}
%
%\section{Reduction from CSAT}

\subsection{Key idea: embedding circuits in colorings}

TODO narrative connective text

\begin{definition}{boolean graph}{boolean-graph}%
  A \Term{boolean graph} is a graph \(G\) whose vertices satisfy the following:
  \begin{itemize}
    \item There are three vertices labeled
      \(s_\True,s_\False,s_\Aux∈\Vertices[G]\), called the ``special''
      vertices, joined to each other by a triangle.
    \item There are \(n\) (not necessarily distinct) vertices
      \(i_1,\dotsc,i_n∈\Vertices[G]\), called the ``input'' vertices, each
      joined by an edge to \(s_\Aux\).  Purely for convenience, we allow each
      of the input vertices to be the same vertex as \(s_\True\) or
      \(s_\False\) (but not \(s_\Aux\), since that would require edges from
      \(s_\Aux\) to itself, which we forbid).
    \item There exists a vertex \(o∈\Vertices[G]\), called the ``output''
      vertex, also joined by an edge to \(s_\Aux\).  Again, for convenience, we
      allow \(o\) to coincide with \(s_\True\), \(s_\False\), or any of the
      input vertices.
  \end{itemize}

  Now, let \(κ\colon\Vertices[G]→\Set{0,1,2}\) be an arbitrary \emph{proper}
  3-coloring.  Because the three special vertices \(s_\True,s_\False,s_\Aux\)
  are joined by a triangle, we know that \(κ\) assigns them all three
  (distinct) available colors.

  For each input/output vertex \(v\), since \(v\) neighbors \(s_\Aux\) by
  construction, we know \(κ(v)≠κ(s_\Aux)\); then, since there are only three
  colors, we know \(κ(v)\) must equal \(κ(s_\True)\) or \(κ(s_\False)\).  We
  say the \Term{boolean value} assigned (by \(κ\)) to each input/output vertex
  \(v\) is \(\True\) if \(κ(v)=κ(s_\True)\) and \(\False\) if
  \(κ(v)=κ(s_\False)\).

  To simplify later discussions, we overload/abuse the notation \(κ(v)\) to
  denote the boolean value of \(v\); i.e., \(κ(v)=\True\) if \(v\) shares a
  color with \(s_\True\), etc.
\end{definition}

\begin{definition}{boolean graphs as boolean functions}{boolean-graph-functions}%

  Let \(G\) be a boolean graph.  We say that \(G\) computes a well-defined
  boolean function \(ϕ\colon\Set{\True,\False}^n→\Set{\True,\False}\), if, for
  every combination of boolean values \(x_1,\dotsc,x_n∈\Set{\True,\False}^n\),
  the following hold:
  \begin{itemize}
    \item There exists at least one proper 3-coloring that assigns boolean
      values \(x_1,\dotsc,x_n\) to input vertices \(i_1,\dotsc,i_n\),
      respectively.

      (There exists at least one 3-coloring \(κ\) such that \(κ\) is proper,
      and \(κ(i_j)=x_j\) for each \(j=1,\dotsc,n\).)
    \item Every such coloring assigns the (same) boolean value
      \(ϕ(x_1,\dotsc,x_n)\) to the output vertex \(o\).

      (For every such coloring \(κ\), \(κ(o)=ϕ(x₁,\dotsc,xₙ)\).)
  \end{itemize}
\end{definition}

\begin{example}{A graph computing the boolean identity function}{}%
  The following boolean graph computes the boolean identity function,
  \(ϕ(x_1)=x_1\).
  \begin{center}
    \begin{tikzpicture}[x=3em, y=3em]
      \coordinate[vertex](t) at (120:1);
      \coordinate[vertex](f) at (60:1);
      \coordinate[vertex](a);
      \draw (t) -- (f) -- (a) -- (t);
      \node[vertex label, left] at (t) {\(s_\True\)};
      \node[vertex label, right] at (f) {\(s_\False\)};
      \node[vertex label, left] at (a) {\(s_\Aux\)};

      \coordinate[vertex](i) at (-90:1);
      \node[vertex label, below] at (i.south) {\(i_1=o\)};
      \draw (i) -- (a);
    \end{tikzpicture}
  \end{center}

  TODO is explanation even needed, or is it actually obvious that this works?
\end{example}

\begin{lemma}{\NOT, \OR, and \AND{} graphs}{boolean-operation-graphs}%
  There exist graphs computing each of the basic boolean operations \NOT, \OR,
  and \AND.
\end{lemma}

\begin{proof}

  \tikzset{
    boolean graph/.style={x=3em, y=3em},
    over/.style={
      preaction={draw=white, line width=3pt},
    },
    triangle/.pic={
      \draw (0,0) -- (0,-1) -- (1,0) -- cycle
      (0,0) coordinate[vertex, fill=ks-true] node[left]{\(s_\True\)}
      (1,0) coordinate[vertex, fill=ks-false] node[right]{\(s_\False\)}
      (0,-1) coordinate[vertex, fill=ks-aux] node[left]{\(s_\Aux\)};
    },
    semi-or graph/.pic={
      \coordinate[vertex](i1);
      \coordinate[vertex](i2) at (0,-1);
      \coordinate[vertex](i1') at (1,0);
      \coordinate[vertex](i2') at (1,-1);
      \coordinate[vertex](t) at (2,0);
      \coordinate[vertex](a) at (0,-1/2);

      \node[vertex label, left] at (i1){\(i_1\)};
      \node[vertex label, left] at (i2){\(i_2\)};
      \node[vertex label, left] at (a){\(s_\Aux\)};
      \node[vertex label, right] at (t){\(s_\True\)};

      \draw (i2') -- (i2) -- (a) -- (i1) -- (i1') -- (t) -- (i2') -- (i1');
    },
    and-or graph/.pic={
      \coordinate[vertex](i1) at (-1,-1);
      \coordinate[vertex](ai) at (0,-3/2);
      \coordinate[vertex](i2) at (-1,-2);

      \coordinate[vertex](i1') at (1,1);
      \coordinate[vertex](i2') at (1,0);

      \coordinate[vertex](i') at (2,1);
      \coordinate[vertex](a') at (2,0);

      \coordinate[vertex](n1) at (1,-1);
      %\coordinate[vertex](an) at (1,-3/2);
      \coordinate[vertex](n2) at (1,-2);

      \coordinate[vertex](no) at (5,1);
      \coordinate[vertex](o) at (6,0);
      \coordinate[vertex](ao) at (6,1);

      \coordinate[vertex](i'') at (3,1);
      \coordinate[vertex](n1') at (2,-1);
      \coordinate[vertex](n2') at (2,-2);

      \coordinate[vertex](no') at (4,1);
      \coordinate[vertex](o1') at (4,-1);
      \coordinate[vertex](o2') at (4,-2);

      \coordinate[vertex](to) at (3,0);
      \coordinate[vertex](ti) at (3,-3/2);

      \node[vertex label, below] at (ai) {\(s_\Aux\)};
      \node[vertex label, below] at (a') {\(s_\Aux\)};

      \node[vertex label, below] at (to) {\(s_\True\)};
      \node[vertex label, below] at (ti) {\(s_\True\)};

      \node[vertex label, above] at (ao) {\(s_\Aux\)};

      \node[vertex label, above] at (i') {\(i'\)};


      \draw
      (i1) -- (ai) -- (i2) (n1) -- (ai) -- (n2)
      (i1) -- (n1) -- (n1') -- (o1') -- (o) (n1') -- (ti) -- (o1')
      (i2) -- (n2) -- (n2') -- (o2') -- (o) (n2') -- (ti) -- (o2')
      %(n1) -- (an) -- (n2)
      (o) -- (no) -- (ao) -- (o)
      (i1) -- (i1') -- (i2') -- (i') -- (i1')
      (a') -- (i') -- (i'') -- (no') -- (no) (no') -- (to) -- (i'');

      \draw[over] (i2) -- (i2');


    },
    or graph/.pic={

      \pic{and-or graph};
      \node[vertex label, left] at (i1) {\(i₁\)};
      \node[vertex label, left] at (i2) {\(i₂\)};
      \node[vertex label, below] at (n1) {\(¬i₁\)};
      \node[vertex label, below] at (n2) {\(¬i₂\)};
      \node[vertex label, right] at (o) {\(o\)};
      \node[vertex label, above] at (no) {\(¬o\)};

    },
    and graph/.pic={

      \pic{and-or graph};
      \node[vertex label, left] at (i1) {\(¬i₁\)};
      \node[vertex label, left] at (i2) {\(¬i₂\)};
      \node[vertex label, below] at (n1) {\(i₁\)};
      \node[vertex label, below] at (n2) {\(i₂\)};
      \node[vertex label, right] at (o) {\(¬o\)};
      \node[vertex label, above] at (no) {\(o\)};

    },
    not graph/.pic={
      \coordinate[vertex](i);
      \coordinate[vertex](o) at (1,0);
      \coordinate[vertex](a) at (0,-1);

      \node[vertex label, left] at (a) {\(s_\Aux\)};
      \node[vertex label, left] at (i) {\(i₁\)};
      \node[vertex label, right] at (o) {\(o\)};
      \draw (i) -- (o) -- (a) -- (i);
    },
  }

  We demonstrate constructions of graphs computing each of the boolean
  operations.

  To improve readability, we adopt the following conventions in the
  illustrations below:
  \begin{itemize}
    \item Assume implicitly the presence of special vertices
      \(s_\True,s_\False,s_\Aux\) joined by a triangle.  We omit them from the
      diagrams, using them only when needed.
    \item To avoid excessive edge crossings, we sometimes illustrate one vertex
      as multiple ``duplicate'' vertices with the same label.
  \end{itemize}

  \begin{description}

  \item[\NOT] The following graph computes the boolean \NOT{} operation:

    \begin{center}
      \begin{tikzpicture}[boolean graph]
        \pic{not graph};
      \end{tikzpicture}
    \end{center}

    Since \(i₁\) neighbors \(o\) (and both neighbor \(s_\Aux\)), they
    necessarily have opposite colors.  Below we show colorings for both
    possible input values (unique up to permutation of colors):

    \[
      \begin{array}{c|c}
        x₁=\True & x₁=\False \\ \midrule
        \begin{tikzpicture}[boolean graph]
          \pic{not graph};
          \coordinate[vertex, fill=ks-true]() at (i);
          \coordinate[vertex, fill=ks-false]() at (o);
          \coordinate[vertex, fill=ks-aux]() at (a);

          \pic at (3,0) {triangle};
        \end{tikzpicture}
        &
        \begin{tikzpicture}[boolean graph]
          \pic{not graph};
          \coordinate[vertex, fill=ks-false]() at (i);
          \coordinate[vertex, fill=ks-true]() at (o);
          \coordinate[vertex, fill=ks-aux]() at (a);

          \pic at (3,0) {triangle};
        \end{tikzpicture}
      \end{array}
    \]

  \item[\OR] Our construction of the boolean \OR{} gate is slightly
    complicated.  To that end, before giving that construction, we first
    introduce a \emph{helper} graph, which we call the ``semi-\OR'' graph.

    \begin{aside}

      \begin{center}
        \begin{tikzpicture}[boolean graph]
          \pic{semi-or graph};
        \end{tikzpicture}
      \end{center}

      Notice that this graph does not yet define a boolean graph, because it has
      no output vertex.  However, it has some useful properties resembling that
      of an \OR-gate.  Examine each of the possible input combinations:
      \begin{itemize}
        \item When \(i_1\) and \(i_2\) are both assigned \True, a proper
          coloring exists:
          \begin{center}
            \begin{tikzpicture}[boolean graph]
              \pic{semi-or graph};
              \coordinate[vertex, fill=ks-true]() at (i1);
              \coordinate[vertex, fill=ks-true]() at (i2);
              \coordinate[vertex, fill=ks-false]() at (i1');
              \coordinate[vertex, fill=ks-aux]() at (i2');
              \coordinate[vertex, fill=ks-aux]() at (a);
              \coordinate[vertex, fill=ks-true]() at (t);

              \pic at (4,0){triangle};
            \end{tikzpicture}
          \end{center}

        \item When \(i_1\) and \(i_2\) are both assigned \False, then no proper
          coloring exists:
          \begin{center}
            \begin{tikzpicture}[boolean graph]
              \pic{semi-or graph};
              \coordinate[vertex, fill=ks-false]() at (i1);
              \coordinate[vertex, fill=ks-false]() at (i2);
              %\coordinate[vertex, fill=ks-false]() at (i1');
              %\coordinate[vertex, fill=ks-aux]() at (i2');
              \coordinate[vertex, fill=ks-aux]() at (a);
              \coordinate[vertex, fill=ks-true]() at (t);
              %\node[above] at (i1') {\(i₁'\)};
              %\node[below] at (i2') {\(i₂'\)};

              \pic at (4,0) {triangle};
            \end{tikzpicture}
          \end{center}

          Each of the two uncolored vertices neighbor \(s_\True\) and an input
          vertex \(i₁\) or \(i₂\), whose color matches \(s_\False\).  Thus the
          uncolored vertices would have to be colored same as \(s_\Aux\).
          However, they neighbor each other as well, forcing them to share a
          color.  Thus there is no proper coloring.

        \item When exactly one of \(i_1,i_2\) is assigned \True{} and the other
          \False, then a proper coloring exists:
          \begin{center}
            \begin{tikzpicture}[boolean graph]
              \pic{semi-or graph};
              \coordinate[vertex, fill=ks-true]() at (i1);
              \coordinate[vertex, fill=ks-false]() at (i2);
              \coordinate[vertex, fill=ks-false]() at (i1');
              \coordinate[vertex, fill=ks-aux]() at (i2');
              \coordinate[vertex, fill=ks-aux]() at (a);
              \coordinate[vertex, fill=ks-true]() at (t);
            \end{tikzpicture}
          \end{center}

    \end{itemize}
    Together, these observations reveal that the semi-\OR{} graph does not
    \emph{compute} the \OR{} operation, but it has the property of being
    \emph{properly-3-colorable} if and only if \(i_1\) or \(i_2\) is assigned
    \True.


  \end{aside}

  Now, we are ready to construct a (``full'') \OR{} graph:

  \begin{center}
    \begin{tikzpicture}[boolean graph]
      \pic{or graph};
    \end{tikzpicture}
  \end{center}

  %\todo[inline]{i made two versions of this graph diagram---the above has less
  %  vertex duplication and is laid out more like a ``gate'', but the below
  %  graph more explicitly/clearly illustrates the parts and may be easier to
  %  explain, at the expense of many more duplicated vertices. @nick, do you
  %find one nicer than the other?}

  \begin{center}
    \begin{tikzpicture}[boolean graph]

      \coordinate[vertex](i1);
      \coordinate[vertex](i2) at (0,-1);
      \coordinate[vertex](a) at (0,-1/2);
      \coordinate[vertex](i') at (2,0);
      \coordinate[vertex](a') at (2,-1/2);
      \coordinate[vertex](no') at (2,-1);
      \coordinate[vertex](no'') at (3,-1);
      \coordinate[vertex](i'') at (3,0);
      \coordinate[vertex](to) at (4,0);

      \coordinate[vertex](in1) at (0,-2);
      \coordinate[vertex](in2) at (0,-4);

      \coordinate[vertex](on) at (0,-6);
      \coordinate[vertex](no) at (1,-6);
      \coordinate[vertex](ao) at (0,-7);


      \foreach \i in {1,2} {
        \coordinate[vertex](i\i') at ($ (i\i) + (1,0) $);
        \node[vertex label, left] at (i\i) {\(i_{\i}\)};
        \node[vertex label, left] at (in\i) {\(i_{\i}\)};

        \coordinate[vertex](n\i) at ($ (in\i) + (1,0) $);
        \coordinate[vertex](a\i) at ($ (in\i) + (0,-1) $);
        \coordinate[vertex](n\i') at ($ (n\i) + (1,0) $);
        \coordinate[vertex](an\i) at ($ (n\i) + (0,-1/2) $);
        \coordinate[vertex](o\i) at ($ (n\i) + (0,-1) $);
        \coordinate[vertex](o\i') at ($ (o\i) + (1,0) $);
        \coordinate[vertex](t\i) at ($ (n\i') + (1,0) $);

        \node[vertex label, above] at (n\i) {\(¬i_{\i}\)};
        \node[vertex label, left] at (o\i) {\(o\)};
        \node[vertex label, right] at (an\i) {\(s_\Aux\)};
        \node[vertex label, right] at (t\i) {\(s_\True\)};

        \draw
          (in\i) -- (n\i) -- (a\i) -- (in\i)
          (n\i) -- (an\i) -- (o\i) -- (o\i') -- (n\i') -- (n\i)
          (n\i') -- (t\i) -- (o\i');
      }

      \foreach \v in {a,a1,a2,ao} {
        \node[vertex label, left] at (\v) {\(s_\Aux\)};
      }
      \node[vertex label, right] at (a') {\(s_\Aux\)};

      \node[vertex label, above] at (i') {\(i'\)};
      \node[vertex label, left] at (on) {\(o\)};
      \node[vertex label, left] at (no') {\(¬o\)};
      \node[vertex label, right] at (no) {\(¬o\)};

      \node[vertex label, right] at (to) {\(s_\True\)};

      \draw
        (i1) -- (a) -- (i2) -- (i2') -- (i1') -- (i1)
        (i1') -- (i') -- (i2')
        (i') -- (a') -- (no') -- (no'') -- (i'') -- (i')
        (i'') -- (to) -- (no'')
        (on) -- (no) -- (ao) -- (on);

    \end{tikzpicture}
  \end{center}

  To see why this construction works,

  %\todo{still not done}




\item[\AND] To construct an \AND{} gate, we apply DeMorgan's law to rewrite
  \AND{} in terms of \NOT{} and \OR:
  \[
    y=x₁∧x₂ ⟺ ¬y=¬x₁∨¬x₂.
  \]

  To that end, we implement an \AND{} graph by negating both the input
  vertices and the output vertex in the \OR{} graph.  We do so by swapping
  vertices
  \[
    i₁↔¬i₁, \qquad i₂↔¬i₂, \qquad o↔¬o
  \]
  in the \OR{} graph construction:

  \begin{center}
    \begin{tikzpicture}[boolean graph]
      \pic{and graph};
    \end{tikzpicture}
  \end{center}


\end{description}

\end{proof}

\begin{theorem}{}{circuit-to-graph}%
  For any boolean circuit \(C\), there exists a boolean graph that computes
  [the function defined by] \(C\).  Moreover, there exists a polynomial-time
  algorithm that performs this conversion from boolean circuits to graphs.
\end{theorem}

\begin{proof}
  We describe an algorithm that, given a circuit \(C\), generates a boolean
  graph computing \(C\).

  First, construct the special triangle \(s_\True,s_\False,s_\Aux\).

  For each wire in \(C\) (including the input and output wires), create a
  corresponding vertex and join it by an edge to \(s_\Aux\). (The color of this
  vertex will correspond to the value carried by the wire.)

  Next, for each [\NOT, \OR, and \AND] gate \(g\) in \(C\), apply
  \cref{lem:boolean-operation-graphs} to construct a subgraph \(γ\) computing
  \(g\).  The input vertices of \(γ\) should exactly correspond to the input
  wires of \(g\), and the output vertex of \(γ\) should correspond to the
  output wire of \(g\).

  %\todo{not done, need some induction to make precise}

\end{proof}

\begin{example}{}{}
  Conversion of an example circuit (e.g., XOR gate?  or something simpler, like
  \(x₁∧¬x₂\)) to graph.

  TODO
\end{example}

\subsection{Graph-colorability}

\begin{definition}{The graph 3-colorability problem (\Problem{3col})}{}

  The \Term{graph 3-colorability problem} is stated as the following yes/no
  question: given a graph, is it (properly) 3-colorable?

\end{definition}

\begin{theorem}{}{}
  \Problem{3col} is \NP-complete.
\end{theorem}

\begin{proof}
  It is straightforward to see that \(\Problem{3col}∈\NP\), since it is
  solvable by polynomial-time guess-and-check: guess a color assignment for
  each vertex, then verify that, for each edge \(e\), the two vertices joined
  by \(e\) have distinct colors.  The ``check'' procedure takes
  \(\O(\Abs{\Edges[G]})\) time, which is polynomial with respect to the size of
  the graph.

  To show that \(\Problem{3col}\) is \NP-hard, we show
  \(\CircSat≤\Problem{3col}\) by reducing \CircSat{} to \Problem{3col}.
  Specifically, given a circuit \(C\) with inputs \(x₁,\dotsc,xₙ\), we wish to
  construct a corresponding graph \(G\) such that \(G\) is 3-colorable if and
  only if \(C\) is satisfiable.

  Let such a circuit \(C\) be given.  Apply the algorithm from
  \cref{th:circuit-to-graph} to produce a graph \(G\) that computes the boolean
  function \(C\), such that the inputs \(x₁,\dotsc,xₙ\) of \(C\) correspond
  respectively to input vertices \(i₁,\dotsc,iₙ\), and the output of \(C\)
  corresponds to the output vertex \(o\).

  Force the output vertex of \(G\) to take on the boolean value \(\True\) by
  ``merging'' vertices \(o\) and \(s_\True\) into one vertex, keeping all of
  their connections to other vertices.  (Equivalently, if we wish to avoid such
  a ``merging'' operation, we may construct \(G\) assuming \emph{a priori} that
  \(o=s_\True\).  Yet another alternative is to create edges joining \(o\) with
  \(s_\False\) and \(s_\Aux\), thereby forcing it to share a color with
  \(s_\True\).)

  We claim that this resulting graph \(G\) is 3-colorable if and only if \(C\)
  is satisfiable.
  \begin{itemize}
    \item[(\(⟹\))] Suppose that \(G\) is 3-colorable.  Then let \(κ\) be a
      proper 3-coloring of \(G\).  Let \(x₁,\dotsc,xₙ\) be the respective
      boolean values assigned by \(κ\) to the input vertices \(i₁,\dotsc,iₙ\)
      of \(G\). Since \(G\) computes \(C\)
      (\cref{def:boolean-graph-functions}), we know that
      \(κ(o)=C(x₁,\dotsc,xₙ)\).  At the same time, we have also ensured by
      construction that \(κ(o)=\True\).  Thus \(x₁,\dotsc,xₙ\) is a satisfying
      assignment for \(C\).  Thus \(C\) is satisfiable.

    \item[(\(⟸\))] Suppose that \(C\) is satisfiable.  Then let
      \(x₁,\dotsc,xₙ\) be a satisfying assignment for \(C\).  We construct a
      3-coloring for \(G\) as follows:
      \begin{itemize}
        \item Arbitrarily color the special vertices
          \(s_\True,s_\False,s_\Aux\).  Call those colors \(\True,\False,\Aux\)
          respectively.
        \item For each input TODO unfinished
      \end{itemize}


  \end{itemize}


\end{proof}



\subsection{Graph-coloring games}



\section{Set covering games}
