\chapter{A primer on boolean logic}
\label{ch:boolean}

Mathematical logic is founded on true-or-false statements---statements such as:
\begin{itemize}[nosep]
  \item property \(A\) is \emph{true} when condition \(B\) is \emph{false},
  \item property \(X\) is \emph{true} when both condition \(Y\) and condition
    \(Z\) are \emph{true},
\end{itemize}
and so on.  Boolean logic refers to the algebra of how \emph{truthiness} and
\emph{falsiness} combine and transform under various logical operations.

It is no surprise, given the foundational role of booleans in mathematical
logic, that they also underpin all computational logic. For instance, all
modern computer architectures deal with data encoded in binary \(0\)s
(\emph{false}) and \(1\)s (\emph{true}).  Furthermore, it follows that
everything we conceive of as ``computer'' can be represented as boolean
circuits---because, essentially, they literally are boolean circuits.

% TODO: this observation underlies importance of boolean puzzles/games, etc.; how this comes up later, blah

In this short chapter, we outline some basic definitions and facts about
boolean-logical operations and circuits, along with some notational conventions
used throughout the rest of this thesis.

\begin{definition}{basic boolean operations: \NOT, \AND, \OR}{}

  \begin{description}

  \item[\NOT] takes one input and outputs its opposite value.  In
    boolean-algebraic expressions, we denote \NOT{} with the symbol \(¬\).
    \[
      ¬\colon\Set{\True,\False}→\Set{\True,\False} \qquad
      ¬x =
      \begin{cases}
        \True & x=\False \\
        \False & x=\True
      \end{cases}.
    \]
    The \NOT{} operation is also commonly known as \Term{negation}.

  \item[\AND] takes two inputs and outputs \True{} if and only if \emph{both}
    of its inputs are \True.  We denote \AND{} with the symbol \(∧\).
    \[
      ∧\colon\Set{\True,\False}²→\Set{\True,\False} \qquad
      x∧y =
      \begin{cases}
        \True & x=y=\True \\
        \False & \text{otherwise}
      \end{cases}.
    \]

    For convenience, we sometimes omit the \(∧\) and simply denote \AND{} by
    concatenating the operands, as in \(xy\) instead of \(x∧y\).  (This
    notation looks like multiplication because it is: if we represent boolean
    values with \(\Set{1,0}\) instead of \(\Set{\True,\False}\), then
    \(x∧y=x⋅y\).)

    \AND{} is also known as the \Term{conjunction} operation.

  \item[\OR] takes two inputs and outputs \True{} if \emph{at least one} of its
    inputs are \True.  We denote \OR{} with the symbol \(∨\).
    \[
      ∨\colon\Set{\True,\False}²→\Set{\True,\False} \qquad
      x∨y =
      \begin{cases}
        \False & x=y=\False \\
        \True & \text{otherwise}
      \end{cases}.
    \]

    \OR{} is also known as the \Term{disjunction} operation.

  \end{description}

  Notationally, \(∧\) takes higher precedence than \(∨\).  For instance, we
  interpret \(x∨y∧z=x∨yz=x∨(y∧z)\), and so on.

  \begin{aside}
    I personally find the \(∧\) and \(∨\) symbols for \AND{} and \OR{} quite
    easy to mix up with each other.  Here's a mnemonic that helps me remember
    which is which:
    \begin{itemize}[nosep]
      \item \(∧\) looks like the \(\mathrm{\scriptstyle A}\) in \AND, so \(∧\)
        means \AND…
      \item \(∨\) is the other one.
    \end{itemize}
  \end{aside}

\end{definition}

\section{Algebraic properties of \(¬,∧,∨\)}

What algebraic behaviors do \(¬\), \(∧\), and \(∨\) exhibit?

\paragraph{Commutativity \& associativity} It follows straightforwardly from
their definitions that they are both commutative and associative.  In general,
for any \(x₁,x₂,\dotsc,xₙ∈\Set{\True,\False}\),
\begin{align*}
  ⋀ᵢ₌₁ⁿ xᵢ &= x₁∧\dotsb∧xₙ = \text{\True{} if and only if \emph{every one} of
  \(x₁,\dotsc,xₙ\) is \True}, \\
  ⋁ᵢ₌₁ⁿ xᵢ &= x₁∨\dotsb∨xₙ = \text{\True{} if and only if \emph{at least one} of \(x₁,\dotsc,xₙ\) is \True}.
\end{align*}

\paragraph{Distributivity} Another interesting, sometimes useful, property of
\(∧\) and \(∨\) is that they distribute over each other.  For all
\(x,y,z∈\Set{\True,\False}\),
\[
  x∧(y∨z) = (x∧y)∨(x∧z), \qquad
  x∨(y∧z) = (x∨y)∧(x∨z).
\]

%\begin{aside}
%  Here are two intuitive examples demonstrating this distributivity.
%
%  \begin{itemize}
%
%    \item Consider the statement, ``Alex and (either Blake or Charlie) ate
%      pizza'', encoded as the boolean statement \(A(B∨C)\).  What are the
%      possible combinations of pizza-eaters?
%
%      The answer: either Alex and Blake, or Alex and Charlie.  That is,
%      \[
%        A(B∨C) = AB∨AC.
%      \]
%
%    \item Consider the statement, ``either Alex is a vegetarian, or Charlie and
%      Blake both are''.
%
%  \end{itemize}
%
%
%\end{aside}

%\section{Computing arbitrary boolean functions}
%
%Any boolean function can be expressed in terms of the three operators
%\(¬,∧,∨\).


\subsection{DeMorgan's identities}

Consider the statement, ``\(x,y\) are both \False''.  There are two equivalent
ways to express this statement algebraically:
\begin{itemize}[nosep]
  \item \(x\) is \False, and \(y\) is \False: \(¬x∧¬y\).
  \item Neither of \(x,y\) is \True: \(¬(x∨y)\).
\end{itemize}
The equivalence of these two expressions gives rise to an identity: for all
\(x,y∈\Set{\True,\False}\),
\[
  ¬x∧¬y = ¬(x∨y).
\]

Similarly, the statement ``at least one of \(x,y\) are \False'' can be
expressed in two ways,
\begin{itemize}[nosep]
  \item \(x\) is \False, or \(y\) is \False: \(¬x∨¬y\).
  \item \(x,y\) are not both simultaneously \True: \(¬(x∧y)\).
\end{itemize}
This equivalence gives rise to a dual identity,
\[
  ¬x∨¬y = ¬(x∧y).
\]

A particularly useful consequence of these DeMorgan identities is that having
all three logical operations is \emph{redundant}.  We didn't need to define all
three as the basic building-block operations; having only \NOT/\OR{} or
\NOT/\AND{} suffices, since the third operation can simply be constructed in
terms of the other two:
\[
  x∧y = ¬(¬x∨¬y), \qquad
  x∨y = ¬(¬x∧¬y).
\]

We make use of this convenience later in chapters TODO, when we try to embed
boolean logic within other ``computer-like'' systems such as graph colorings
and exact set coverings, etc.

% The usefulness of this redundancy
%
% that if we were trying to simulate boolean logic within some other system (we
% explore this idea later in more detail when we explore reductions from boolean
% circuits in chapters TODO),
%
% universality of not/and and not/or

% TODO miscellaneous identities?


% TODO boolean logical operations can compute arbitrary boolean functions

\section{Boolean circuits}

Boolean \emph{expressions} such as \(¬x∧y\) are one way to specify computations
on boolean variables.  \emph{Circuits} generalize expressions by essentially
chaining together a pipeline of expressions, allowing intermediate results at
each stage to be saved and reused.  To illustrate, consider the following
example expression:
\[
  ϕ(x₁,x₂,y₁,y₂,z₁,z₂)
  = (x₁∨x₂)(y₁∨y₂) ∨ (y₁∨y₂)(z₁∨z₂) ∨ (z₁∨z₂)(x₁∨x₂).
\]
Notice that each \((□₁∨□₂)\) term appears twice in the expression, making the
expression inefficient to evaluate (each repeated term would be unnecessarily
recomputed), not to mention cumbersome to specify.  A more elegant way to
specify this computation is to store and reuse the intermediate terms:
\begin{align*}
  X &= x₁∨x₂, \\
  Y &= y₁∨y₂, \\
  Z &= z₁∨z₂, \\
  ϕ &= XY∨YZ∨ZX.
\end{align*}
This chain of assignments may be visualized as a sort of data-processing
``pipeline'', with intermediate inputs and outputs at each stage:

{

  \tikzset{
    input/.style={
      circle,
      fill,
      inner sep=0pt,
      minimum size=3pt,
    },
    gate/.style={
      draw,
      rounded corners=1em/8,
    },
    pipe/.style={
      rounded corners=1em/2,
      to path={
        (\tikztostart)
        -- ($ (\tikztostart -| \tikztotarget)!3em!(\tikztostart) $)
        %-- ($ (\tikztotarget)!1em!(\tikztostart |- \tikztotarget) $)
        -- (\tikztotarget)
      },
      ->,
    },
    over/.style={
      preaction={
        draw=white,
        line width=2pt,
      },
    },
    gates/.style={
      row sep=2em, column sep=6em, matrix of math nodes, nodes=gate,
    },
    wires/.pic={

      \foreach \var in {x,y,z} {
        \coordinate (\var1) at ($ (\var.west) + (-4em,2em/3) $);
        \coordinate (\var2) at ($ (\var.west) + (-4em,-2em/3) $);
        \draw[pipe] (\var1) node[left]{\(\var₁\)} to ($ (\var.north west)!2/3!(\var.west) $);
        \draw[pipe] (\var2) node[left]{\(\var₂\)} to ($ (\var.south west)!2/3!(\var.west) $);
        \node[above right] at (\var.east) {\(\MakeUppercase{\var}=\var₁∨\var₂\)};
      }
      \draw[pipe] (y.east) to (xy.south west);
      %\draw[pipe] ($ (yz.west)!3em!(y.east) $) to ($ (xy.west)!1/2!(xy.south west) $);
      \draw[pipe] (y.east) to (yz.west);
      %\draw[pipe] ($ (zx.west)!3em!(z.east) $) to ($ (yz.west)!1/2!(yz.south west) $);
      \draw[pipe] (z.east) to (yz.south west);
      \draw[pipe] (z.east) to (zx.west);
      %\draw[pipe] ($ (zx.west)!3em!(z.east) $) to ($ (yz.west)!1/2!(yz.south west) $);
      \draw[pipe, over] (x.east) to (zx.north west);
      \draw[pipe] (x.east) to (xy.west);

      \foreach \gate in {xy,yz,zx} {
        \node[above right] at (\gate.east) {\(\MakeUppercase{\gate}\)};
      }

    },
  }

  \begin{center}
    \begin{tikzpicture}
      \matrix[gates, ampersand replacement=\&]{
        |(x)|∨ \&[2em] |(xy)|∧ \\
        |(y)|∨ \& |(yz)|∧ \& |(out)|∨ \\
        |(z)|∨ \& |(zx)|∧ \\
      };

      \pic{wires};
      \draw[pipe] (xy.east) to (out.north west);
      \draw[pipe] (yz.east) to (out.west);
      \draw[pipe] (zx.east) to (out.south west);
      \draw[->] (out.east) -- +(2em,0) node[right]{\(ϕ=XY∨YZ∨ZX\)};
    \end{tikzpicture}
  \end{center}

  This is \emph{essentially} a boolean circuit.  More precisely, in a boolean
  circuit, each variable (e.g., \(x₂\) or \(Y\)) is represented as a \emph{wire}
  carrying a boolean value, and each ``stage'' of computation, called a
  \emph{logic gate}, computes an individual boolean operation.

  For simplicity's sake, we also require that each \AND/\OR{} gate operates on
  exactly two inputs.  Thus the last \OR{} operation \(XY∨YZ∨ZX\) should
  actually be associatively grouped as \((XY∨YZ)∨ZX\).  The corrected circuit
  is shown below:

  \begin{center}
    \begin{tikzpicture}
      \matrix[gates, ampersand replacement=\&]{
        |(x)|∨ \&[2em] |(xy)|∧ \\
        |(y)|∨ \& |(yz)|∧ \&\& |(or')|∨ \\
        |(z)|∨ \& |(zx)|∧ \\
      };

      \node[gate](or) at ($ (xy)!1/2!(or') $){\(∨\)};

      \pic{wires};
      \draw[pipe] (xy.east) to ($ (or.west)!1/3!(or.north west) $);
      \draw[pipe] (yz.east) to ($ (or.west)!1/3!(or.south west) $);
      \draw[pipe] (or.east) to ($ (or'.west)!1/3!(or'.north west) $);
      \draw[pipe] (zx.east) to ($ (or'.west)!1/3!(or'.south west) $);
      \node[above right] at (or.east) {\(XY∨YZ\)};
      \draw[->] (or'.east) -- +(2em,0) node[right]{\(ϕ=(XY∨YZ)∨ZX\)};

    \end{tikzpicture}
  \end{center}

}

%Lastly, we use the algebraic symbols \(∨,∧\) to denote the logic gates in the
%circuit diagram above

Below, we give a precise definition of boolean circuits and introduce some
relevant terminology.

\begin{definition}{boolean circuits, logic gates}{}

  A \Term{boolean circuit}

  TODO precise definition

\end{definition}


%We give a precise definition of
%
%\begin{definition}{boolean circuits}{}
%
%  % TODO consider using switches and lightbulbs in formalism
%
%  A \Term{circuit} consists of a network of such wires and logic gates, with
%  the following requirements for well-defined-ness:
%  \begin{itemize}
%    \item Some wires are marked as circuit-level \Term{inputs}.  These wires
%      may not be the output wire of any gate within the circuit.
%    \item Every wire that isn't a circuit-level input must be the output wire
%      of exactly one logic gate.
%    \item There must be no (directed) cycles in the circuit.  That is, there
%      must not exist gates \(g₁,\dotsc,gₖ\) such that the output of \(g₁\) is
%      an input to \(g₂\), the output of \(g₂\) an input to \(g₃\), …, and the
%      output of \(gₖ\) an input to \(g₁\).
%    \item Finally, there is exactly one wire marked the \Term{output} of the
%      circuit.
%  \end{itemize}
%  Under these requirements, each combination of boolean values assigned to the
%  circuit-level input wires, uniquely determines the circuit-level output's
%  value. Therefore, a circuit with \(n\) inputs defines a function
%  \(\Set{\False,\True}ⁿ→\Set{\False,\True}\).
%
%  To simplify notation, we refer to each circuit and its boolean function by
%  the same name.  That is, if \(C\) refers to a circuit, we denote by
%  \(C(x₁,\dotsc,xₙ)\) the output computed by \(C\) on input values
%  \(x₁,\dotsc,xₙ\).  Occasionally, if we need to disambiguate between a circuit
%  and its function, we denote the circuit \(C\) and its function \(ϕ_C\).
%
%\end{definition}

