\chapter{Future work}

\label{ch:future}

I've discussed most of the roadblocks, challenges, and open
questions/conjectures I've encountered.  Here, I will merely summarize what I
think are the main next steps \& components of the work for next semester:

\begin{itemize}

  \item Figure out an elegant \CSAT{} to \Problem{3col} reduction.  In the
    worst case, I might not be able to come up with anything better than what
    already exists, which is
    \[
      \CSAT \le \SAT \le \Problem{cnfsat} \le \Problem{3col}.
    \]
    But still, it is worth trying.

    Another useful source on this exploration is
    \textcite{ddls.sat-variants-ph}, which explores how many other variants of
    \SAT{} are placed in the polynomial hierarchy.

  \item Figure out a good \emph{game} generalization of \Problem{3col} that,
    ideally, does not make the game trivial for the adversarial player, but
    also minimizes ``special-case'' treatment.  Prove that these games are
    complete for \SigmaP k, \PiP k, etc.

  \item Move on to other puzzles, such as Sudoku, or
    \Problem{hamiltonian-path}, or really any of the famous 21 \NP-complete
    problems discussed in \textcite{karp.np}.  Wonder if there is a
    generalizable scheme/technique by which any \NP-complete puzzle can be
    lifted up in the polynomial hierarchy, one step at a time.

\end{itemize}

Finally, if all that is done (ha, as if!), there are a few more topics, still
in the theme of ``converting puzzles to games'', that I have preliminarily
explored and found sources on:
\begin{itemize}

  \item \label{itm:intro.q.generation} Puzzle generation.  If I wish to solve a
    puzzle, you can play a game with me by constructing puzzle \emph{instances}
    for me to solve.  For instance, \emph{solving} Sudoku is an \NP-complete
    problem; your task is to \emph{generate} (partially-filled) Sudoku boards
    for me to solve.

    How hard is it to do so?  Moreover, how hard is it to generate \emph{good}
    puzzle instances, for various definitions of \emph{good} (sufficiently
    challenging to solve, or having unique solutions, or solvable/unsolvable by
    certain strategies)?

    Most of the formal theoretical work in this topic has been done by
    \textcite{language-instances,test-gen-complexity,hard-diverse-graph-tests,maximum-clique-generators}.
    A few other interesting, relevant snippets on other aspects of the
    generation problem (e.g., uniqueness, strategy-solvability, difficulty
    assessments) are found in \textcite{strategy-solvable-sudoku,sudoku-education,unique-sudoku-poly,difficulty-driven-sudoku,sudoku-difficulty-oracle} (and many more).

  \item \label{itm:intro.q.pspace} \PSPACE-complete games with arbitrarily many
    (or, at least, polynomially-many) moves.  In our definitions of the \CSAT{}
    games, the number of moves is \emph{fixed}.  However, in many real life
    games such as Chess, Checkers, Go, Reversi, etc., the number of moves
    scales with the size of the board, allowing the complexity of the problem
    to also scale accordingly.  Thus, these games stretch beyond the polynomial
    hierarchy into \PSPACE{} (polynomial space), which is suspected, but not
    proven, to be strictly more complex than the polynomial hierarchy.  Several
    \PSPACE-complete games can be derived from \NP-complete puzzles, such as
    graph coloring.  For instance, in one such game, on a given graph, two
    players alternate turns assigning colors to individual vertices at a time,
    never introducing an improper pair; the first player to have no available
    moves loses.

    Can other \NP-complete puzzles be similarly generalized into games?  Will
    those games also be \PSPACE-complete?

    Specific examinations of graph coloring games like the one discussed above
    are found in
    \textcite{bodlaender.coloring,bh.placement,kbd.impartial,cpss.coloring}.
    Countlessly many more games have been studied and found \PSPACE-complete;
    useful summary-style discussions of these results may be found in
    \textcite{eppstein.cgt,demaine.acgt}.

    Finally, a \emph{general} theoretical treatment of lifting \NP-complete
    puzzles to \PSPACE-complete games, in similar spirit to Karp's original
    \NP-complete categorization, has been attempted in
    \textcite{schaefer.deriving,schaefer.games}.

\end{itemize}




