\chapter{Introduction}

The basic question of computational complexity---``how hard is this problem for
a computer to solve?''---is central to nearly every topic in computer science.
And yet, the formalisms of complexity theory often seem, in my own experience,
intimidatingly abstract, phrased in terms of intangible models of computation
such as non-deterministic Turing machines and oracles.

The remedy, I believe, lies in studying complexity theory through the lens of
\emph{puzzles} and \emph{games}.  Not only do they provide a concrete grounding
for the abstractions, they also offer a particularly insightful, accessible,
and most importantly fun approach to understanding complexity theory.  In fact,
many of the most popularly known and appreciated results in complexity theory
are those about so-called ``\NP-complete puzzles'', such as Sudoku, and
``\PSPACE-complete games'', such as Checkers and Go.

This thesis emphasizes that approach in its exploration of a particularly
foundational, yet often overlooked, ladder of complexity classes known as the
\emph{polynomial hierarchy}.  If \NP{} is the class of (one-player)
``puzzles'', and \PSPACE{} the class of (two-player) ``games'', then the
polynomial hierarchy encompasses everything in between.  And the (in)famous
\P-vs-\NP{} question is but the first in a ladder of questions that are,
arguably, just as crucial and impactful.

%The polynomial hierarchy is as central to
%complexity theory as the \P-vs-\NP{} problem is well-known.

TODO: outline/overview of chapters, after those chapters are written

%p-vs-np well known, polynomial-hierarchy central

%puzzles and games; hierarchy lies in the interstices.  we examine a few
%interesting (by no means exhaustive, or even close to comprehensive)
%np-complete puzzles with pspace-complete analogues, and we









%Famously central to the theory of computational complexity is the \P-vs-\NP{}
%question, and essential to our understanding of that question is the study of
%\NP-complete problems such as the Boolean Satisfiability puzzle, the Graph
%Colorability puzzle, and countless more.  Puzzles like these, which nearly any
%layperson can appreciate, offer a particularly insightful, intuitive, and
%\emph{fun} lens through which to study computational complexity. Explorations
%of more complex problem-classes such as \PSPACE{} can be similarly approached
%through the study of strategic decision \emph{games} such as Othello, Checkers,
%and Go.
%
%What lies in the interstices between \emph{puzzles} and \emph{games}?  How do
%we take a puzzle and generalize it into a game, and what are the puzzle-games
%we encounter along the way?  And how hard exactly are these puzzle-games to
%decide?  These questions are the focus of my thesis.

%So far, I have explored these questions from three angles:
%\begin{enumerate}
%
%  \item \label{itm:intro.q.generation} Puzzle generation.  If I wish to solve a
%    puzzle, you can play a game with me by constructing puzzle \emph{instances}
%    for me to solve.  For instance, \emph{solving} Sudoku is an \NP-complete
%    problem; your task is to \emph{generate} (partially-filled) Sudoku boards
%    for me to solve.
%
%    How hard is it to do so?  Moreover, how hard is it to generate \emph{good}
%    puzzle instances, for various definitions of \emph{good} (sufficiently
%    challenging to solve, or having unique solutions, or solvable/unsolvable by
%    certain strategies)?
%
%    % lauren sanchis
%
%  \item \label{itm:intro.q.pspace} \PSPACE-complete games derived from
%    \NP-complete puzzles.  A canonical \NP-complete puzzle is the \Problem{sat}
%    (Boolean Satisfiability) puzzle: given a Boolean formula \(\phi(x_1, \dots,
%    x_n)\), does there exist an assignment to its inputs \(x_1, \dots, x_n\)
%    such that \(\phi(\dots) = 1\)? In an analogous game, two players alternate
%    turns assigning \(x_1, \dots, x_n\); player 1 wins if \(\phi(\dots)=1\),
%    and player 2 wins if \(\phi(\dots)=0\).  Does either player have a
%    (guaranteed) winning strategy?  This game, known as \Problem{qsat}
%    (Quantified Satisfiability), is a canonical example of a \PSPACE-complete
%    game.
%
%    Can other \NP-complete puzzles be similarly generalized into games?  Will
%    those games also be \PSPACE-complete?
%
%    % schaefer
%
%  \item \label{itm:intro.q.ph} Fixed-turn games and the polynomial hierarchy.
%    In between the complexity classes \NP{} and \PSPACE{} lies a chain of
%    increasingly-complex problem-classes known as the \emph{polynomial
%    hierarchy}.  In some cases, problems in the polynomial hierarchy may be
%    thought of as game generalizations of \NP-complete puzzles with a
%    \emph{fixed} number of turns.  For instance, in a two-turn version of
%    \Problem{sat}, inputs are partitioned into two (disjoint) groups \(X_1\)
%    and \(X_2\); on turn 1, player 1 assigns \(X_1\), and on turn 2, player 2
%    assigns \(X_2\).  As before, player 1 (respectively 2) wins if
%    \(\phi(\dots) = 1\) (respectively \(0\)).  Determining whether player 1 has
%    a winning strategy is complete for a complexity class known as
%    \(\SigmaP2\), which lies just above \NP{} in the hierarchy, and analogous
%    games with \(k\) turns are \(\SigmaP k\)-complete.
%
%    Do polynomial-hierarchy generalizations of other \NP-complete puzzles
%    exist?
%
%\end{enumerate}
%
%\Cref{ch:progress} discusses my progress so far in each of these areas.
%Questions \ref{itm:intro.q.generation} and \ref{itm:intro.q.pspace} have been
%explored in-depth by others, while question \ref{itm:intro.q.ph} appears to be
%scarcely explored.  As such, I provide only brief summaries of/reflections on
%the existing work pertaining to \ref{itm:intro.q.generation} and
%\ref{itm:intro.q.pspace}.  Meanwhile, I describe in greater detail question
%\ref{itm:intro.q.ph}, which is the focus of my explorations so far.
%
%\Cref{ch:future} summarizes the primary questions \& goals that will guide
%my exploration next semester.
%
%Finally, \cref{ch:bib} contains an annotated bibliography of existing work
%pertaining to each of these topics.


