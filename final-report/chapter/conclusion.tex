\chapter{Conclusion}

\label{ch:conclusion}

In this thesis, I explore the computational complexity of fixed-turn games under
the \emph{polynomial hierarchy}, with \SigmaP k representing the class of ``can
the first player guarantee a win?'' decision problems for all games with \(k\)
turns, and \(\PiP k\) representing the class of complement decision problems,
``is the first player doomed to lose?''.  Using the foundational
\Problem{Circuit Satisfiability} games as a starting point, we search for other
\SigmaP k/\PiP k-\emph{complete} games—games that are maximally hard for each
class and are therefore ideal representatives characterizing the difficulty of
each class.  Finally, we introduce the \(k\)-turn \Problem{Graph 3-Colorability}
games on graphs, and by embedding circuits in graph 3-colorings show that
\(k\)-turn \Problem{3-Colorability} is \SigmaP k-complete.

Unfortunately, in a mere half year of thesis (having spent the first half
figuring out only \emph{what} question to explore), I had time to include in my
thesis the full treatment of only one flavor of
game—\Problem{3-Colorability}—even though I am absolutely confident there are
tons more.

For instance, consider the \Problem{Exact Set Covering} problem, which I like to
informally call the ``sushi problem'': you run an eccentric sushi restaurant
serving \(n\) distinct flavors of sushi, numbered \(1,\dotsc,n\); your
restaurant's menu contains a list of \emph{combos}, each a subset of
\(\Set{1,\dotsc,n}\); is there a way to order certain combos such that each
sushi flavor is included \emph{exactly} once?  This is a well-known \NP-complete
problem, and it can be straightforwardly extended into a \(k\)-turn game by
partitioning the combos into \(k\) groups and having players take turns choosing
whether or not to order each combo, subject at each turn to the ``properness''
constraint that ordering a previously-ordered sushi flavor causes immediate
defeat.  There is a straightforward way to map logic gates to sushi flavors and
combos, and this game is therefore \SigmaP k-complete—though there is no room
for me to detail this treatment here.

Moreover, my success with these two problems leads me to suspect that there
should be a straightforward way to extend almost any well-known \NP-complete
\emph{puzzle} into interesting \SigmaP k-complete games.  To this end, I can
think of two overarching future directions for this work:
\begin{itemize}

  \item Explore a ton of \NP-complete problems (perhaps starting with Karp's
    famous 21 \citep{karp.np}), and try to see how naturally they extend to
    \(\SigmaP k\)-complete multi-turn games.

  \item Drawing on commonalities between the 3-colorability games explored in
    \cref{ch:misc} and the sushi games sketched above, come up with a general
    framework for lifting \(1\)-turn puzzles into \(k\)-turn games.

    For instance, both the 3-colorability games and the sushi games share a
    \emph{properness} constraint: in 3-colorings, it is that no two neighboring
    vertices may share a color; in sushi orders, it is that no two selected
    combos may share a flavor.  Likewise, both games have a condition indicating
    the total completion of the game: in 3-colorings, it is totality of the
    coloring after all vertices have been filled in; in sushi orders, it is that
    each flavor is ordered at least once.  Together, totality \emph{and}
    properness result in successful ``emulation'' of circuits.

    Perhaps there is some way to define general notions of \emph{turn},
    \emph{properness}, \emph{totality}, and \emph{emulation} in order to
    streamline the process of showing \SigmaP k-completeness of games.

\end{itemize}

Finally, to recap it all, the most important question surrounding this thesis
(and really any work) is, why is it interesting?  To this, I say: everybody
knows about \P-vs-\NP—the pervasion of \NP-completeness among interesting
real-world puzzles makes the \P-vs-\NP{} question extremely impactful, both
theoretically and practically.  Taking the perspective of this thesis, we view
\P{} as the class of \(0\)-turn games and \(\NP\) the class of \(1\)-turn games;
why stop at \P-vs-\NP?  Why not ask about \(\NP\)-vs-\SigmaP 2?
\SigmaP2-vs-\SigmaP3?  The central question really boils down (or boils
\emph{up}, perhaps?) to, \emph{does the number of turns in a game make a
difference}?  The answers to this question, at \emph{every} level, not just
\(0\)-vs-\(1\), are worth exploring, and together, they give resounding insight
on the general structure of puzzles and games.


%future work: general framework to convert NP-complete problems to
%sigmap-complete games.  basic idea--there is a consistency/properness condition
%enforced at each step of the players' moves, and a completion condition (all
%things have been filled in).  if the consistency condition is violated at any
%step, the violating player loses.  otherwise, if the completion condition is
%attained, the last player wins.  finally, map two conditions onto circuits.
%
%other future work: explore more games/problems
