\documentclass{report-snippet}

\title{Circuits}

\begin{document}

\section{Circuits}

A (boolean) circuit is a network of \emph{wires} and \emph{logic gates}.  Each
wire carries a value in \(\Set{\True,\False}\).  There are three kinds of logic
gates, each of which takes some wires as \emph{inputs} and sends to another
wire an \emph{output} value \(\Set{\True,\False}\) as a function of its inputs.
Below we define the three logic gates and show their diagram representations:
\begin{itemize}

  \item A \NOT{} gate takes one input and produces the opposite value as
    output.

    \begin{center}
      \begin{tikzpicture}[circuit logic US]
        \node[not gate](n){};
        \draw[wire]
          (n.input) -- +(-2em,0) node[left]{input}
          (n.output) -- +(2em,0) node[right]{output};
      \end{tikzpicture}
    \end{center}
    \[
      \begin{array}{c|c}
        \text{input} & \text{output} \\ \midrule
        \False & \True \\
        \True & \False
      \end{array}
    \]

  \item An \AND{} gate takes two inputs and produces \True{} if and only if
    \emph{both} of its inputs are \True.

    \begin{center}
      \begin{tikzpicture}[circuit logic US]
        \node[and gate](a){};
        \draw[wire]
          (a.input 1) -- +(-2em,0) coordinate(a')
          (a.input 2) -- +(-2em,0) coordinate(a'') ($(a')!1/2!(a'')$) node[left]{inputs}
          (a.output) -- +(2em,0) node[right]{output};
      \end{tikzpicture}
    \end{center}
    \[
      \begin{array}{cc|c}
        \multicolumn{2}{c|}{\text{inputs}} & \text{output} \\ \midrule
        \False & \False & \False \\
        \False & \True & \False \\
        \True & \False & \False \\
        \True & \True & \True
      \end{array}
    \]

  \item An \OR{} gate takes two inputs and produces \True{} if and only if \emph{at
    least one} of its inputs is \True.

    \begin{center}
      \begin{tikzpicture}[circuit logic US]
        \node[or gate](o){};
        \draw[wire]
          (o.input 1) -- +(-2em,0) coordinate(o')
          (o.input 2) -- +(-2em,0) coordinate(o'') ($(o')!1/2!(o'')$) node[left]{inputs}
          (o.output) -- +(2em,0) node[right]{output};
      \end{tikzpicture}
    \end{center}
    \[
      \begin{array}{cc|c}
        \multicolumn{2}{c|}{\text{inputs}} & \text{output} \\ \midrule
        \False & \False & \False \\
        \False & \True & \True \\
        \True & \False & \True \\
        \True & \True & \True
      \end{array}
    \]

\end{itemize}

\begin{definition}{circuit}{}

  % TODO consider using switches and lightbulbs in formalism

  A \Term{circuit} consists of a network of such wires and logic gates, with
  the following requirements for well-defined-ness:
  \begin{itemize}
    \item Some wires are marked as circuit-level \Term{inputs}.  These wires
      may not be the output wire of any gate within the circuit.
    \item Every wire that isn't a circuit-level input must be the output wire
      of exactly one logic gate.
    \item There must be no (directed) cycles in the circuit.  That is, there
      must not exist gates \(g₁,\dotsc,gₖ\) such that the output of \(g₁\) is
      an input to \(g₂\), the output of \(g₂\) an input to \(g₃\), …, and the
      output of \(gₖ\) an input to \(g₁\).
    \item Finally, there is exactly one wire marked the \Term{output} of the
      circuit.
  \end{itemize}
  Under these requirements, each combination of boolean values assigned to the
  circuit-level input wires, uniquely determines the circuit-level output's
  value. Therefore, a circuit with \(n\) inputs defines a function
  \(\Set{\False,\True}ⁿ→\Set{\False,\True}\).

  To simplify notation, we refer to each circuit and its boolean function by
  the same name.  That is, if \(C\) refers to a circuit, we denote by
  \(C(x₁,\dotsc,xₙ)\) the output computed by \(C\) on input values
  \(x₁,\dotsc,xₙ\).  Occasionally, if we need to disambiguate between a circuit
  and its function, we denote the circuit \(C\) and its function \(ϕ_C\).

\end{definition}

Here are some example circuits.

\begin{center}
  \begin{tikzpicture}[circuit logic US]

    \matrix[gates]{
      \coordinate(i1); & \node[not gate](n1){}; & \node[and gate](a){}; & \coordinate(out); \\
      \coordinate(i2); & \node[not gate](n2){}; \\
    };

  \draw[wire]
  foreach \i in {1,2} { (i\i) -- (n\i.input) (n\i.output) to (a.input \i) }
  (a.output) -- (out);

  \end{tikzpicture}
\end{center}

\begin{center}
  \begin{tikzpicture}[circuit logic US]
    \matrix[gates]{
      \coordinate(i1); && \node[not gate](n1){}; & \node[and gate](a1){}; &
      \node[or gate](o){}; & \coordinate(out); \\
      & \coordinate(i1'); \\
      & \coordinate(i2'); \\
      \coordinate(i2); && \node[not gate](n2){}; & \node[and gate](a2){}; \\
    };

    \draw[wire]
    foreach \i in {1,2} {
      (i\i) to (n\i.input)
      (n\i.output) to (a\i.input \i)
      (a\i.output) to (o.input \i)
    }
    (i1) to (i1') to (a2.input 1) (i2) to (i2')
    (o.output) -- (out);

    \draw[wire, over] (i2') to (a1.input 2);
  \end{tikzpicture}
\end{center}


%\begin{definition}{}{}
%
%  Let \(C\) be a boolean circuit, with input wires \(i₁,\dotsc,iₙ\) and output
%  wire \(o\).  The \Term{function computed by \(C\)} is a mapping
%  \(\Set{\False,\True}ⁿ→\Set{\False,\True}\) that sends each combination of
%  values on \(i₁,\dotsc,iₙ\) to the resulting value assigned by \(C\) to the
%  output wire \(o\).
%
%\end{definition}

\section{Puzzles and games on circuits}

The most straightforward circuit problem is this: given a circuit \(C\) and
fixed assignments \(x₁,\dotsc,xₙ\) to its inputs, compute its output value.
Call this problem the \Term{\Problem{CircuitValue} problem}, or
\Term{\(\CSAT₀\)}.

\begin{definition}{\Problem{CircuitValue} / \(\CSAT₀\)}{}

  Given a circuit \(C\) and fixed assignments to its inputs \(x₁,\dotsc,xₙ\),
  compute \(C(x₁,\dotsc,xₙ)\).  To formalize this as a decision problem, return
  \emph{yes} if \(C(x₁,\dotsc,xₙ)=\True\) and \emph{no} otherwise:
  \[
    \Problem{CircuitValue} =
    \SetBuilder{(\text{circuit \(C\)}, \text{inputs \((x₁,\dotsc,xₙ)\)})}{
      C(x₁,\dotsc,xₙ)=\True
    }.
  \]

\end{definition}

\Problem{CircuitValue} is fairly easy to solve: iterate through each gate,
starting with those closest to the circuit inputs, whose values are given;
compute and write down the gate's output, then repeat.

% TODO consider being more precise here, blah blah topological sort, or lazy backfill

We now define a hierarchy of puzzles and games played on circuits.

\subsection{The circuit satisfiability puzzle, a.k.a. \(\CSAT₁\)}






%You run a news organization.


\end{document}
