\documentclass{final-report}

\title{You can solve it, but can you play it?}

\subtitle{Puzzles, games, and the polynomial hierarchy}

\author{Kye Shi}
\advisor{Nicholas Pippenger}
\reader{Arthur T. Benjamin}
\date{2021 November}

\bibliography{bibliography.bib}
\bibliography{newbib.bib}

\definecolor{ks0}{HTML}{0050c0}
\definecolor{ks1}{HTML}{e00060}
\definecolor{ks2}{HTML}{f0b000}

\tikzset{
  gates/.style={
    row sep=1em/2,
    column sep=2em,
    matrix of nodes,
  },
  every circuit symbol/.style={
    fill,
    fill opacity=1/8,
    anchor=output,
  },
  vertex/.style={
    circle,
    draw,
    minimum size=1em/2,
    inner sep=0pt,
  },
  vertex label/.style={
    text height=2em/3,
    text depth=1em/3,
  },
  edge/.style={
  },
  wire/.style={
    rounded corners=1em/4,
    to path={
      -- ($ (\tikztostart)!1em!(\tikztostart -| \tikztotarget) $)
      -- ($ (\tikztotarget)!1em!(\tikztostart |- \tikztotarget) $)
      -- (\tikztotarget)
    },
  },
}

\begin{document}

\frontmatter
\maketitle
\tableofcontents

TODO:

\begin{itemize}
  \item Overview, maybe.
  \item Introduction: everybody knows P vs. NP, maybe PSPACE; this thesis
    explores the space in-between.  the author suspects that for every NP
    problem, there is a sigma1p analog, etc. blah blah; studying problems by
    thinking of them as games provides valuable intuition, etc.
  \item Abstract complexity background: P, NP, PSPACE, and polynomial hierarchy
    \begin{itemize}
      \item Circuit value, satisfiability games
    \end{itemize}
  \item SAT variation games
    \begin{itemize}
      \item Circuits could just as well be plain boolean formulae, or CNF
        formulae, etc.; list some prior work on other known SAT games in
        PSPACE, polynomial-hierarchy, etc.
    \end{itemize}
  \item Graph coloring games
    \begin{itemize}
      \item Intro---overview of existing results: NP-completeness of graph
        coloring, col and snort as PSPACE-complete games.
      \item Custom presentation of NP-completeness proof via circuit reduction
      \item Introduction of 2-turn game, 3-turn game, k-turn game, etc.
      \item Complexity claims and proofs
    \end{itemize}
  \item three-dimensional matching games
  \item Sudoku??
  \item Conclusion: only a few of countlessly many problems introduced/explored
\end{itemize}


\mainmatter

\chapter{Introduction}

The basic question of computational complexity---``how hard is this problem for
a computer to solve?''---is central to nearly every topic in computer science.
And yet the formalisms of complexity theory often seem, in my own experience,
intimidatingly abstract, phrased in terms of intangible models of computation
such as non-deterministic Turing machines and oracles.

The remedy, I believe, lies in studying complexity theory through the lens of
\emph{puzzles} and \emph{games}.  Not only do they provide a concrete grounding
for the abstractions, they also offer a particularly insightful, accessible,
and most importantly fun approach to understanding complexity theory.  In fact,
many of the most popularly known and appreciated results in complexity theory
are those about so-called ``\NP-complete puzzles'', such as Sudoku, and
``\PSPACE-complete games'', such as Checkers and Go.

This thesis emphasizes that approach in its exploration of a particularly
foundational, yet often overlooked, ladder of complexity classes known as the
\emph{polynomial hierarchy}.  \NP{} is the class of (one-player) ``puzzles'',
and \PSPACE{} is the class of (two-player) ``games'' of polynomial length; the
polynomial hierarchy, then, lies in the middle, encompassing games of
\emph{fixed} length.  Through this lens, the (in)famous \P-vs-\NP{} question is
but the first in a ladder of questions that are, arguably, just as crucial and
impactful.

%The polynomial hierarchy is as central to
%complexity theory as the \P-vs-\NP{} problem is well-known.

\section{Overview}

This document is structured as follows.  First, \cref{ch:background,ch:boolean}
establish preliminary background concepts and conventions adopted throughout
this thesis.  Next, \cref{ch:circuit} lays the central theoretical groundwork,
defining the \emph{polynomial hierarchy} through a fundamental family of
problems known as the \Problem{Circuit Satisfiability} games.  Next,
\cref{ch:misc} explores a novel family of games generalized from the
\Problem{Graph 3-Colorability} puzzle and establishes \emph{hardness} bounds on
each of those games.  Finally, \cref{ch:conclusion} concludes by discussing the
future directions of this work and its broader implications.

\section{Prior work and inspirations}

Much of the background exposition on complexity theory referenced in this thesis
is reproduced from Christos Papadimitriou's textbook,
\citet{papadimitriou.cc} (though many of the foundational ideas were
originally introduced/proven elsewhere, e.g.
\citet{cook.np,levin.np,stockmeyer.ph}), reframed through the
puzzles-and-games perspective and supplemented with a few comments on intuition.

The main family of games explored in this thesis, fixed-turn
\Problem{3-Colorability} games (\cref{ch:misc}), is a generalization of
(one-turn) \Problem{3-Colorability}, a well-known \NP-complete puzzle originally
proven \NP-complete by \citet{karp.np}.  Others have studied (multi-turn)
game generalizations of \Problem{3-Colorability}, but all versions that I've
encountered are \PSPACE-complete, in which the number of turns played during the
game scales proportionally with the size of the graph
\citep{bodlaender.coloring,bh.placement,kbd.impartial,cpss.coloring,schaefer.games}. As far as I'm aware, the variations I explore here—with fixed
numbers of turns regardless of the size of the graph—is unexplored, and the main
theorem about its \SigmaP k-completeness (\cref{th:yayay}) is novel.  The basic
idea underlying my proof is the composition of two well-known results:
\begin{itemize}[nosep]
  \item \citet{karp.np}'s classic proof of the \NP-hardness of the \Problem{3-Colorability} puzzle, via a reduction from \Problem{3CNF-Satisfiability};
  \item 's transformation from boolean circuits to equivalent 3CNF-clauses.
\end{itemize}

Without further ado, let's begin.


%TODO: outline/overview of chapters, after those chapters are written

%TODO: also give general citations here, e.g. papadimitriou for many
%foundational background info, etc.
%
%TODO: notation table also belongs in this chapter i think

%p-vs-np well known, polynomial-hierarchy central

%puzzles and games; hierarchy lies in the interstices.  we examine a few
%interesting (by no means exhaustive, or even close to comprehensive)
%np-complete puzzles with pspace-complete analogues, and we









%Famously central to the theory of computational complexity is the \P-vs-\NP{}
%question, and essential to our understanding of that question is the study of
%\NP-complete problems such as the Boolean Satisfiability puzzle, the Graph
%Colorability puzzle, and countless more.  Puzzles like these, which nearly any
%layperson can appreciate, offer a particularly insightful, intuitive, and
%\emph{fun} lens through which to study computational complexity. Explorations
%of more complex problem-classes such as \PSPACE{} can be similarly approached
%through the study of strategic decision \emph{games} such as Othello, Checkers,
%and Go.
%
%What lies in the interstices between \emph{puzzles} and \emph{games}?  How do
%we take a puzzle and generalize it into a game, and what are the puzzle-games
%we encounter along the way?  And how hard exactly are these puzzle-games to
%decide?  These questions are the focus of my thesis.

%So far, I have explored these questions from three angles:
%\begin{enumerate}
%
%  \item \label{itm:intro.q.generation} Puzzle generation.  If I wish to solve a
%    puzzle, you can play a game with me by constructing puzzle \emph{instances}
%    for me to solve.  For instance, \emph{solving} Sudoku is an \NP-complete
%    problem; your task is to \emph{generate} (partially-filled) Sudoku boards
%    for me to solve.
%
%    How hard is it to do so?  Moreover, how hard is it to generate \emph{good}
%    puzzle instances, for various definitions of \emph{good} (sufficiently
%    challenging to solve, or having unique solutions, or solvable/unsolvable by
%    certain strategies)?
%
%    % lauren sanchis
%
%  \item \label{itm:intro.q.pspace} \PSPACE-complete games derived from
%    \NP-complete puzzles.  A canonical \NP-complete puzzle is the \Problem{sat}
%    (Boolean Satisfiability) puzzle: given a Boolean formula \(\phi(x_1, \dots,
%    x_n)\), does there exist an assignment to its inputs \(x_1, \dots, x_n\)
%    such that \(\phi(\dots) = 1\)? In an analogous game, two players alternate
%    turns assigning \(x_1, \dots, x_n\); player 1 wins if \(\phi(\dots)=1\),
%    and player 2 wins if \(\phi(\dots)=0\).  Does either player have a
%    (guaranteed) winning strategy?  This game, known as \Problem{qsat}
%    (Quantified Satisfiability), is a canonical example of a \PSPACE-complete
%    game.
%
%    Can other \NP-complete puzzles be similarly generalized into games?  Will
%    those games also be \PSPACE-complete?
%
%    % schaefer
%
%  \item \label{itm:intro.q.ph} Fixed-turn games and the polynomial hierarchy.
%    In between the complexity classes \NP{} and \PSPACE{} lies a chain of
%    increasingly-complex problem-classes known as the \emph{polynomial
%    hierarchy}.  In some cases, problems in the polynomial hierarchy may be
%    thought of as game generalizations of \NP-complete puzzles with a
%    \emph{fixed} number of turns.  For instance, in a two-turn version of
%    \Problem{sat}, inputs are partitioned into two (disjoint) groups \(X_1\)
%    and \(X_2\); on turn 1, player 1 assigns \(X_1\), and on turn 2, player 2
%    assigns \(X_2\).  As before, player 1 (respectively 2) wins if
%    \(\phi(\dots) = 1\) (respectively \(0\)).  Determining whether player 1 has
%    a winning strategy is complete for a complexity class known as
%    \(\SigmaP2\), which lies just above \NP{} in the hierarchy, and analogous
%    games with \(k\) turns are \(\SigmaP k\)-complete.
%
%    Do polynomial-hierarchy generalizations of other \NP-complete puzzles
%    exist?
%
%\end{enumerate}
%
%\Cref{ch:progress} discusses my progress so far in each of these areas.
%Questions \ref{itm:intro.q.generation} and \ref{itm:intro.q.pspace} have been
%explored in-depth by others, while question \ref{itm:intro.q.ph} appears to be
%scarcely explored.  As such, I provide only brief summaries of/reflections on
%the existing work pertaining to \ref{itm:intro.q.generation} and
%\ref{itm:intro.q.pspace}.  Meanwhile, I describe in greater detail question
%\ref{itm:intro.q.ph}, which is the focus of my explorations so far.
%
%\Cref{ch:future} summarizes the primary questions \& goals that will guide
%my exploration next semester.
%
%Finally, \cref{ch:bib} contains an annotated bibliography of existing work
%pertaining to each of these topics.




%\chapter{Introduction}

The basic question of computational complexity---``how hard is this problem for
a computer to solve?''---is central to nearly every topic in computer science.
And yet the formalisms of complexity theory often seem, in my own experience,
intimidatingly abstract, phrased in terms of intangible models of computation
such as non-deterministic Turing machines and oracles.

The remedy, I believe, lies in studying complexity theory through the lens of
\emph{puzzles} and \emph{games}.  Not only do they provide a concrete grounding
for the abstractions, they also offer a particularly insightful, accessible,
and most importantly fun approach to understanding complexity theory.  In fact,
many of the most popularly known and appreciated results in complexity theory
are those about so-called ``\NP-complete puzzles'', such as Sudoku, and
``\PSPACE-complete games'', such as Checkers and Go.

This thesis emphasizes that approach in its exploration of a particularly
foundational, yet often overlooked, ladder of complexity classes known as the
\emph{polynomial hierarchy}.  \NP{} is the class of (one-player) ``puzzles'',
and \PSPACE{} is the class of (two-player) ``games'' of polynomial length; the
polynomial hierarchy, then, lies in the middle, encompassing games of
\emph{fixed} length.  Through this lens, the (in)famous \P-vs-\NP{} question is
but the first in a ladder of questions that are, arguably, just as crucial and
impactful.

%The polynomial hierarchy is as central to
%complexity theory as the \P-vs-\NP{} problem is well-known.

\section{Overview}

This document is structured as follows.  First, \cref{ch:background,ch:boolean}
establish preliminary background concepts and conventions adopted throughout
this thesis.  Next, \cref{ch:circuit} lays the central theoretical groundwork,
defining the \emph{polynomial hierarchy} through a fundamental family of
problems known as the \Problem{Circuit Satisfiability} games.  Next,
\cref{ch:misc} explores a novel family of games generalized from the
\Problem{Graph 3-Colorability} puzzle and establishes \emph{hardness} bounds on
each of those games.  Finally, \cref{ch:conclusion} concludes by discussing the
future directions of this work and its broader implications.

\section{Prior work and inspirations}

Much of the background exposition on complexity theory referenced in this thesis
is reproduced from Christos Papadimitriou's textbook,
\citet{papadimitriou.cc} (though many of the foundational ideas were
originally introduced/proven elsewhere, e.g.
\citet{cook.np,levin.np,stockmeyer.ph}), reframed through the
puzzles-and-games perspective and supplemented with a few comments on intuition.

The main family of games explored in this thesis, fixed-turn
\Problem{3-Colorability} games (\cref{ch:misc}), is a generalization of
(one-turn) \Problem{3-Colorability}, a well-known \NP-complete puzzle originally
proven \NP-complete by \citet{karp.np}.  Others have studied (multi-turn)
game generalizations of \Problem{3-Colorability}, but all versions that I've
encountered are \PSPACE-complete, in which the number of turns played during the
game scales proportionally with the size of the graph
\citep{bodlaender.coloring,bh.placement,kbd.impartial,cpss.coloring,schaefer.games}. As far as I'm aware, the variations I explore here—with fixed
numbers of turns regardless of the size of the graph—is unexplored, and the main
theorem about its \SigmaP k-completeness (\cref{th:yayay}) is novel.  The basic
idea underlying my proof is the composition of two well-known results:
\begin{itemize}[nosep]
  \item \citet{karp.np}'s classic proof of the \NP-hardness of the \Problem{3-Colorability} puzzle, via a reduction from \Problem{3CNF-Satisfiability};
  \item 's transformation from boolean circuits to equivalent 3CNF-clauses.
\end{itemize}

Without further ado, let's begin.


%TODO: outline/overview of chapters, after those chapters are written

%TODO: also give general citations here, e.g. papadimitriou for many
%foundational background info, etc.
%
%TODO: notation table also belongs in this chapter i think

%p-vs-np well known, polynomial-hierarchy central

%puzzles and games; hierarchy lies in the interstices.  we examine a few
%interesting (by no means exhaustive, or even close to comprehensive)
%np-complete puzzles with pspace-complete analogues, and we









%Famously central to the theory of computational complexity is the \P-vs-\NP{}
%question, and essential to our understanding of that question is the study of
%\NP-complete problems such as the Boolean Satisfiability puzzle, the Graph
%Colorability puzzle, and countless more.  Puzzles like these, which nearly any
%layperson can appreciate, offer a particularly insightful, intuitive, and
%\emph{fun} lens through which to study computational complexity. Explorations
%of more complex problem-classes such as \PSPACE{} can be similarly approached
%through the study of strategic decision \emph{games} such as Othello, Checkers,
%and Go.
%
%What lies in the interstices between \emph{puzzles} and \emph{games}?  How do
%we take a puzzle and generalize it into a game, and what are the puzzle-games
%we encounter along the way?  And how hard exactly are these puzzle-games to
%decide?  These questions are the focus of my thesis.

%So far, I have explored these questions from three angles:
%\begin{enumerate}
%
%  \item \label{itm:intro.q.generation} Puzzle generation.  If I wish to solve a
%    puzzle, you can play a game with me by constructing puzzle \emph{instances}
%    for me to solve.  For instance, \emph{solving} Sudoku is an \NP-complete
%    problem; your task is to \emph{generate} (partially-filled) Sudoku boards
%    for me to solve.
%
%    How hard is it to do so?  Moreover, how hard is it to generate \emph{good}
%    puzzle instances, for various definitions of \emph{good} (sufficiently
%    challenging to solve, or having unique solutions, or solvable/unsolvable by
%    certain strategies)?
%
%    % lauren sanchis
%
%  \item \label{itm:intro.q.pspace} \PSPACE-complete games derived from
%    \NP-complete puzzles.  A canonical \NP-complete puzzle is the \Problem{sat}
%    (Boolean Satisfiability) puzzle: given a Boolean formula \(\phi(x_1, \dots,
%    x_n)\), does there exist an assignment to its inputs \(x_1, \dots, x_n\)
%    such that \(\phi(\dots) = 1\)? In an analogous game, two players alternate
%    turns assigning \(x_1, \dots, x_n\); player 1 wins if \(\phi(\dots)=1\),
%    and player 2 wins if \(\phi(\dots)=0\).  Does either player have a
%    (guaranteed) winning strategy?  This game, known as \Problem{qsat}
%    (Quantified Satisfiability), is a canonical example of a \PSPACE-complete
%    game.
%
%    Can other \NP-complete puzzles be similarly generalized into games?  Will
%    those games also be \PSPACE-complete?
%
%    % schaefer
%
%  \item \label{itm:intro.q.ph} Fixed-turn games and the polynomial hierarchy.
%    In between the complexity classes \NP{} and \PSPACE{} lies a chain of
%    increasingly-complex problem-classes known as the \emph{polynomial
%    hierarchy}.  In some cases, problems in the polynomial hierarchy may be
%    thought of as game generalizations of \NP-complete puzzles with a
%    \emph{fixed} number of turns.  For instance, in a two-turn version of
%    \Problem{sat}, inputs are partitioned into two (disjoint) groups \(X_1\)
%    and \(X_2\); on turn 1, player 1 assigns \(X_1\), and on turn 2, player 2
%    assigns \(X_2\).  As before, player 1 (respectively 2) wins if
%    \(\phi(\dots) = 1\) (respectively \(0\)).  Determining whether player 1 has
%    a winning strategy is complete for a complexity class known as
%    \(\SigmaP2\), which lies just above \NP{} in the hierarchy, and analogous
%    games with \(k\) turns are \(\SigmaP k\)-complete.
%
%    Do polynomial-hierarchy generalizations of other \NP-complete puzzles
%    exist?
%
%\end{enumerate}
%
%\Cref{ch:progress} discusses my progress so far in each of these areas.
%Questions \ref{itm:intro.q.generation} and \ref{itm:intro.q.pspace} have been
%explored in-depth by others, while question \ref{itm:intro.q.ph} appears to be
%scarcely explored.  As such, I provide only brief summaries of/reflections on
%the existing work pertaining to \ref{itm:intro.q.generation} and
%\ref{itm:intro.q.pspace}.  Meanwhile, I describe in greater detail question
%\ref{itm:intro.q.ph}, which is the focus of my explorations so far.
%
%\Cref{ch:future} summarizes the primary questions \& goals that will guide
%my exploration next semester.
%
%Finally, \cref{ch:bib} contains an annotated bibliography of existing work
%pertaining to each of these topics.



%\chapter{Current progress}

\label{ch:progress}

\section{\NP{} as puzzles, or one-move games}

Recall that \NP{} is the class of problems solvable by guess-and-check, with a
\emph{check} problem in \P{} (\cref{defn:np}):
\[
  \NP = \SetBuilder{L}{
    \exists \underbrace{\mathstrut L' \in \P}_{\mathclap{\text{the ``check'' problem}}} \;
    \forall x \quad
    x \in L \iff \underbrace{\exists g \; (x, g) \in L'}_{\mathclap{\text{guess-and-check}}}
  }.
\]
(In the above, it is \emph{implicitly} required that \(\Abs g\) be
polynomially-bounded with respect to \(\Abs x\), but we have omitted it in
notation for readability.)

Another famous example of a problem in \NP{} is \Problem{sudoku}, framed as the
following decision problem:
\begin{definition}[\Problem{sudoku}]%
  We are given a square grid with dimensions \(n^2\times n^2\), some of whose
  cells are filled in with numbers in \(\{1,\dotsc,n^2\}\).  The grid is
  evenly partitioned into \(n\) chunks along each axis, resulting in \(n^2\)
  \emph{blocks} each with dimensions \(n\times n\).

  Does there exist a way to fill in the rest of the cells so that each row,
  column, and block on the grid contains each number in \(\{1,\dotsc,n^2\}\)
  exactly once?
\end{definition}

\todo[inline]{TODO add illustration of Sudoku board}

For this problem, a ``guess'' \(g\) consists of a list of numbers in
\(\{1,\dotsc,n^2\}\) specifying the values with which to fill in the empty
cells in the given grid.  The ``check'' problem, then, is stated as follows:
\begin{nested}
  Given a fully-filled-in Sudoku board, does each row, column, and block on the
  grid contain each of \(\{1,\dotsc,n^2\}\) exactly once?
\end{nested}




%Also recall an example of an \NP{} problem, \Problem{hamiltonian-path}
%(\cref{def:hamiltonian-path}), which asks: given a graph, does it have a
%Hamiltonian path?  Here, the ``check'' problem \(L'\) can be stated as follows:
%\begin{nested}
%  Given a graph \(x = \Gamma\) with vertices \(v_1,\dotsc,v_n\), along with a
%  permutation \(g = \phi(1),\dotsc,\phi(n)\), does the sequence
%  \(v_{\phi(1)},\dotsc,v_{\phi(n)}\) specify a valid path on \(\Gamma\)?
%\end{nested}
%
%We can now intuitively reframe \Problem{hamiltonian-path} as a one-player
%``game'', played on an input ``board'' in the form

%in which the player writes down some
%permutation \(\phi\).  They ``win'' if it meets the validity condition \((x, g)
%\in L'\) and ``lose'' if it doesn't.  Under this framing, the decision problem
%becomes the following question: does the player have a winning \emph{strategy}?

\section{Multi-turn games and the polynomial hierarchy}

Consider, now, a two player, a ``solver'' and an
``adversary'', in two turns:
\begin{itemize}
  \item First, the solver
  \item Next, the adversary
\end{itemize}

This ``guess-and-check'' extension of \P{} may be continued

to define higher
and higher complexity classes, known as the \emph{polynomial hierarchy}:
\begin{definition}[polynomial hierarchy]
  \begin{align*}
    \SigmaP1 &= \NP = \SetBuilder{L}{\exists L' \in \P \; \forall x \quad
      x \in L \iff \exists g \; (x, g) \in L'
    }, \\
    \SigmaP2 &= \NP = \SetBuilder{L}{\exists L' \in \P \; \forall x \quad
      x \in L \iff \exists g \; (x, g) \in L'
    }, \\
  \end{align*}
\end{definition}




%In general, any complexity class has a guess-and-check equivalent, called its
%\emph{projection}:
%\begin{definition}[projection]
%  Let \(\C\) be a complexity class.  Its \emph{projection} is the class of
%  problems
%  \[
%    \pro\C = \SetBuilder*{L}{
%      \exists L' \in \C, \text{polynomial \(p\)}; \; \forall x \quad
%      x \in L \iff
%      \exists g \; \text{\(\Abs g \le p(\Abs x)\) and \((x, g) \in L'\)}
%    }.
%  \]
%
%  (Observe, then, that \(\NP = \pro\P\).)
%\end{definition}








\section{Boolean Satisfiability}

\subsection{The Satisfiability puzzle}

\todo{context on what booleans are?}

Our puzzles-and-games characterization of the polynomial hierarchy begins with
a well-known family of problems generally referred to as Boolean Satisfiability
problems.  Here is perhaps the simplest, most well-known Satisfiability puzzle:

\begin{definition}[\Problem{sat}]%
  Given a Boolean formula \(\phi(x_1, \dots, x_n)\), does there exist an
  assignment of Boolean values to inputs \(x_1, \dots, x_n\) such that
  \(\phi(x_1, \dots, x_n) = 1\)?  \Problem{sat} consists of the formula
  instances for which the answer is \emph{yes}.

  Formally:
  \[
    \Problem{sat} = \SetBuilder \phi {
      \exists (x_1, \dots, x_n) \in \Set{0,1}^n \quad \phi(x_1, \dots, x_n) = 1
    }.
  \]
\end{definition}

The \SAT{} puzzle is particularly useful and worth studying because of its
generality.  Booleans form the foundation of mathematical logic: every logical
statement can be encoded, in some manner, as a Boolean formula.  Consequently,
\SAT{} is, on an intuitive level, the most general possible puzzle---given any
other puzzle, encoding its rules in terms of Booleans reveals that it is merely
a special case of \SAT.  This idea is expressed formally as the Cook-Levin
theorem:

\begin{theorem}[Cook-Levin]
  \SAT{} is \NP-complete.
\end{theorem}

%\begin{proof}
%  \todo[inline]{put proof.  the most important reason to have the proof here is
%  to illustrate}
%\end{proof}

\subsection{Satisfiability games}

\begin{definition}[The two-turn \SAT{} games]%
  The two-turn \SAT{} game is played on a Boolean formula \(\phi(x_1, \dots,
  x_n, y_1, \dots, y_n)\) with inputs partitioned into two groups \(X =
  \Set{x_i}\) and \(Y = \Set{y_i}\).  The two turns proceed as follows:
  \begin{enumerate}
    \item Player 1 assigns values to \(X\).
    \item Player 2 assigns values to \(Y\).
  \end{enumerate}
  Player 1 wins if \(\phi\) is satisfied (\(\phi(\dots) = 1\)), and player 2
  wins if \(\phi\) is falsified.

  Who wins?  Two decision problems arise from this game:
  \begin{itemize}
    \item Does player 1 have a winning strategy?  That is, can player 1 make
      some first move so that no matter what player 2 does, player 1 always
      wins?

    \item Does player 2 have a winning strategy?  That is, no matter what
      player 1 plays, can player 2 respond with some move guaranteeing a win?
  \end{itemize}

\end{definition}


\section{Graph coloring}

\begin{definition}[\Problem{3col}]%
  Given a graph \(\Gamma\), is there a way to color each vertex in  \(\Gamma\)
  with one of three colors so that every pair of adjacent vertices has distinct
  colors?
\end{definition}

\begin{theorem}
  \Problem{3col} is \NP-complete.
\end{theorem}

\subsection{Graph coloring games}

\begin{definition}[Two-turn \Problem{3col}]%
  The two-turn \Problem{3col} game is played on a graph \(\Gamma\) whose
  vertices are partitioned into two (disjoint) groups \(X\) and \(Y\).  Two
  players take turns assigning one of three colors to vertices.  First, player
  1 colors vertices in \(X\); second, player 2 colors  vertices in \(Y\).
  Player 1 wins if the resulting coloring is \emph{invalid}---that is, there
  exists a pair of vertices sharing the same color; player 2 wins if the
  resulting coloring is valid.

\end{definition}

\begin{conjecture}
  Two-turn \Problem{3col} is complete for \SigmaP2 (or \PiP2, depending on
  which player's winning strategy we examine).
\end{conjecture}










%\section{Puzzle generation}
%
%\label{sec:progress.generation}
%
%The topic of puzzle generation difficulty has been explored in detail by Laura
%Sanchis
%\parencite{language-instances,test-gen-complexity,hard-diverse-graph-tests}.
%
%\todo[inline]{incomplete; i'm focusing my effort on the fixed-turn section first
%because that's more novel/interesting}
%
%\subsection{Puzzles with unique solutions}
%
%In many popularly-known puzzle games, one criterion for ``good'' puzzle
%generation is that the generated puzzle instance should have a unique solution.
%For example, given a \(9\times9\) Sudoku grid (partially pre-filled with
%numbers \(1,\dotsc,9\)), there should be \emph{exactly one} way to complete the
%grid---no more, no less.
%
%This formulation is incompatible with our
%
%\section{\PSPACE-complete games}
%
%\label{sec:progress.pspace}
%
%\todo[inline]{incomplete}
%
%\section{Fixed-turn games}
%
%\label{sec:progress.ph}
%
%

%\chapter{Future work}

\label{ch:future}

Since I did most thinking about the polynomial hierarchy question, most of the
``future work'' (basically, next semester) will be about that.  Concrete open
questions and proof TODOs go here, puzzles I want to investigate immediately
following graph-coloring, etc.  As time permits, the two other areas.

%\include{chapter/bib}

\printbibliography[heading=bibnumbered]

\end{document}
