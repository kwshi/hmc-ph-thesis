\documentclass{final-report}

\title{Games for one, games for two!}

\subtitle{computationally complex fun for polynomial-hierarchical families}

\author{Kye Shi}
\advisor{Nicholas Pippenger}
\reader{Arthur T. Benjamin}
\date{May 2022}

%\titlespacing\chapter{0em}{0em}{0em}
%\setlength{\beforechapskip}{0em}

\usepackage{caption}
\captionsetup{labelsep=period, labelfont=bf, margin=.5in}

\usepackage{tocloft}
\setlength\cftbeforetoctitleskip{-.5in}

%\usepackage{natbib}
%\bibliography{bibliography.bib}
\bibliography{newbib.bib}

\tikzset{
  gates/.style={
    row sep=1em/2,
    column sep=2em,
    matrix of nodes,
  },
  every circuit symbol/.style={
    fill,
    fill opacity=1/8,
    anchor=output,
  },
  %vertex/.style={
  %  circle,
  %  draw,
  %  minimum size=1em/2,
  %  inner sep=0pt,
  %},
  %vertex label/.style={
  %  text height=2em/3,
  %  text depth=1em/3,
  %},
  %wire/.style={
  %  rounded corners=1em/4,
  %  to path={
  %    -- ($ (\tikztostart)!1em!(\tikztostart -| \tikztotarget) $)
  %    -- ($ (\tikztotarget)!1em!(\tikztostart |- \tikztotarget) $)
  %    -- (\tikztotarget)
  %  },
  %},
}

\begin{document}

\pagenumbering{roman}
\frontmatter
\maketitle

\chapter*{Abstract}
\addcontentsline{toc}{chapter}{Abstract}

%perspective of viewing

In the first half of this thesis, we explore the \emph{polynomial-time
hierarchy}, emphasizing an intuitive perspective that associates decision
problems in the polynomial hierarchy to combinatorial games with fixed numbers
of turns. Specifically, problems in \P{} are thought of as \(0\)-turn games,
\NP{} as \(1\)-turn ``puzzle'' games, and in general \(\SigmaP k\) as \(k\)-turn
games, in which decision problems answer the binary question, ``can the starting
player guarantee a win?'' We introduce the formalisms of the polynomial
hierarchy through this perspective, alongside definitions of \(k\)-turn
\Problem{Circuit Satisfiability} games, whose \SigmaP k-completeness is assumed
from prior work (we briefly justify this assumption on intuitive grounds, but no
proof is given).

In the second half, we introduce and explore the properties of a novel family of
games called the \(k\)-turn \Problem{Graph 3-Colorability} games.  By embedding
boolean circuits in proper graph 3-colorings, we construct reductions from
\(k\)-turn \Problem{Circuit Satisfiability} games to \(k\)-turn
\Problem{3-Colorability} games, thereby showing that \(k\)-turn
\Problem{3-Colorability} is \SigmaP k-complete.

%Viewing the polynomial hierarchy as classes of fixed-turn-games frames the
%infamous \P-vs-\NP{} question as part of a more general question, ``does the
%number of turns in a game make a difference (to its difficulty)?''  Given how
%thoroughly this question has been explored at the first level,

Finally, we conclude by discussing possible future generalizations of this work,
vis-\`a-vis extending arbitrary \NP-complete puzzles to interesting \SigmaP
k-complete games.

%Ultimately, I hope that this games-and-puzzles interpretation of the polynomial
%hierarchy sheds intuitive light on the (sometimes daunting) abstractions of
%complexity theory,




%In this thesis, I give a detailed exposition of the \emph{polynomial-time
%hierarchy}, particularly emphasizing the perspective
%
%lens of viewing
%
%combinatorial games with fixed numbers of turns.

\pagestyle{plain}
%\vspace*{-1in}
\newpage
\tableofcontents

\chapter*{Acknowledgments}
\addcontentsline{toc}{chapter}{Acknowledgments}

First and foremost, I wish to thank my thesis advisor, Professor Nicholas
Pippenger, for generously signing on to advise me despite being retired, for
being tremendously patient with me despite how frequently I showed up to our
weekly meetings empty-handed, and of course for his helpful guidance throughout
this project.  While we're at it, let me also thank all the other wonderful
professors who have mentored me through the years—Professor Stephan Garcia,
Professor Dagan Karp, Professor Lucas Bang, Professor Weiqing Gu—you helped make
me the mathematician and the person I am now.

I am also grateful to my friends at
\href{http://mathily.org}{MathILy}—sarah-marie, Tom, Hannah, Brian, Corrine,
Max, Josh—you're the reason I fell in love with math in the first place, and
working with you continually inspires me to approach math with unrelenting
levity.

%Thank you, especially, to Corrine and Josh, for being there to support
%me that one time I \emph{really} needed it—

And finally, dearest thanks to my close friends outside of math: Sophia Cheng,
for supporting me in every way and for being a fabulous dance partner; Forest
Kobayashi, for befriending me ; Cole Kurashige, for his
sagely life advice; Kaveh Pezeshki, my OneWheel buddy for life; and last but not
least, my mom, for funding my exorbitant college education (and also, of course,
for taking care of me my entire life).  Cheers!

\cleardoublepage
\pagenumbering{arabic}

\pagestyle{thesis}

%draft outline:
%
%\begin{itemize}
%  \item Introduction and overview.
%  \item Basic complexity background: P, reductions, completeness
%  \item More complexity background---NP, PSPACE, polynomial hierarchy as games,
%    etc. Haven't made up my mind yet about whether to introduce this abstractly
%    on its own, or introduce it concurrently with CSAT games to keep things
%    concrete/tangible.  I'm leaning the latter.
%    \begin{itemize}
%      \item Circuit value, satisfiability games
%    \end{itemize}
%  \item SAT variation games
%    \begin{itemize}
%      \item Circuits could just as well be plain boolean formulae, or CNF
%        formulae, etc.; list some prior work on other known SAT games in
%        PSPACE, polynomial-hierarchy, etc.  (paper on this in ann. bib.)
%    \end{itemize}
%  \item Graph coloring games
%    \begin{itemize}
%      \item Intro---overview of existing results: NP-completeness of graph
%        coloring, col and snort as PSPACE-complete games.
%      \item Custom presentation of NP-completeness proof via circuit reduction
%      \item Introduction of 2-turn game, 3-turn game, k-turn game, etc.
%      \item Complexity claims and proofs
%    \end{itemize}
%  \item Exact-covering games
%  \item three-dimensional matching games
%  \item Conclusion % only a few of countlessly many problems introduced/explored
%\end{itemize}

%TODO: table of contents for figures, or theorems, definitions, etc.
%
%TODO: at some point i also need to figure out whether/how to make a glossary/index


\mainmatter

\chapter{Introduction}

The basic question of computational complexity---``how hard is this problem for
a computer to solve?''---is central to nearly every topic in computer science.
And yet the formalisms of complexity theory often seem, in my own experience,
intimidatingly abstract, phrased in terms of intangible models of computation
such as non-deterministic Turing machines and oracles.

The remedy, I believe, lies in studying complexity theory through the lens of
\emph{puzzles} and \emph{games}.  Not only do they provide a concrete grounding
for the abstractions, they also offer a particularly insightful, accessible,
and most importantly fun approach to understanding complexity theory.  In fact,
many of the most popularly known and appreciated results in complexity theory
are those about so-called ``\NP-complete puzzles'', such as Sudoku, and
``\PSPACE-complete games'', such as Checkers and Go.

This thesis emphasizes that approach in its exploration of a particularly
foundational, yet often overlooked, ladder of complexity classes known as the
\emph{polynomial hierarchy}.  \NP{} is the class of (one-player) ``puzzles'',
and \PSPACE{} is the class of (two-player) ``games'' of polynomial length; the
polynomial hierarchy, then, lies in the middle, encompassing games of
\emph{fixed} length.  Through this lens, the (in)famous \P-vs-\NP{} question is
but the first in a ladder of questions that are, arguably, just as crucial and
impactful.

%The polynomial hierarchy is as central to
%complexity theory as the \P-vs-\NP{} problem is well-known.

\section{Overview}

This document is structured as follows.  First, \cref{ch:background,ch:boolean}
establish preliminary background concepts and conventions adopted throughout
this thesis.  Next, \cref{ch:circuit} lays the central theoretical groundwork,
defining the \emph{polynomial hierarchy} through a fundamental family of
problems known as the \Problem{Circuit Satisfiability} games.  Next,
\cref{ch:misc} explores a novel family of games generalized from the
\Problem{Graph 3-Colorability} puzzle and establishes \emph{hardness} bounds on
each of those games.  Finally, \cref{ch:conclusion} concludes by discussing the
future directions of this work and its broader implications.

\section{Prior work and inspirations}

Much of the background exposition on complexity theory referenced in this thesis
is reproduced from Christos Papadimitriou's textbook,
\citet{papadimitriou.cc} (though many of the foundational ideas were
originally introduced/proven elsewhere, e.g.
\citet{cook.np,levin.np,stockmeyer.ph}), reframed through the
puzzles-and-games perspective and supplemented with a few comments on intuition.

The main family of games explored in this thesis, fixed-turn
\Problem{3-Colorability} games (\cref{ch:misc}), is a generalization of
(one-turn) \Problem{3-Colorability}, a well-known \NP-complete puzzle originally
proven \NP-complete by \citet{karp.np}.  Others have studied (multi-turn)
game generalizations of \Problem{3-Colorability}, but all versions that I've
encountered are \PSPACE-complete, in which the number of turns played during the
game scales proportionally with the size of the graph
\citep{bodlaender.coloring,bh.placement,kbd.impartial,cpss.coloring,schaefer.games}. As far as I'm aware, the variations I explore here—with fixed
numbers of turns regardless of the size of the graph—is unexplored, and the main
theorem about its \SigmaP k-completeness (\cref{th:yayay}) is novel.  The basic
idea underlying my proof is the composition of two well-known results:
\begin{itemize}[nosep]
  \item \citet{karp.np}'s classic proof of the \NP-hardness of the \Problem{3-Colorability} puzzle, via a reduction from \Problem{3CNF-Satisfiability};
  \item 's transformation from boolean circuits to equivalent 3CNF-clauses.
\end{itemize}

Without further ado, let's begin.


%TODO: outline/overview of chapters, after those chapters are written

%TODO: also give general citations here, e.g. papadimitriou for many
%foundational background info, etc.
%
%TODO: notation table also belongs in this chapter i think

%p-vs-np well known, polynomial-hierarchy central

%puzzles and games; hierarchy lies in the interstices.  we examine a few
%interesting (by no means exhaustive, or even close to comprehensive)
%np-complete puzzles with pspace-complete analogues, and we









%Famously central to the theory of computational complexity is the \P-vs-\NP{}
%question, and essential to our understanding of that question is the study of
%\NP-complete problems such as the Boolean Satisfiability puzzle, the Graph
%Colorability puzzle, and countless more.  Puzzles like these, which nearly any
%layperson can appreciate, offer a particularly insightful, intuitive, and
%\emph{fun} lens through which to study computational complexity. Explorations
%of more complex problem-classes such as \PSPACE{} can be similarly approached
%through the study of strategic decision \emph{games} such as Othello, Checkers,
%and Go.
%
%What lies in the interstices between \emph{puzzles} and \emph{games}?  How do
%we take a puzzle and generalize it into a game, and what are the puzzle-games
%we encounter along the way?  And how hard exactly are these puzzle-games to
%decide?  These questions are the focus of my thesis.

%So far, I have explored these questions from three angles:
%\begin{enumerate}
%
%  \item \label{itm:intro.q.generation} Puzzle generation.  If I wish to solve a
%    puzzle, you can play a game with me by constructing puzzle \emph{instances}
%    for me to solve.  For instance, \emph{solving} Sudoku is an \NP-complete
%    problem; your task is to \emph{generate} (partially-filled) Sudoku boards
%    for me to solve.
%
%    How hard is it to do so?  Moreover, how hard is it to generate \emph{good}
%    puzzle instances, for various definitions of \emph{good} (sufficiently
%    challenging to solve, or having unique solutions, or solvable/unsolvable by
%    certain strategies)?
%
%    % lauren sanchis
%
%  \item \label{itm:intro.q.pspace} \PSPACE-complete games derived from
%    \NP-complete puzzles.  A canonical \NP-complete puzzle is the \Problem{sat}
%    (Boolean Satisfiability) puzzle: given a Boolean formula \(\phi(x_1, \dots,
%    x_n)\), does there exist an assignment to its inputs \(x_1, \dots, x_n\)
%    such that \(\phi(\dots) = 1\)? In an analogous game, two players alternate
%    turns assigning \(x_1, \dots, x_n\); player 1 wins if \(\phi(\dots)=1\),
%    and player 2 wins if \(\phi(\dots)=0\).  Does either player have a
%    (guaranteed) winning strategy?  This game, known as \Problem{qsat}
%    (Quantified Satisfiability), is a canonical example of a \PSPACE-complete
%    game.
%
%    Can other \NP-complete puzzles be similarly generalized into games?  Will
%    those games also be \PSPACE-complete?
%
%    % schaefer
%
%  \item \label{itm:intro.q.ph} Fixed-turn games and the polynomial hierarchy.
%    In between the complexity classes \NP{} and \PSPACE{} lies a chain of
%    increasingly-complex problem-classes known as the \emph{polynomial
%    hierarchy}.  In some cases, problems in the polynomial hierarchy may be
%    thought of as game generalizations of \NP-complete puzzles with a
%    \emph{fixed} number of turns.  For instance, in a two-turn version of
%    \Problem{sat}, inputs are partitioned into two (disjoint) groups \(X_1\)
%    and \(X_2\); on turn 1, player 1 assigns \(X_1\), and on turn 2, player 2
%    assigns \(X_2\).  As before, player 1 (respectively 2) wins if
%    \(\phi(\dots) = 1\) (respectively \(0\)).  Determining whether player 1 has
%    a winning strategy is complete for a complexity class known as
%    \(\SigmaP2\), which lies just above \NP{} in the hierarchy, and analogous
%    games with \(k\) turns are \(\SigmaP k\)-complete.
%
%    Do polynomial-hierarchy generalizations of other \NP-complete puzzles
%    exist?
%
%\end{enumerate}
%
%\Cref{ch:progress} discusses my progress so far in each of these areas.
%Questions \ref{itm:intro.q.generation} and \ref{itm:intro.q.pspace} have been
%explored in-depth by others, while question \ref{itm:intro.q.ph} appears to be
%scarcely explored.  As such, I provide only brief summaries of/reflections on
%the existing work pertaining to \ref{itm:intro.q.generation} and
%\ref{itm:intro.q.pspace}.  Meanwhile, I describe in greater detail question
%\ref{itm:intro.q.ph}, which is the focus of my explorations so far.
%
%\Cref{ch:future} summarizes the primary questions \& goals that will guide
%my exploration next semester.
%
%Finally, \cref{ch:bib} contains an annotated bibliography of existing work
%pertaining to each of these topics.



\chapter{The basics}

The fundamental question driving the study of computational complexity theory
is, ``how difficult are certain problems for computers to solve?''  In order to
answer this question precisely, we must start by figuring out what exactly it
asks.  That is, formally, what do we mean by \emph{difficulty}?  For that
matter, what constitutes a \emph{problem}?  What counts as a \emph{computer}?

Conventionally, \emph{computers} are formalized as Turing machines, with
\emph{difficulty} being measured by the number of Turing machine execution
steps.  For the purposes of this thesis, we avoid delving into the formalism of
Turing machines.  Instead, we assume an informal notion of computers given by
any algorithm or procedure straightforwardly implementable in modern,
high-level programming languages such as C/C++, Python, Java, etc.  Detailed
treatment of the relevant formalisms may be found in \textcite[Chapter
2]{papadimitriou.cc}.  In particular, there are theorems \parencite[Theorem
2.5]{papadimitriou.cc} showing that modern CPU/RAM-based computer architectures
are, for our purposes, equivalent to Turing machines, thereby justifying the
informal approach we take here.

In the following sections, we discuss what exactly constitutes a
\emph{problem}, how we describe the complexity (i.e., difficulty) of problems,
and how we categorize problems by difficulty into \emph{complexity classes}.

%emphasize intuitive descriptions of
%algorithms in terms of modern, ``high-level'' programming concepts exhibited by
%programming languages such as Python.

%An
%alternative treatment uses ``random access machines'', which mimic modern
%CPU/RAM-based computer architectures.  In this thesis, we avoid delving into
%these formal details.

%in terms of modern, high-level programming concepts

%In the conventional formalism, computers are modeled as Turing machines.
%Difficulty, then, refers to the number of execution steps required by a Turing
%Machine to solve a problem.  Alternatively, computers could be modeled as
%``random access machines'' \parencite[Section 2.6]{papadimitriou.cc}, which
%mimic modern CPU/RAM-based computer architectures.  For our purposes, the two
%models of computation are equivalent \parencite[Theorem 2.5]{papadimitriou.cc}.

%Formal treatments of these
%definitions are found in \textcite[Chapter 2]{papadimitriou.cc}

%Of particular
%note, \textcite[Theorem 2.5]{papadimitriou.cc} shows that these two models of
%computation are, for our purposes, equivalent.  Therefore, for our purposes, we will
%assume the latter model, allowing us to think in terms of modern programming
%patterns and give

%By \emph{hardness}, what we really mean is: given a problem input
%encoded in \(n\) bits, how much computational time, asymptotically with respect
%to \(n\), is required to solve that problem?  In order to discuss this question
%precisely, we have to clearly define what we mean by ``computational time'',
%and, for that matter, what we mean by ``computer''.  A traditional approach
%takes ``computer'' to mean Turing Machines and ``time'' to be Turing Machine
%execution steps; another approach defines ``computer'' via modern-day,
%CPU/RAM-based architectures, with ``time'' given by CPU instruction cycles.

%relatively \emph{informal} descriptions of algorithms.

\section{Decision problems}

The simplest flavor of computational problem is a \emph{decision problem}, or a
yes/no question: given an input \(X\), does \(X\) satisfy certain conditions?
Here are some examples of decision problems:
\begin{itemize}[nosep]
  \item Given an integer \(K\), is \(K\) even?
  \item Given a string of letters \(S\), is \(S\) a palindrome?
  \item A silly decision problem, but nevertheless a valid one: given any input
    \(X\), always return ``yes''.
\end{itemize}

In order for a yes/no question to qualify as a decision problem, it must be
stated in terms of an arbitrary input.  For instance, consider the following
question:
\begin{itemize}[nosep]
  \item Is \(314159\) a prime number?  (Answer: yes.  Proof: see WolframAlpha.)
\end{itemize}
This is a yes/no question, but it takes no inputs (the value \(314159\) is not
an input; it is merely part of the question statement).  In this sense, it is
computationally uninteresting: in order to solve this question, an algorithm
only needs to return the fixed answer ``yes''.  In contrast, what we're really
interested in is the general problem of primality testing:
\begin{itemize}[nosep]
  \item Given an arbitrary positive integer \(K\), is \(K\) prime?
\end{itemize}

We formalize the definition of decision problems below.

\begin{definition}{(decision) problem}{}

  A \Term{decision problem} is a function
  \(Π\colon\Set{0,1}^*→\Set{\text{yes},\text{no}}\).  Equivalently, a
  \Term{decision problem} is the set \(Π⊆\Set{0,1}^*\) comprising exactly the
  inputs, a.k.a. \Term{instances}, that result in ``yes'' answers.

  That is, for any input \(X∈\Set{0,1}^*\), we say \(X∈Π\) (in the \emph{set}
  sense) if \(Π(X)=\text{yes}\) (in the \emph{function} sense), and \(X∉Π\) to
  mean \(Π(X)=\text{no}\).

  \begin{aside}
    Formally, inputs to decision problems are always encoded as binary strings.
    Essentially, this requirement follows from the fact that all modern
    computers encode data in binary anyway.  Furthermore, it allows us to
    rigorously discuss notions such as \emph{input size}.  This is an important
    formal detail, but for the most part, we avoid dealing with any binary
    encoding/decoding technicalities.  We mention this detail here only to
    clarify the role of \(\Set{0,1}^*\) in the definition above.
  \end{aside}

\end{definition}

There is another notion of \emph{problems}, called \Term{function problems}, as
arbitrary (binary-encoded) functions \(\Set{0,1}^*→\Set{0,1}^*\).  However,
function problems seem to generally receive less attention than decision
problems, perhaps because decision problems are conceptually simpler but still
versatile enough to capture the core ideas of complexity theory.  Whatever the
reason, this thesis abides by that tradition.  As such, the vast majority of
the problems examined in this thesis are decision problems, so for convenience
we will simply say ``problems'' to mean decision problems, unless otherwise
specified.

\section{Complexities and classes}

When we ask how difficult a problem is, we are essentially asking, how much
time (or other resources, such as memory) does a computer need to solve that
problem?  Of course, the answer depends on the input: some inputs are easy to
solve, and others are harder.  Certainly, we expect the difficulty to scale
with input size: the larger the input, the more work it generally takes an
algorithm to process it.  Thus, the complexity of a problem is given as a
function of the input size.  Specifically, we ask, if an algorithm is given an
input string (recall, encoded in binary) of length \(n\), how much time in the
worst case is required, as a function of \(n\)?

However, exact function bounds are unnecessarily sensitive to pedantic
technicalities, e.g., slight variations in implementations of the same
algorithm, or specific details in the formal models of ``computer''. Instead,
loosely speaking, we are mostly interested in how these costs asymptotically
\emph{scale} as the input size gets large.  Thus, we categorize problems with
``similar'' complexities into \emph{complexity classes}.

So then, what counts as \emph{similar}?  As a starting approximation, we assert
that \emph{polynomials are small}: any algorithm whose running time is bounded
by some polynomial function is considered relatively ``fast''; problems with
polynomial-time solutions are considered relatively ``easy''.  We formalize
this idea in the definition of the complexity class \P{} below.

\begin{definition}{Polynomial-time problems, \P}{}

  Let \(A\) be an algorithm computing some (decision) problem (i.e., it takes a
  binary string as input and returns ``yes'' or ``no'').  We say \(A\) runs in
  \Term{polynomial time} if there exists some polynomial \(p\) such that, on
  any input \(X∈\Set{0,1}^*\), the algorithm \(A\) terminates in \(≤p(\Abs X)\)
  steps.

  The complexity class \P{} is the set of (decision) problems correctly
  solvable in polynomial time.

  \begin{aside}
    For contrast, we say an algorithm is \Term{super-polynomial} if its running
    time cannot be bounded by some polynomial.  Examples of super-polynomial
    functions include \(n^{\log n}\), \(2ⁿ\), etc.
  \end{aside}

\end{definition}

To be clear, taking polynomials to mean ``easy'' is a very crude rule-of-thumb:
there are important practical subdivisions \emph{within} \P{} that this
categorization plainly ignores (e.g., linear-time vs quadratic-time); there are
also a few notable examples of super-polynomial-time algorithms that are, by
this rule, slow, but quite efficient \emph{in practice} (e.g., the simplex
algorithm for linear programming).  Nevertheless, this delineation remains an
extremely useful (and arguably elegant) starting point for the classification
of problems.

% TODO examples of 𝐏 problems, and perhaps discuss papadimitriou theorem 2.5
% again with more precision?

\section{Hard problems and reductions}

Above, we establish that a problem is considered \emph{easy} if it has a
polynomial-time solution.  Hard problems, then, are those without
polynomial-time solutions… right?  Sure.  But how do we go about showing that a
problem is actually hard?  And how hard, exactly?

For an easy problem, proving \emph{existence} of a polynomial-time algorithm is
straightforward---simply construct one.  On the other hand, for a problem that
appears to be hard, we would have to prove \emph{non-existence} of a
polynomial-time algorithm---that it is \emph{impossible} to find a
polynomial-time algorithm. In general, this is incredibly difficult to show;
this difficulty is a large part of why the infamous \P-vs-\NP{} question
remains unsolved.

Thus, we take a different approach to understanding hard problems: by comparing
them to each other.

TODO: simple examples of two problems that reduce to each other, then
definition of reduction.  then definition of completeness.


Suppose we have two problems, \(Π₁\) and \(Π₂\).


%\section{Puzzles, non-determinism, and \NP}

%Consider the following problem.
%\begin{itemize}
%  \item Given a graph \(Γ\) and a positive integer \(k\), does \(Γ\) contain a
%    clique of size \(k\)?
%\end{itemize}


%\section{Hard problems and completeness}

%Notationally, we say \(X\) is \emph{in} the problem \(L\), or \(X\in L\), if
%the answer is yes; otherwise, we say \(X\notin L\).
%
%An example of a decision problem is the graph reachability problem:
%\begin{definition}[\Problem{reachability}]%
%  Given a graph with \(n\) vertices \(v_1, \dots, v_n\), does there exist a
%  path connecting \(v_1\) to \(v_n\)?
%\end{definition}
%This problem may be solved using simple graph-search algorithms such as
%Breadth-First/Depth-First Search, whose asymptotic running time is \(\O(n)\)
%---that is, bounded by a linear function of \(n\).  As such, this problem is
%considered relatively ``easy'' to solve.
%
%More generally, \Problem{reachability} belongs to the class of decision
%problems known as \P:
%\begin{definition}[\P]%
%  The class of decision problems whose solution runtime is bounded by a
%  polynomial function of the input length.
%\end{definition}
%We consider problems in \P{} to be ``easy''---at least, from the standpoint of
%computational complexity.
%
%Another example of a decision problem is the Hamiltonian path problem:
%\begin{definition}[\Problem{hamiltonian-path}]%
%  \label{def:hamiltonian-path} Given a graph with \(n\) vertices, does it
%  contain a Hamiltonian path (i.e., a path that visits each vertex exactly
%  once)?
%\end{definition}
%This problem is not known to be in \P.  In fact, the best known algorithms
%solving \Problem{hamiltonian-path} are essentially brute-force guess-and-check:
%\emph{guess} a possible Hamiltonian path (e.g., by writing down some
%permutation of the vertices), then \emph{check} that it is valid (e.g., that
%each pair of adjacent vertices in the guessed path are actually connected by an
%edge in the graph).  In the worst case, if our guesses are really unlucky, we
%may have to repeat up to \(n!\) iterations, which is definitely not polynomial.
%However, setting aside the cost associated with brute-forcing guesses, note
%that individual \emph{checking} steps \emph{do} run in polynomial time.
%Problems like this, which are solvable via guess-and-check, where the ``check''
%problem is in \P, belong to a class of problems known as \NP:
%\begin{definition}[\NP]%
%  \label{def:np} A decision problem \(L\) is in \NP{} if\dots
%  \begin{nested}
%    there exists a corresponding decision problem \(L'\in\P\) (intuitively: the
%    ``check'' problem) and a polynomial \(p\) such that\dots
%    \begin{nested}
%      for all input strings \(x\)\dots
%      \begin{nested}
%        \(x \in \NP\) if and only if\dots
%        \begin{nested}
%          there exists a ``guess'' \(g\) with length \(\Abs g \le p(\Abs x)\)
%          such that \((x, g) \in L'\) (intuitively: \(g\) passes the
%          ``check'').
%        \end{nested}
%      \end{nested}
%    \end{nested}
%  \end{nested}
%
%  Note that the \(\Abs g \le p(\Abs x)\) requirement is present in order to
%  ensure that the guesses are not so obscenely long as to abuse the idea of
%  ``efficient'' checking.  This requirement is not central to understanding the
%  definition of \NP{} but is nevertheless an important technical subtlety.
%\end{definition}
%
%The infamous \P-vs-\NP{} open question asks: is \NP{} truly more difficult than
%\P?  Does there exist some problem in \NP{} that definitively cannot be solved
%within polynomial time?  I, a baby undergraduate, am not in the business of
%answering that question.
%
%As such, the best we can do to determine the difficulty of a given problem is
%to compare them to other problems, deriving a \emph{relative} ordering telling
%us which problems are easier/harder than other ones.  To this end, we must
%define what easier/harder means---intuitively, we think of a problem \(L_1\) as
%easier than another problem \(L_2\) if knowing how to solve \(L_2\)
%automatically also tells us how to solve \(L_1\), with minimal
%(polynomially-bounded) overhead.  More precisely:
%\begin{definition}[reductions]
%  \label{def:reduction}
%  Let \(L_1\) and \(L_2\) be decision problems.  We say \(L_1\) is
%  \emph{reducible to} \(L_2\), or that \(L_1\) is \emph{at least as easy as}
%  \(L_2\)'', denoted \(L_1 \le L_2\), if\dots
%  \begin{nested}
%    there exists a function \(f\), called a \emph{reduction}, converting input
%    strings for \(L_1\) to inputs for \(L_2\), such that \(f\) is computable
%    within polynomial time, and\dots
%    \begin{nested}
%      for any input \(x_1\)\dots
%      \begin{nested}
%        \(x_1 \in L_1\) if and only if \(x_2 \in L_2\).
%      \end{nested}
%    \end{nested}
%  \end{nested}
%
%  Note that this definition of reductions is slightly different than the one
%  given in \textcite{papadimitriou.cc}, whose requirement on \(f\) is that it
%  is computable in \emph{logarithmic-space} rather than polynomial-time.
%  However, for the purposes of this project, the distinction between the two is
%  unimportant.
%\end{definition}
%
%This notion of comparison also gives us a good way of comparing problems to
%entire classes:
%\begin{definition}[hardness and completeness]%
%  \label{def:hard-complete}
%  Let \(\mathbfit C\) be a complexity class.
%  \begin{itemize}[nosep]
%    \item A problem \(L\) is \emph{hard for \(\mathbfit C\)}, or
%      \emph{\(\mathbfit C\)-hard}, if \(L\ge K\) for every \(K\in\mathbfit C\).
%    \item A problem \(L\) is \emph{complete for \(\mathbfit C\)}, or
%      \emph{\(\mathbfit C\)-complete}, if \(L\) is \(\mathbfit C\)-hard
%      \emph{and} \(L\in\mathbfit C\).
%  \end{itemize}
%\end{definition}
%
%In particular, \emph{complete} problems for a class \(\mathbfit C\) are at
%least as hard as everything else in \(\mathbfit C\) and simultaneously
%themselves \emph{in} \(\mathbfit C\).  In this sense, for any complexity class,
%its complete problems are its \emph{hardest} problems, giving us an effective,
%``exact'' characterization of the class in terms of its problems.
%
%This approach to characterizing complexity classes is the driving motivation
%behind our exploration of puzzles and games.
%
%%\todo[inline]{unfinished.  formalism of turing machines, decision problems,
%%  oracles \& the definition of polynomial hierarchy, proofs of completeness of
%%  SAT \& QSAT for classes in the polynomial hierarchy.  I imagine this stuff
%%  will be needed in the final thesis; is it needed also for the midyear
%%report?}
%%
%%\begin{definition}[decision problem/language]%
%%  A \textbf{decision problem} is a yes/no question posed on binary input
%%  strings, or problem \textbf{instances}.  As such, we may think of a decision
%%  problem as a mapping
%%  \[
%%    L \colon \Set{0, 1}^* \to \Set{\text{yes}, \text{no}}.
%%  \]
%%
%%  More commonly, we associate a problem with its ``yes'' instances, the set of
%%  which is a \textbf{language}:
%%  \[
%%    L(L) = \SetBuilder* {x \in \Set{0, 1}^*} {L(x) = \text{yes}}.
%%  \]
%%  Here, for clarity, we are distinguishing notationally between \(L\) and
%%  \(L(L)\), but in general we conflate the two notions and refer to both as
%%  the problem \(L\).
%%\end{definition}
%%
%%\begin{definition}[\NP]
%%  \NP{} is the class of problems solvable by a \emph{non-deterministic} Turing
%%  machine in \emph{polynomial time}.
%%\end{definition}
%%
%%
%%
%%

\chapter{A primer on boolean logic}
\label{ch:boolean}

Mathematical logic is founded on true-or-false statements---statements such as:
\begin{itemize}[nosep]
  \item property \(A\) is \emph{true} when condition \(B\) is \emph{false},
  \item property \(X\) is \emph{true} when both condition \(Y\) and condition
    \(Z\) are \emph{true},
\end{itemize}
and so on.  Boolean logic refers to the algebra of how \emph{truthiness} and
\emph{falsiness} combine and transform under various logical operations.

It is no surprise, given the foundational role of booleans in mathematical
logic, that they also underpin all computational logic. For instance, all
modern computer architectures deal with data encoded in binary \(0\)s
(\emph{false}) and \(1\)s (\emph{true}).  Furthermore, it follows that
everything we conceive of as ``computer'' can be represented as boolean
circuits---because, essentially, they literally are boolean circuits.

% TODO: this observation underlies importance of boolean puzzles/games, etc.; how this comes up later, blah

In this short chapter, we outline some basic definitions and facts about
boolean-logical operations and circuits, along with some notational conventions
used throughout the rest of this thesis.

\begin{definition}{basic boolean operations: \NOT, \AND, \OR}{}

  \begin{description}

  \item[\NOT] takes one input and outputs its opposite value.  In
    boolean-algebraic expressions, we denote \NOT{} with the symbol \(¬\).
    \[
      ¬\colon\Set{\True,\False}→\Set{\True,\False} \qquad
      ¬x =
      \begin{cases}
        \True & x=\False \\
        \False & x=\True
      \end{cases}.
    \]
    The \NOT{} operation is also commonly known as \Term{negation}.

  \item[\AND] takes two inputs and outputs \True{} if and only if \emph{both}
    of its inputs are \True.  We denote \AND{} with the symbol \(∧\).
    \[
      ∧\colon\Set{\True,\False}²→\Set{\True,\False} \qquad
      x∧y =
      \begin{cases}
        \True & x=y=\True \\
        \False & \text{otherwise}
      \end{cases}.
    \]

    For convenience, we sometimes omit the \(∧\) and simply denote \AND{} by
    concatenating the operands, as in \(xy\) instead of \(x∧y\).  (This
    notation looks like multiplication because it is: if we represent boolean
    values with \(\Set{1,0}\) instead of \(\Set{\True,\False}\), then
    \(x∧y=x⋅y\).)

    \AND{} is also known as the \Term{conjunction} operation.

  \item[\OR] takes two inputs and outputs \True{} if \emph{at least one} of its
    inputs are \True.  We denote \OR{} with the symbol \(∨\).
    \[
      ∨\colon\Set{\True,\False}²→\Set{\True,\False} \qquad
      x∨y =
      \begin{cases}
        \False & x=y=\False \\
        \True & \text{otherwise}
      \end{cases}.
    \]

    \OR{} is also known as the \Term{disjunction} operation.

  \end{description}

  Notationally, \(∧\) takes higher precedence than \(∨\).  For instance, we
  interpret \(x∨y∧z=x∨yz=x∨(y∧z)\), and so on.

  \begin{aside}
    I personally find the \(∧\) and \(∨\) symbols for \AND{} and \OR{} quite
    easy to mix up with each other.  Here's a mnemonic that helps me remember
    which is which:
    \begin{itemize}[nosep]
      \item \(∧\) looks like the \(\mathrm{\scriptstyle A}\) in \AND, so \(∧\)
        means \AND…
      \item \(∨\) is the other one.
    \end{itemize}
  \end{aside}

\end{definition}

\section{Algebraic properties of \(¬,∧,∨\)}

What algebraic behaviors do \(¬\), \(∧\), and \(∨\) exhibit?

\paragraph{Commutativity \& associativity} It follows straightforwardly from
their definitions that they are both commutative and associative.  In general,
for any \(x₁,x₂,\dotsc,xₙ∈\Set{\True,\False}\),
\begin{align*}
  ⋀ᵢ₌₁ⁿ xᵢ &= x₁∧\dotsb∧xₙ = \text{\True{} if and only if \emph{every one} of
  \(x₁,\dotsc,xₙ\) is \True}, \\
  ⋁ᵢ₌₁ⁿ xᵢ &= x₁∨\dotsb∨xₙ = \text{\True{} if and only if \emph{at least one} of \(x₁,\dotsc,xₙ\) is \True}.
\end{align*}

\paragraph{Distributivity} Another interesting, sometimes useful, property of
\(∧\) and \(∨\) is that they distribute over each other.  For all
\(x,y,z∈\Set{\True,\False}\),
\[
  x∧(y∨z) = (x∧y)∨(x∧z), \qquad
  x∨(y∧z) = (x∨y)∧(x∨z).
\]

%\begin{aside}
%  Here are two intuitive examples demonstrating this distributivity.
%
%  \begin{itemize}
%
%    \item Consider the statement, ``Alex and (either Blake or Charlie) ate
%      pizza'', encoded as the boolean statement \(A(B∨C)\).  What are the
%      possible combinations of pizza-eaters?
%
%      The answer: either Alex and Blake, or Alex and Charlie.  That is,
%      \[
%        A(B∨C) = AB∨AC.
%      \]
%
%    \item Consider the statement, ``either Alex is a vegetarian, or Charlie and
%      Blake both are''.
%
%  \end{itemize}
%
%
%\end{aside}

%\section{Computing arbitrary boolean functions}
%
%Any boolean function can be expressed in terms of the three operators
%\(¬,∧,∨\).


\subsection{DeMorgan's identities}

Consider the statement, ``\(x,y\) are both \False''.  There are two equivalent
ways to express this statement algebraically:
\begin{itemize}[nosep]
  \item \(x\) is \False, and \(y\) is \False: \(¬x∧¬y\).
  \item Neither of \(x,y\) is \True: \(¬(x∨y)\).
\end{itemize}
The equivalence of these two expressions gives rise to an identity: for all
\(x,y∈\Set{\True,\False}\),
\[
  ¬x∧¬y = ¬(x∨y).
\]

Similarly, the statement ``at least one of \(x,y\) is \False'' can be
expressed in two ways,
\begin{itemize}[nosep]
  \item \(x\) is \False, or \(y\) is \False: \(¬x∨¬y\).
  \item \(x,y\) are not both simultaneously \True: \(¬(x∧y)\).
\end{itemize}
This equivalence gives rise to a dual identity,
\[
  ¬x∨¬y = ¬(x∧y).
\]

A particularly useful consequence of these DeMorgan identities is that having
all three logical operations is \emph{redundant}.  We didn't need to define all
three as the basic building-block operations; having only \NOT/\OR{} or
\NOT/\AND{} suffices, since the third operation can simply be constructed in
terms of the other two:
\[
  x∧y = ¬(¬x∨¬y), \qquad
  x∨y = ¬(¬x∧¬y).
\]

We make use of this convenience later in chapter \cref{ch:misc}, when we try to
embed boolean logic within other ``boolean-like'' systems such as graph
3-colorings.

% The usefulness of this redundancy
%
% that if we were trying to simulate boolean logic within some other system (we
% explore this idea later in more detail when we explore reductions from boolean
% circuits in chapters TODO),
%
% universality of not/and and not/or

% TODO miscellaneous identities?


% TODO boolean logical operations can compute arbitrary boolean functions

\section{Boolean circuits}

Boolean \emph{expressions} such as \(¬x∧y\) are one way to specify computations
on boolean variables.  \emph{Circuits} generalize expressions by essentially
chaining together a pipeline of expressions, allowing intermediate results at
each stage to be saved and reused.  To illustrate, consider the following
example expression:
\[
  ϕ(x₁,x₂,y₁,y₂,z₁,z₂)
  = (x₁∨x₂)(y₁∨y₂) ∨ (y₁∨y₂)(z₁∨z₂) ∨ (z₁∨z₂)(x₁∨x₂).
\]
Notice that each \((□₁∨□₂)\) term appears twice in the expression, making the
expression inefficient to evaluate (each repeated term would be unnecessarily
recomputed), not to mention cumbersome to specify.  A more elegant way to
specify this computation is to store and reuse intermediate terms in the
expression:
\begin{align*}
  X &= x₁∨x₂, \\
  Y &= y₁∨y₂, \\
  Z &= z₁∨z₂, \\
  ϕ &= XY∨YZ∨ZX.
\end{align*}
This chain of assignments may be visualized as a sort of data-processing
``pipeline'', with intermediate inputs and outputs at each stage:

{

  \tikzset{
    input/.style={
      circle,
      fill,
      inner sep=0pt,
      minimum size=3pt,
    },
    gate/.style={
      draw,
      rounded corners=1em/8,
    },
    pipe/.style={
      rounded corners=1em/2,
      to path={
        (\tikztostart)
        -- ($ (\tikztostart -| \tikztotarget)!1/2!(\tikztostart) $)
        %-- ($ (\tikztotarget)!1em!(\tikztostart |- \tikztotarget) $)
        -- (\tikztotarget)
      },
      ->,
    },
    gates/.style={
      row sep=2em, column sep=4em, matrix of math nodes, nodes=gate,
    },
    wires/.pic={

      \foreach \var in {x,y,z} {
        \coordinate (\var1) at ($ (\var.west) + (-4em,2em/3) $);
        \coordinate (\var2) at ($ (\var.west) + (-4em,-2em/3) $);
        \draw[pipe] (\var1) node[left]{\(\var₁\)} to ($ (\var.north west)!2/3!(\var.west) $);
        \draw[pipe] (\var2) node[left]{\(\var₂\)} to ($ (\var.south west)!2/3!(\var.west) $);
        \node[above right] at (\var.east) {\(\MakeUppercase{\var}=\var₁∨\var₂\)};
      }
      \draw[pipe] (y.east) to (xy.south west);
      %\draw[pipe] ($ (yz.west)!3em!(y.east) $) to ($ (xy.west)!1/2!(xy.south west) $);
      \draw[pipe] (y.east) to (yz.west);
      %\draw[pipe] ($ (zx.west)!3em!(z.east) $) to ($ (yz.west)!1/2!(yz.south west) $);
      \draw[pipe] (z.east) to (yz.south west);
      \draw[pipe] (z.east) to (zx.west);
      %\draw[pipe] ($ (zx.west)!3em!(z.east) $) to ($ (yz.west)!1/2!(yz.south west) $);
      \draw[pipe, over] (x.east) to (zx.north west);
      \draw[pipe] (x.east) to (xy.west);

      \foreach \gate in {xy,yz,zx} {
        \node[above right] at (\gate.east) {\(\MakeUppercase{\gate}\)};
      }

    },
  }

  \begin{center}
    \begin{tikzpicture}
      \matrix[gates, ampersand replacement=\&]{
        |(x)|∨ \&[4em] |(xy)|∧ \\
        |(y)|∨ \& |(yz)|∧ \& |(out)|∨ \\
        |(z)|∨ \& |(zx)|∧ \\
      };

      \pic{wires};
      \draw[pipe] (xy.east) to (out.north west);
      \draw[pipe] (yz.east) to (out.west);
      \draw[pipe] (zx.east) to (out.south west);
      \draw[->] (out.east) -- +(2em,0) node[right]{\(ϕ=XY∨YZ∨ZX\)};
    \end{tikzpicture}
  \end{center}

  This is \emph{essentially} a boolean circuit.  More precisely, in a boolean
  circuit, each variable (e.g., \(x₂\) or \(Y\)) is represented as a \emph{wire}
  carrying a boolean value, and each ``stage'' of computation, called a
  \emph{logic gate}, computes an individual boolean operation.

  For simplicity's sake, we also require that each \AND/\OR{} gate operates on
  exactly two inputs.  Thus the last \OR{} operation \(XY∨YZ∨ZX\) should
  actually be associatively grouped as \((XY∨YZ)∨ZX\).  The corrected circuit
  is shown below:

  \begin{center}
    \begin{tikzpicture}
      \matrix[gates, ampersand replacement=\&]{
        |(x)|∨ \&[5em] |(xy)|∧ \\
        |(y)|∨ \& |(yz)|∧ \&\& |(or')|∨ \\
        |(z)|∨ \& |(zx)|∧ \\
      };

      \node[gate](or) at ($ (xy)!1/2!(or') $){\(∨\)};

      \pic{wires};
      \draw[pipe] (xy.east) to ($ (or.west)!1/3!(or.north west) $);
      \draw[pipe] (yz.east) to ($ (or.west)!1/3!(or.south west) $);
      \draw[pipe] (or.east) to ($ (or'.west)!1/3!(or'.north west) $);
      \draw[pipe] (zx.east) to ($ (or'.west)!1/3!(or'.south west) $);
      \node[above right] at (or.east) {\(XY∨YZ\)};
      \draw[->] (or'.east) -- +(2em,0) node[right]{\(ϕ=(XY∨YZ)∨ZX\)};

    \end{tikzpicture}
  \end{center}

}

%Lastly, we use the algebraic symbols \(∨,∧\) to denote the logic gates in the
%circuit diagram above

Below, we state a precise definition of boolean circuits and introduce some
relevant terminology.

\begin{definition}{boolean circuits}{circuit}

  A \Term{boolean circuit} \(C\) consists of:
  \begin{itemize}
    \item A set of \Term{(circuit-level) input wires}.
    \item A sequence of \Term{logic gates}.  Each logic gate computes a logical
      operation on one or two previously-computed wires and produces its output
      on a new wire.  More precisely, each logic gate defines a new
      (intermediate) \Term{output wire} \(w\), related to previously-defined
      wires by one of the following:
      \begin{itemize}[nosep]
        \item \NOT{} gate: \(w=¬w₁\), where \(w₁\) is an input wire or an the
          output wire of another logic gate specified earlier in the sequence.
        \item \AND{} gate: \(w=w₁∧w₂\), where \(w₁,w₂\) are earlier-defined
          wires.
        \item \OR{} gate: \(w=w₁∨w₂\), where \(w₁,w₂\) are earlier-defined
          wires.
      \end{itemize}
  \end{itemize}
  Finally, exactly one of the output wires is labeled the \Term{circuit-level
  output wire} of \(C\), representing the overall/final output of the circuit.

  Assume the circuit-level input wires are ordered as \(w₁,w₂,\dotsc,wₙ\).  Then
  the circuit \(C\) defines a boolean function \(ϕ_C\colon\TF[n]→\TF\) as
  follows.  Given \((x₁,x₂,\dotsc,xₙ)∈\TF[n]\), assign \(xⱼ\) to \(wⱼ\) for each
  \(j=1,\dotsc,n\).  Then, for each logic gate, in order of specification,
  evaluate the gate's boolean operation on its inputs to compute its output
  value, assigning that value to its output wire.  Finally, after all gates have
  been evaluated, the boolean value of the circuit-level output wire is the
  final output, \(ϕ(x₁,\dotsc,xₙ)\).

  To assist discourse, we say a wire \(w\) \emph{(directly) depends on} another
  wire \(w'\), or that \(w'\) is a \Term{(direct) dependency of} \(w\), if there
  exists some logic gate with \(w'\) as one of its inputs and \(w\) as its
  output.

  \begin{aside}
    We require that logic gates be specified sequentially, operating only on
    previous gate-outputs, to ensure that there are no cyclic dependencies,
    which give rise to ill-defined computations such as \(x=¬x\) (contradictory)
    or \(x=¬y,y=¬x\) (not contradictory, but ill-defined because the output of a
    computation must be unique/deterministic), etc.
  \end{aside}

\end{definition}


%We give a precise definition of
%
%\begin{definition}{boolean circuits}{}
%
%  % TODO consider using switches and lightbulbs in formalism
%
%  A \Term{circuit} consists of a network of such wires and logic gates, with
%  the following requirements for well-defined-ness:
%  \begin{itemize}
%    \item Some wires are marked as circuit-level \Term{inputs}.  These wires
%      may not be the output wire of any gate within the circuit.
%    \item Every wire that isn't a circuit-level input must be the output wire
%      of exactly one logic gate.
%    \item There must be no (directed) cycles in the circuit.  That is, there
%      must not exist gates \(g₁,\dotsc,gₖ\) such that the output of \(g₁\) is
%      an input to \(g₂\), the output of \(g₂\) an input to \(g₃\), …, and the
%      output of \(gₖ\) an input to \(g₁\).
%    \item Finally, there is exactly one wire marked the \Term{output} of the
%      circuit.
%  \end{itemize}
%  Under these requirements, each combination of boolean values assigned to the
%  circuit-level input wires, uniquely determines the circuit-level output's
%  value. Therefore, a circuit with \(n\) inputs defines a function
%  \(\Set{\False,\True}ⁿ→\Set{\False,\True}\).
%
%  To simplify notation, we refer to each circuit and its boolean function by
%  the same name.  That is, if \(C\) refers to a circuit, we denote by
%  \(C(x₁,\dotsc,xₙ)\) the output computed by \(C\) on input values
%  \(x₁,\dotsc,xₙ\).  Occasionally, if we need to disambiguate between a circuit
%  and its function, we denote the circuit \(C\) and its function \(ϕ_C\).
%
%\end{definition}


\chapter{Boolean circuit puzzles and games}

In this chapter, we begin to explore landscape of puzzle-and-game complexity
classes---specifically, the \emph{polynomial hierarchy}---through a series of
games played on boolean circuits.

\section{The \Problem{Circuit Value} problem, and \P}

To set the stage, we start with a relatively ``easy'' problem, known as the
\Problem{Circuit Value} problem, or \Problem{CircVal} for short:

\begin{definition}{\(\Problem{Circuit Value}=\Problem{CircVal}\)}{}

  Let \(C\) be a given boolean circuit with all input wires/variables
  specified. What is the final output value of \(C\)? As a decision problem:
  \(C∈\Problem{Circuit Value}\) if it outputs \True, and \(C∉\Problem{Circuit
  Value}\) if it outputs \False.

\end{definition}

It is well-known that \(\Problem{Circuit Value}∈\P\) (i.e., it is actually
``easy'').  We give one version of a proof below.

\begin{proof}
  We give a polynomial-time algorithm solving \Problem{Circuit Value} below.
  (Note that this is not the most efficient algorithm doing so; we choose it
  here only for its simplicity.)

  \begin{algorithm}{}{}
    \begin{algorithmic}
      \Given{\(C\), a boolean circuit with all inputs fully specified}
      \LComment{Call a wire \emph{finished} if it has been assigned a boolean
        value. Initially, all the input wires are finished, since their values
      were given, and all intermediate and output wires are unfinished.}%
      \While{final output wire is not finished}%
      \ForEach{unfinished logic gate \(g\) in \(C\)}%
      \If{all input wires of \(g\) are finished}%
      \State{compute and assign the output value of \(g\) and to its output wire}%
      \EndIf%
      \EndFor%
      \EndWhile%
      \State \Return value assigned to final output wire%
    \end{algorithmic}
  \end{algorithm}

  We argue that this algorithm terminates in polynomial time.  On each
  iteration of the ``while'' loop, at least one logic gate is guaranteed to
  have all of its inputs done, since there are no cyclic dependencies in the
  circuit.  Thus each iteration of the ``while'' loop finishes at least one
  additional wire.  Therefore, the number of ``while'' iterations is at most
  the number of wires in the circuit, and the work done within each iteration
  is also polynomial with respect to the size of the circuit, so the overall
  algorithm terminates in polynomial time.
\end{proof}

To kickstart the puzzles-and-games perspective, we think of \Problem{Circuit
Value}---and actually, every problem in \P---as a game with \(0\) turns: the
player does nothing, and an (efficient) algorithm automatically decides whether
the player wins or loses.

This seems like a silly (arguably boring) idea.  But, as we see in the next few
sections, this approach allows us to generalize \Problem{Circuit Value} into
very powerful puzzles and games.

\section{The \Problem{Circuit Satisfiability} puzzle, and \NP}

By \emph{puzzle}, we really mean \(1\)-turn games: games in which a player
makes a sequence of ``moves'' on a given ``game board'', and an (efficient)
algorithm then determines whether the player's moves constitute a win.
Formulated as decision problems, the computational puzzle is the yes/no
question:
\begin{center}
  Does the player have a winning strategy?
\end{center}

For example, consider a puzzle-ification of \Problem{CircVal}, where the
circuit's inputs are no longer specified but rather chosen by the player (this
is the ``move'' made by the player).  Recall that the player wins if the
circuit's output is \True.  Thus, when we allow the player to choose inputs, a
winning strategy means a combination of inputs causing the circuit to output
\True.  The decision problem asking whether such a winning move exists is
called \Problem{Circuit Satisfiability}, or \Problem{CircSat} for short:

\begin{definition}{\(\Problem{Circuit Satisfiability}=\Problem{CircSat}\)}{}

  Let \(C\), a boolean circuit, be given. Does there exist a combination of
  boolean input values to \(C\) causing it to output \True?

\end{definition}

\begin{problem}{\Problem{Circuit Satisfiability} / \CircSat}{circ-sat}

  \(C\), a boolean circuit with \(n\) inputs

  \tcblower

  there exists \(X∈\TF[n]\) so that \(C(X)=\True\)

\end{problem}

Briefly: how (computationally) difficult is \Problem{CircSat}?  As it turns
out, nobody knows for sure, but it seems \emph{quite} difficult.  Loosely
speaking, all known algorithms for solving \Problem{CircSat} amount to brute
force with optimizations that enhance performance on ``practical'', real-world
inputs but do not save them from performing poorly in the worst case.
Tentatively, then, most computer scientists suspect that
\(\Problem{CircSat}∉\P\)---i.e., there is no polynomial-time solution for
\Problem{CircSat}.

% SURVEY on pnp opinion: https://dl.acm.org/doi/10.1145/564585.564599
% https://www.researchgate.net/publication/292393040_The_PNP_poll

% TODO maybe cite an up-to-date result about how good the bound is, but whatever

% useful citation about best-known SAT bounds: https://cstheory.stackexchange.com/questions/1060/best-upper-bounds-on-sat

% https://www.sciencedirect.com/science/article/pii/S0304397501001748?via%3Dihub
% 3sat solvable in 1.5^n?



Anyway, back to puzzles.  \Problem{CircSat} is one example of how a \(0\)-turn
game such as \Problem{CircVal} may be generalized into a \(1\)-turn game---a
puzzle.  How can we do this in general?

In the example of \Problem{CircSat}, we do this by making the player supplement
the input to the the \(0\)-turn analog, \Problem{CircVal}.  This approach is
readily generalized.  Given some input \(X\) (the ``game board''), construct a
\(1\)-turn game in which the player specifies a supplementary input \(Y\);
victory is decided by whether the pair of inputs \((X,Y)\) meets the \(0\)-turn
winning condition.  As before, the decision problem asks whether the player can
win---i.e., whether there exists \(Y\) such that the player wins.

The complexity class of problems constructed in this manner is called \NP:

\begin{definition}{\NP}{}

  Let \(Π\) be a decision problem.  We say \(Π\) (the \(1\)-turn \emph{puzzle})
  is in \NP{} if…
  \begin{nest}
    there exists another problem \(Π'∈\P\) (the \(0\)-turn \emph{winning
    condition}) and a polynomial \(p\) such that…
    \begin{nest}
      for each input \(X∈\Set{\True,\False}^*\)…
      \begin{nest}
        \(X∈Π\) if and only if…
        \begin{nest}
          there exists \(Y∈\Set{\True,\False}^*\) such that…
          \begin{nest}
            \((X,Y)∈Π'\) and \(\Abs Y≤p(\Abs X)\).
          \end{nest}
        \end{nest}
      \end{nest}
    \end{nest}
  \end{nest}

  The polynomial-length requirement on \(Y\) is there to prevent the
  player-provided input from unduly distorting the size of the problem.
  Essentially, it enforces the idea that the winning-condition problem \(Π'\)
  scales polynomially with the size of the \emph{game board} \(X\), roughly
  independent of the player's move \(Y\).  (Without this requirement, for
  example, the input string can be padded with a large number of meaningless
  zeroes to artificially inflate its size relative to the cost of computing the
  winning condition.)

\end{definition}

%We can also think of problems in \NP, or puzzles, as problems solvable by
%guess-and-check: guess a move \(Y\), and check whether it meets the winning
%condition.

Unsurprisingly, \(\CircSat∈\NP\). Fitting this observation to the formalism
above, we express \CircSat{} as the set of circuits
\[
  \CircSat=\SetBuilder{
    \text{circuit \(X\) with \(n\) inputs}
  }{
    ∃Y∈\Set{\True,\False}ⁿ\ldotp (X,Y)∈\CircVal
  }.
\]

But what makes \CircSat{} particularly interesting is that it is also
\NP-\emph{complete}.  In other words, \CircSat{} is the hardest of the
\NP{} puzzles: any other \NP{} problem reduces to \CircSat.  This result is
known as the Cook--Levin theorem:

\begin{theorem}{Cook--Levin}{}

  \CircSat{} is \NP-complete.

\end{theorem}

For sake of brevity, we will not reproduce here the full proof of the
Cook--Levin theorem.  Instead, we give some informal intuition about the basic
idea behind the proof and why the result should not be surprising.  As
mentioned in \cref{ch:boolean}, any computer can be expressed in terms of
boolean circuits; in fact, modern computers literally are implemented using
boolean circuits.  Therefore, the execution of any algorithm \(A\) is just a
sequence of circuit computations, one for each time-step of the algorithm.
Thus every \(1\)-turn game is really just a \emph{special case} of the
\Problem{Circuit Satisfiability} problem.

\subsection{\True{} or \False?}

In our definition of \CircSat, we say the player wins if the output of the
circuit is set to \True.  But there is nothing special about \True---we could
have defined the puzzle so that the player wins if the circuit outputs \False;
the two formulations are exactly equivalent in difficulty.  Call the
win-if-\False{} version of the puzzle \Problem{Circuit Falsifiability}:

\begin{definition}{\Problem{Circuit Falsifiability}}{}

  Let \(C\), a boolean circuit, be given.  Does there exist an input
  combination to \(C\) setting its output to \False?

\end{definition}

Do not confuse \Problem{Circuit Falsifiability} with the \emph{complement} of
\Problem{Circuit Satisfiability}, whose answer is ``yes'' when the player's
moves \emph{always} lead to a \False{} output:
\begin{align*}
  \Problem{Circuit Satisfiability} &= \SetBuilder{C}{∃X\ldotp C(X)=\True} \\
  (\Problem{Circuit Satisfiability})\Complement &= \SetBuilder{C}{∄X\ldotp C(X)=\True}
  = \SetBuilder{C}{∀X\ldotp C(X)=\False} \\
  \Problem{Circuit Falsifiability} &= \SetBuilder{C}{∃X\ldotp C(X)=\False}.
\end{align*}

To see that this formulation is equivalent in difficulty to \Problem{Circuit
Satisfiability}, we show that both problems reduce to each other.

\begin{theorem}{}{}

  \Problem{Circuit Satisfiability} and \Problem{Circuit Falsifiability} are
  equivalent in difficulty.

\end{theorem}

\begin{proof}

  Let any circuit \(C\) be given.  Compose its output with a \NOT{} gate,
  forming a new circuit \(C'\) whose output is always opposite that of \(C\).

  Therefore, \(C\) is satisfiable if and only if \(C'\) is falsifiable;
  conversely, \(C\) is falsifiable if and only if \(C'\) is satisfiable.  Thus
  the compose-with-\NOT-gate transformation is a reduction going both ways:
  \begin{align*}
    \Problem{Circuit Falsifiability} &≤ \Problem{Circuit Satisfiability}, \\
    \Problem{Circuit Satisfiability} &≤ \Problem{Circuit Falsifiability}.
    \qedhere
  \end{align*}

\end{proof}

A corollary of this result is that \Problem{Circuit Falsifiability} is also
\NP-complete, and therefore ``characterizes'' \NP.  A corollary-corollary,
then, is that \(\co\Problem{Circuit Falsifiability}\) is \(\co\NP\)-complete.
We leverage this result in the next section, where we introduce a second player
whose goal is, indeed, to \emph{falsify} the circuit.

\section{Two-player circuit games, and the polynomial hierarchy}

Recall, in the \(1\)-turn game \CircSat, a single player assigns inputs to a
given circuit, with the goal of getting the circuit to output \True.  Now, we
introduce a second player, an \emph{antagonist}, working towards the opposite
goal.  The two players now take turns assigning inputs in the circuit; when all
inputs have been assigned, the circuit's final output dictates the winner
(\(\True⟹\text{first player wins}\), \(\False⟹\text{second player wins}\)).
Now, framing this game as a decision problem, we ask the yes/no question,
\begin{center}
  Does the \emph{first} player have a winning strategy?
\end{center}

To start with a concrete example, consider the version of this game with \(2\)
turns.  A circuit \(C\) is given; its (unassigned) inputs are partitioned into
two groups, \(I₁\) and \(I₂\).  Two turns proceed: the first player assigns
values to all inputs in \(I₁\), then the second player assigns values to all
inputs in \(I₂\).  Finally, if the circuit outputs \True, the first player
wins; otherwise, the second player wins.  Now, we ask, does the first player
have a winning strategy?

To be more precise, by \emph{winning strategy}, we mean a move the first player
can make in order to guarantee a win, no matter what the second player plays in
response.  In other words, if the first player plays a winning move, then there
\emph{does not exist} a counter-winning move by the second player.  Thus, in
this example, what we are actually asking is, does there exist \(X₁\) such that
there does not exist \(X₂\) setting \(C(X₁,X₂)=\False\)?  We call this decision
problem \(\CircSat₂\).

\begin{definition}{\Problem{Circuit Satisfiability} with \(2\) turns, a.k.a.
  \(\CircSat₂\)}{}

  Let \(C\), a boolean circuit, be given, with its inputs partitioned into two
  groups \(I₁\) and \(I₂\).  Does there exist some
  \(X₁∈\Set{\True,\False}^{\Abs{I₁}}\) such that…
  \begin{nest}
    there does \emph{not} exist an
    \(X₂∈\Set{\True,\False}^{\Abs{I₂}}\) such that…
    \begin{nest}
      \(C(I₁≔X₁,I₂≔X₂)=\False\)?
    \end{nest}
  \end{nest}

\end{definition}

There is another way to formulate \(\CircSat₂\), in an inductive way.  To aid
this formulation, we introduce the notion of an \emph{augmented circuit}:

\begin{definition}{augmented circuit}{}

  Let \(C\) be a circuit, and let \(I\) refer to a subset of the inputs of
  \(C\).

  Let \(X∈\Set{\True,\False}^{\Abs I}\) be a boolean assignment to the inputs
  in \(I\).  We call the new circuit \(C'\) produced by fixing inputs \(I\) to
  values \(X\) an \Term{augmented circuit \(C'=C[I≔X]\)}.

\end{definition}

Now, we revisit \(\CircSat₂\) in order to formulate it inductively.  On the
first turn, the first player makes an assignment \(X₁∈\TF[\Abs{I₁}]\) to the
inputs \(I₁\).  After that assignment, the remaining circuit is the augmented
circuit \(C'=C[I₁≔X₁]\), whose inputs are just \(I₂\).  The first player's
initial move is a winning move if and only if \(C'\) is now
\emph{unfalsifiable}.
\[
  \CircSat₂ = \SetBuilderLong{
    \text{circuit \(C\) with inputs partitioned into \(I₁,I₂\)}
  }{
    ∃X₁∈\TF[\Abs{I₁}]\ldotp
    C[I₁≔X₁]∉\Problem{Circuit Falsifiability}
  }
\]


This inductive formulation shows a way to construct \(k\)-turn circuit games in
general.  Start with a boolean circuit \(C\), whose inputs are partitioned into
\(k\) groups, \(I₁,I₂,\dotsc,Iₖ\).  The \(k\) turns proceed as follows:
\begin{enumerate}[left=1.5em]
  \item[{[\(1\)]}] On the first turn, the first player assigns values
    \(X₁∈\TF[\Abs{I₁}]\) to the inputs \(I₁\).
  \item[{[\(2\)--\(k\)]}] The remaining \(k-1\) turns proceed inductively, now
    starting with the second player.  It is played on the augmented circuit
    \(C'=C[I₁≔X₁]\), whose inputs are partitioned into \(k-1\) groups,
    \(I₂,\dotsc,Iₖ\).
\end{enumerate}
The first player's move is a winning move if and only if, in the remaining
\((k-1)\)-game, the second player does not have a (counter-)winning strategy.


\begin{definition}{\Problem{Circuit Satisfiability with \(k\) turns}}{}

  Given: a boolean circuit \(C\) with inputs partitioned into \(k\) groups,
  \(I₁,I₂,\dotsc,Iₖ\).


\end{definition}

definitions of sigmap, pip


completeness


%\[
%  \CircSat₂ = \SetBuilder{\text{circuit \(C\)}}{
%    ∃X₁\ldotp (C,X₁)∉\Problem{Circuit Falsifiability}
%  }
%\]


\chapter{Two fun games (and hopefully more)}

In the last chapter, we set the polynomial-hierarchy stage, focusing on the
circuit games \(\CircSatₖ\) as canonical examples of \SigmaP k-complete
problems.  In this chapter, we expand that landscape by exploring two other
(collections of) \SigmaP k-complete games: one played on graph colorings,
another played on set coverings.

It is only due to the time constraints on this thesis that we stop at two:
ultimately, I hope to convey, through the examples presented in this chapter,
the sense that there are many, many \SigmaP k-complete games out there, all of
which intuitive extensions of classic, well-known \NP-complete puzzles.

\section{Graph coloring games}

First, we introduce some preliminary definitions about graphs and colorings.  A
graph is a network of \emph{vertices} connected by \emph{edges}.  Formally:

\begin{definition}{(undirected) graphs}{}

  A \Term{graph} \(Γ\) is a pair \((\Vertices(Γ),\Edges(Γ))\) consisting of:
  \begin{itemize}[nosep]
    \item a finite set of \Term{vertices} \(\Vertices(Γ)\), and
    \item a finite set of \Term{edges} \(\Edges(Γ)⊆\Vertices(Γ)×\Vertices(Γ)\),
      which represent connections between pairs of vertices.
  \end{itemize}

  For our purposes, edges have no directionality.  That is, when specifying an
  edge, the ordering of vertices doesn't matter: \((u,v)\) specifies the same
  edge as \((v,u)\).

  We say that two vertices \(u,v∈\Vertices(Γ)\) are \Term{neighbors}, or that
  they neighbor each other, if \((u,v)∈\Edges(Γ)\).

\end{definition}

% TODO example giraph (fun)

The graph coloring games we explore in this thesis are about assigning colors
to vertices on a graph.  We call such an assignment a \emph{vertex coloring}.
Specifically, for sake of simplicity, we restrict our attention to colorings
that involve only three colors.  The main rule constraining these color
assignments is that neighboring vertices must always be colored distinctly---we
call this the \emph{properness} condition.  These terms are defined precisely
below.

\begin{definition}{vertex 3-colorings, properness}{}%

  Let \(Γ\) be a graph. A \Term{vertex 3-coloring} of \(Γ\) is a map
  \(κ\colon\Vertices(Γ)→\Colors\), which assigns to each vertex one of three
  colors.  In this thesis, we generally just say ``coloring'' to refer to
  ``vertex 3-colorings'', except when specified otherwise.

  A vertex coloring \(κ\) is a \Term{proper} coloring if, for every edge
  \((u,v)∈\Edges(Γ)\), \(κ(u)≠κ(v)\)---i.e., no neighboring vertices share the
  same color.  To simplify discourse, we also call a particular
  edge/neighboring-pair \((u,v)∈\Edges(Γ)\) is \Term{proper} if \(κ(u)≠κ(v)\).
  Thus a proper coloring is one where all edges are proper; an improper
  coloring contains at least one improper edge.

\end{definition}

Having established the basic terminology, we now introduce the graph
(3-)coloring games.

\subsection{The \(0\)-turn game}

The goal of graph coloring games is to assign colors to all vertices so that
the resulting coloring is proper.  To this end, the \(0\)-turn
winning-condition problem is that of checking properness of colorings, called
the \Problem{3-Coloring Properness} problem, or \ColProp{} for short:

\begin{problem}{\Problem{3-Coloring Properness} / \ColProp}{}

  Given a graph and a 3-coloring \(κ\) of it (specified by listing out each
  vertex and its assigned color), determine whether \(κ\) is a proper coloring.

  \tcblower
  \ColProp=\SetBuilder{(\text{graph \(Γ\)},\text{coloring \(κ\)})}{
    ∀(u,v)∈\Edges(Γ)\Q κ(u)≠κ(v)
  }
\end{problem}

In order for \ColProp{} to be usable as a basis for polynomial-hierarchy games,
we must first ensure that it itself is in \P.  Indeed, it is:

\begin{theorem}{\(\ColProp∈\P\)}{}

  \(\ColProp\) is solvable in polynomial time.

\end{theorem}

\begin{proof}

  We describe below a straightforward polynomial-time algorithm computing
  \ColProp.  It iterates through the edges and simply verifies the properness
  condition on each pair of neighbors:

  \begin{algorithm}{a polynomial-time \ColProp{} solver}{}
    \begin{algorithmic}
      \Given{a graph \(Γ\) and a coloring \(κ\colon\Vertices(Γ)→\Colors\)}%
      \ForEach{\((u,v)∈\Edges(Γ)\)}%
      \If{\(κ(u)=κ(v)\)}%
      \LComment{\((u,v)\) is improper!}
      \State{\Return no}%
      \EndIf%
      \EndFor%
      \LComment{all edges have been checked, and no improper ones were found,
      so the coloring is proper}
      \State{\Return yes}%
    \end{algorithmic}
  \end{algorithm}

  The number of edges is, by definition, bounded by the size of the graph, so
  the number of ``for each'' iterations is polynomial. Within each iteration,
  the \(κ(u)=κ(v)\) check runs within polynomial time, so the overall algorithm
  runs in polynomial time as well.  \qedhere

\end{proof}

\subsection{The \(k\)-turn games}

A graph coloring game is played on an initially uncolored graph \(Γ\).  In a
\(k\)-turn game, the graph's vertices are partitioned into \(k\) groups,
\(V₁,V₂,\dotsc,Vₖ\), and players alternate turns assigning colors to the
vertices in each group.  If, on any turn, a player introduces an improper edge
in the (partial) coloring, the other player wins.  If, after all turns, no
improper edges have been introduced---that is, the resulting coloring is
proper---then the \emph{last} player wins.

\begin{definition}{partial (vertex 3-)colorings}{}

  Let \(Γ\) be a graph.  A \Term{partial (vertex 3-)coloring} is a map
  \(κ\colon\Vertices(Γ)→\ColorsOpt\), which \emph{optionally} assigns a color
  to each vertex in \(Γ\) (\None{} means no color is assigned).

  Except where ambiguous, we will simply say ``partial coloring'' to mean
  ``partial vertex 3-coloring''.

  A partial coloring \(κ\) is \Term{proper} if, among the vertices it
  \emph{does} assign a color, there are no improper edges.  That is, for all
  \((u,v)∈\Edges(Γ)\), if both \(κ(u)\) and \(κ(v)\) are not \None, then
  \(κ(u)≠κ(v)\).

\end{definition}








%Given a graph \(G\) along with a \(3\)-coloring on \(G\), is the coloring
%proper?  We can solve this problem by simply checking, for each edge, whether
%the two vertices on that edge have different colors.  The run-time of this
%solution is \(\O(e)\) and therefore polynomially-bounded in the size of \(G\).
%Thus the problem of \emph{checking} whether a given \(3\)-coloring is proper is
%in \P.
%
%\subsection{The \(3\)-coloring puzzle}
%
%The puzzle-ification of this problem comes in the following form:
%\begin{definition}[\Problem{3col}]%
%  Given a graph \(G\), is there a way to properly \(3\)-color the vertices of
%  \(G\)?
%  \[
%    \Problem{3col} = \SetBuilder* G {
%      ∃\,\text{coloring \(C = (c_1,\dotsc,c_n) ∈ \Set{0,1,2}^n\)} \quad \text{\(C\) is proper}
%    }
%  \]
%\end{definition}
%It is straightforward to see from its definition and the fact that
%properness-checking is in \P{} that \(\Problem{3col} ∈ \NP\).
%
%The natural question to ask is: is it also \NP-complete?  After all, earlier,
%we could confidently expect that \NP-completeness from \emph{Boolean} \CircSat{}
%because of the universality of Boolean logic, but, at a glance, it isn't
%obvious that graphs and proper colorings are somehow ``fundamental'' to
%computation as Booleans are.  But, in fact, that is exactly the case:
%\begin{theorem}
%  \Problem{3col} is \NP-complete.
%\end{theorem}
%
%\section{Reduction from CSAT}

\subsection{Key idea: embedding circuits in colorings}

TODO narrative connective text

\begin{definition}{boolean graph}{boolean-graph}%
  A \Term{boolean graph} is a graph \(G\) whose vertices satisfy the following:
  \begin{itemize}
    \item There are three vertices labeled
      \(s_\True,s_\False,s_\Aux∈\Vertices[G]\), called the ``special''
      vertices, joined to each other by a triangle.
    \item There are \(n\) (not necessarily distinct) vertices
      \(i_1,\dotsc,i_n∈\Vertices[G]\), called the ``input'' vertices, each
      joined by an edge to \(s_\Aux\).  Purely for convenience, we allow each
      of the input vertices to be the same vertex as \(s_\True\) or
      \(s_\False\) (but not \(s_\Aux\), since that would require edges from
      \(s_\Aux\) to itself, which we forbid).
    \item There exists a vertex \(o∈\Vertices[G]\), called the ``output''
      vertex, also joined by an edge to \(s_\Aux\).  Again, for convenience, we
      allow \(o\) to coincide with \(s_\True\), \(s_\False\), or any of the
      input vertices.
  \end{itemize}

  Now, let \(κ\colon\Vertices[G]→\Set{0,1,2}\) be an arbitrary \emph{proper}
  3-coloring.  Because the three special vertices \(s_\True,s_\False,s_\Aux\)
  are joined by a triangle, we know that \(κ\) assigns them all three
  (distinct) available colors.

  For each input/output vertex \(v\), since \(v\) neighbors \(s_\Aux\) by
  construction, we know \(κ(v)≠κ(s_\Aux)\); then, since there are only three
  colors, we know \(κ(v)\) must equal \(κ(s_\True)\) or \(κ(s_\False)\).  We
  say the \Term{boolean value} assigned (by \(κ\)) to each input/output vertex
  \(v\) is \(\True\) if \(κ(v)=κ(s_\True)\) and \(\False\) if
  \(κ(v)=κ(s_\False)\).

  To simplify later discussions, we overload/abuse the notation \(κ(v)\) to
  denote the boolean value of \(v\); i.e., \(κ(v)=\True\) if \(v\) shares a
  color with \(s_\True\), etc.
\end{definition}

\begin{definition}{boolean graphs as boolean functions}{boolean-graph-functions}%

  Let \(G\) be a boolean graph.  We say that \(G\) computes a well-defined
  boolean function \(ϕ\colon\Set{\True,\False}^n→\Set{\True,\False}\), if, for
  every combination of boolean values \(x_1,\dotsc,x_n∈\Set{\True,\False}^n\),
  the following hold:
  \begin{itemize}
    \item There exists at least one proper 3-coloring that assigns boolean
      values \(x_1,\dotsc,x_n\) to input vertices \(i_1,\dotsc,i_n\),
      respectively.

      (There exists at least one 3-coloring \(κ\) such that \(κ\) is proper,
      and \(κ(i_j)=x_j\) for each \(j=1,\dotsc,n\).)
    \item Every such coloring assigns the (same) boolean value
      \(ϕ(x_1,\dotsc,x_n)\) to the output vertex \(o\).

      (For every such coloring \(κ\), \(κ(o)=ϕ(x₁,\dotsc,xₙ)\).)
  \end{itemize}
\end{definition}

\begin{example}{A graph computing the boolean identity function}{}%
  The following boolean graph computes the boolean identity function,
  \(ϕ(x_1)=x_1\).
  \begin{center}
    \begin{tikzpicture}[x=3em, y=3em]
      \coordinate[vertex](t) at (120:1);
      \coordinate[vertex](f) at (60:1);
      \coordinate[vertex](a);
      \draw (t) -- (f) -- (a) -- (t);
      \node[vertex label, left] at (t) {\(s_\True\)};
      \node[vertex label, right] at (f) {\(s_\False\)};
      \node[vertex label, left] at (a) {\(s_\Aux\)};

      \coordinate[vertex](i) at (-90:1);
      \node[vertex label, below] at (i.south) {\(i_1=o\)};
      \draw (i) -- (a);
    \end{tikzpicture}
  \end{center}

  TODO is explanation even needed, or is it actually obvious that this works?
\end{example}

\begin{lemma}{\NOT, \OR, and \AND{} graphs}{boolean-operation-graphs}%
  There exist graphs computing each of the basic boolean operations \NOT, \OR,
  and \AND.
\end{lemma}

\begin{proof}

  \tikzset{
    boolean graph/.style={x=3em, y=3em},
    over/.style={
      preaction={draw=white, line width=3pt},
    },
    triangle/.pic={
      \draw (0,0) -- (0,-1) -- (1,0) -- cycle
      (0,0) coordinate[vertex, fill=ks-true] node[left]{\(s_\True\)}
      (1,0) coordinate[vertex, fill=ks-false] node[right]{\(s_\False\)}
      (0,-1) coordinate[vertex, fill=ks-aux] node[left]{\(s_\Aux\)};
    },
    semi-or graph/.pic={
      \coordinate[vertex](i1);
      \coordinate[vertex](i2) at (0,-1);
      \coordinate[vertex](i1') at (1,0);
      \coordinate[vertex](i2') at (1,-1);
      \coordinate[vertex](t) at (2,0);
      \coordinate[vertex](a) at (0,-1/2);

      \node[vertex label, left] at (i1){\(i_1\)};
      \node[vertex label, left] at (i2){\(i_2\)};
      \node[vertex label, left] at (a){\(s_\Aux\)};
      \node[vertex label, right] at (t){\(s_\True\)};

      \draw (i2') -- (i2) -- (a) -- (i1) -- (i1') -- (t) -- (i2') -- (i1');
    },
    and-or graph/.pic={
      \coordinate[vertex](i1) at (-1,-1);
      \coordinate[vertex](ai) at (0,-3/2);
      \coordinate[vertex](i2) at (-1,-2);

      \coordinate[vertex](i1') at (1,1);
      \coordinate[vertex](i2') at (1,0);

      \coordinate[vertex](i') at (2,1);
      \coordinate[vertex](a') at (2,0);

      \coordinate[vertex](n1) at (1,-1);
      %\coordinate[vertex](an) at (1,-3/2);
      \coordinate[vertex](n2) at (1,-2);

      \coordinate[vertex](no) at (5,1);
      \coordinate[vertex](o) at (6,0);
      \coordinate[vertex](ao) at (6,1);

      \coordinate[vertex](i'') at (3,1);
      \coordinate[vertex](n1') at (2,-1);
      \coordinate[vertex](n2') at (2,-2);

      \coordinate[vertex](no') at (4,1);
      \coordinate[vertex](o1') at (4,-1);
      \coordinate[vertex](o2') at (4,-2);

      \coordinate[vertex](to) at (3,0);
      \coordinate[vertex](ti) at (3,-3/2);

      \node[vertex label, below] at (ai) {\(s_\Aux\)};
      \node[vertex label, below] at (a') {\(s_\Aux\)};

      \node[vertex label, below] at (to) {\(s_\True\)};
      \node[vertex label, below] at (ti) {\(s_\True\)};

      \node[vertex label, above] at (ao) {\(s_\Aux\)};

      \node[vertex label, above] at (i') {\(i'\)};


      \draw
      (i1) -- (ai) -- (i2) (n1) -- (ai) -- (n2)
      (i1) -- (n1) -- (n1') -- (o1') -- (o) (n1') -- (ti) -- (o1')
      (i2) -- (n2) -- (n2') -- (o2') -- (o) (n2') -- (ti) -- (o2')
      %(n1) -- (an) -- (n2)
      (o) -- (no) -- (ao) -- (o)
      (i1) -- (i1') -- (i2') -- (i') -- (i1')
      (a') -- (i') -- (i'') -- (no') -- (no) (no') -- (to) -- (i'');

      \draw[over] (i2) -- (i2');


    },
    or graph/.pic={

      \pic{and-or graph};
      \node[vertex label, left] at (i1) {\(i₁\)};
      \node[vertex label, left] at (i2) {\(i₂\)};
      \node[vertex label, below] at (n1) {\(¬i₁\)};
      \node[vertex label, below] at (n2) {\(¬i₂\)};
      \node[vertex label, right] at (o) {\(o\)};
      \node[vertex label, above] at (no) {\(¬o\)};

    },
    and graph/.pic={

      \pic{and-or graph};
      \node[vertex label, left] at (i1) {\(¬i₁\)};
      \node[vertex label, left] at (i2) {\(¬i₂\)};
      \node[vertex label, below] at (n1) {\(i₁\)};
      \node[vertex label, below] at (n2) {\(i₂\)};
      \node[vertex label, right] at (o) {\(¬o\)};
      \node[vertex label, above] at (no) {\(o\)};

    },
    not graph/.pic={
      \coordinate[vertex](i);
      \coordinate[vertex](o) at (1,0);
      \coordinate[vertex](a) at (0,-1);

      \node[vertex label, left] at (a) {\(s_\Aux\)};
      \node[vertex label, left] at (i) {\(i₁\)};
      \node[vertex label, right] at (o) {\(o\)};
      \draw (i) -- (o) -- (a) -- (i);
    },
  }

  We demonstrate constructions of graphs computing each of the boolean
  operations.

  To improve readability, we adopt the following conventions in the
  illustrations below:
  \begin{itemize}
    \item Assume implicitly the presence of special vertices
      \(s_\True,s_\False,s_\Aux\) joined by a triangle.  We omit them from the
      diagrams, using them only when needed.
    \item To avoid excessive edge crossings, we sometimes illustrate one vertex
      as multiple ``duplicate'' vertices with the same label.
  \end{itemize}

  \begin{description}

  \item[\NOT] The following graph computes the boolean \NOT{} operation:

    \begin{center}
      \begin{tikzpicture}[boolean graph]
        \pic{not graph};
      \end{tikzpicture}
    \end{center}

    Since \(i₁\) neighbors \(o\) (and both neighbor \(s_\Aux\)), they
    necessarily have opposite colors.  Below we show colorings for both
    possible input values (unique up to permutation of colors):

    \[
      \begin{array}{c|c}
        x₁=\True & x₁=\False \\ \midrule
        \begin{tikzpicture}[boolean graph]
          \pic{not graph};
          \coordinate[vertex, fill=ks-true]() at (i);
          \coordinate[vertex, fill=ks-false]() at (o);
          \coordinate[vertex, fill=ks-aux]() at (a);

          \pic at (3,0) {triangle};
        \end{tikzpicture}
        &
        \begin{tikzpicture}[boolean graph]
          \pic{not graph};
          \coordinate[vertex, fill=ks-false]() at (i);
          \coordinate[vertex, fill=ks-true]() at (o);
          \coordinate[vertex, fill=ks-aux]() at (a);

          \pic at (3,0) {triangle};
        \end{tikzpicture}
      \end{array}
    \]

  \item[\OR] Our construction of the boolean \OR{} gate is slightly
    complicated.  To that end, before giving that construction, we first
    introduce a \emph{helper} graph, which we call the ``semi-\OR'' graph.

    \begin{aside}

      \begin{center}
        \begin{tikzpicture}[boolean graph]
          \pic{semi-or graph};
        \end{tikzpicture}
      \end{center}

      Notice that this graph does not yet define a boolean graph, because it has
      no output vertex.  However, it has some useful properties resembling that
      of an \OR-gate.  Examine each of the possible input combinations:
      \begin{itemize}
        \item When \(i_1\) and \(i_2\) are both assigned \True, a proper
          coloring exists:
          \begin{center}
            \begin{tikzpicture}[boolean graph]
              \pic{semi-or graph};
              \coordinate[vertex, fill=ks-true]() at (i1);
              \coordinate[vertex, fill=ks-true]() at (i2);
              \coordinate[vertex, fill=ks-false]() at (i1');
              \coordinate[vertex, fill=ks-aux]() at (i2');
              \coordinate[vertex, fill=ks-aux]() at (a);
              \coordinate[vertex, fill=ks-true]() at (t);

              \pic at (4,0){triangle};
            \end{tikzpicture}
          \end{center}

        \item When \(i_1\) and \(i_2\) are both assigned \False, then no proper
          coloring exists:
          \begin{center}
            \begin{tikzpicture}[boolean graph]
              \pic{semi-or graph};
              \coordinate[vertex, fill=ks-false]() at (i1);
              \coordinate[vertex, fill=ks-false]() at (i2);
              %\coordinate[vertex, fill=ks-false]() at (i1');
              %\coordinate[vertex, fill=ks-aux]() at (i2');
              \coordinate[vertex, fill=ks-aux]() at (a);
              \coordinate[vertex, fill=ks-true]() at (t);
              %\node[above] at (i1') {\(i₁'\)};
              %\node[below] at (i2') {\(i₂'\)};

              \pic at (4,0) {triangle};
            \end{tikzpicture}
          \end{center}

          Each of the two uncolored vertices neighbor \(s_\True\) and an input
          vertex \(i₁\) or \(i₂\), whose color matches \(s_\False\).  Thus the
          uncolored vertices would have to be colored same as \(s_\Aux\).
          However, they neighbor each other as well, forcing them to share a
          color.  Thus there is no proper coloring.

        \item When exactly one of \(i_1,i_2\) is assigned \True{} and the other
          \False, then a proper coloring exists:
          \begin{center}
            \begin{tikzpicture}[boolean graph]
              \pic{semi-or graph};
              \coordinate[vertex, fill=ks-true]() at (i1);
              \coordinate[vertex, fill=ks-false]() at (i2);
              \coordinate[vertex, fill=ks-false]() at (i1');
              \coordinate[vertex, fill=ks-aux]() at (i2');
              \coordinate[vertex, fill=ks-aux]() at (a);
              \coordinate[vertex, fill=ks-true]() at (t);
            \end{tikzpicture}
          \end{center}

    \end{itemize}
    Together, these observations reveal that the semi-\OR{} graph does not
    \emph{compute} the \OR{} operation, but it has the property of being
    \emph{properly-3-colorable} if and only if \(i_1\) or \(i_2\) is assigned
    \True.


  \end{aside}

  Now, we are ready to construct a (``full'') \OR{} graph:

  \begin{center}
    \begin{tikzpicture}[boolean graph]
      \pic{or graph};
    \end{tikzpicture}
  \end{center}

  %\todo[inline]{i made two versions of this graph diagram---the above has less
  %  vertex duplication and is laid out more like a ``gate'', but the below
  %  graph more explicitly/clearly illustrates the parts and may be easier to
  %  explain, at the expense of many more duplicated vertices. @nick, do you
  %find one nicer than the other?}

  \begin{center}
    \begin{tikzpicture}[boolean graph]

      \coordinate[vertex](i1);
      \coordinate[vertex](i2) at (0,-1);
      \coordinate[vertex](a) at (0,-1/2);
      \coordinate[vertex](i') at (2,0);
      \coordinate[vertex](a') at (2,-1/2);
      \coordinate[vertex](no') at (2,-1);
      \coordinate[vertex](no'') at (3,-1);
      \coordinate[vertex](i'') at (3,0);
      \coordinate[vertex](to) at (4,0);

      \coordinate[vertex](in1) at (0,-2);
      \coordinate[vertex](in2) at (0,-4);

      \coordinate[vertex](on) at (0,-6);
      \coordinate[vertex](no) at (1,-6);
      \coordinate[vertex](ao) at (0,-7);


      \foreach \i in {1,2} {
        \coordinate[vertex](i\i') at ($ (i\i) + (1,0) $);
        \node[vertex label, left] at (i\i) {\(i_{\i}\)};
        \node[vertex label, left] at (in\i) {\(i_{\i}\)};

        \coordinate[vertex](n\i) at ($ (in\i) + (1,0) $);
        \coordinate[vertex](a\i) at ($ (in\i) + (0,-1) $);
        \coordinate[vertex](n\i') at ($ (n\i) + (1,0) $);
        \coordinate[vertex](an\i) at ($ (n\i) + (0,-1/2) $);
        \coordinate[vertex](o\i) at ($ (n\i) + (0,-1) $);
        \coordinate[vertex](o\i') at ($ (o\i) + (1,0) $);
        \coordinate[vertex](t\i) at ($ (n\i') + (1,0) $);

        \node[vertex label, above] at (n\i) {\(¬i_{\i}\)};
        \node[vertex label, left] at (o\i) {\(o\)};
        \node[vertex label, right] at (an\i) {\(s_\Aux\)};
        \node[vertex label, right] at (t\i) {\(s_\True\)};

        \draw
          (in\i) -- (n\i) -- (a\i) -- (in\i)
          (n\i) -- (an\i) -- (o\i) -- (o\i') -- (n\i') -- (n\i)
          (n\i') -- (t\i) -- (o\i');
      }

      \foreach \v in {a,a1,a2,ao} {
        \node[vertex label, left] at (\v) {\(s_\Aux\)};
      }
      \node[vertex label, right] at (a') {\(s_\Aux\)};

      \node[vertex label, above] at (i') {\(i'\)};
      \node[vertex label, left] at (on) {\(o\)};
      \node[vertex label, left] at (no') {\(¬o\)};
      \node[vertex label, right] at (no) {\(¬o\)};

      \node[vertex label, right] at (to) {\(s_\True\)};

      \draw
        (i1) -- (a) -- (i2) -- (i2') -- (i1') -- (i1)
        (i1') -- (i') -- (i2')
        (i') -- (a') -- (no') -- (no'') -- (i'') -- (i')
        (i'') -- (to) -- (no'')
        (on) -- (no) -- (ao) -- (on);

    \end{tikzpicture}
  \end{center}

  To see why this construction works,

  %\todo{still not done}




\item[\AND] To construct an \AND{} gate, we apply DeMorgan's law to rewrite
  \AND{} in terms of \NOT{} and \OR:
  \[
    y=x₁∧x₂ ⟺ ¬y=¬x₁∨¬x₂.
  \]

  To that end, we implement an \AND{} graph by negating both the input
  vertices and the output vertex in the \OR{} graph.  We do so by swapping
  vertices
  \[
    i₁↔¬i₁, \qquad i₂↔¬i₂, \qquad o↔¬o
  \]
  in the \OR{} graph construction:

  \begin{center}
    \begin{tikzpicture}[boolean graph]
      \pic{and graph};
    \end{tikzpicture}
  \end{center}


\end{description}

\end{proof}

\begin{theorem}{}{circuit-to-graph}%
  For any boolean circuit \(C\), there exists a boolean graph that computes
  [the function defined by] \(C\).  Moreover, there exists a polynomial-time
  algorithm that performs this conversion from boolean circuits to graphs.
\end{theorem}

\begin{proof}
  We describe an algorithm that, given a circuit \(C\), generates a boolean
  graph computing \(C\).

  First, construct the special triangle \(s_\True,s_\False,s_\Aux\).

  For each wire in \(C\) (including the input and output wires), create a
  corresponding vertex and join it by an edge to \(s_\Aux\). (The color of this
  vertex will correspond to the value carried by the wire.)

  Next, for each [\NOT, \OR, and \AND] gate \(g\) in \(C\), apply
  \cref{lem:boolean-operation-graphs} to construct a subgraph \(γ\) computing
  \(g\).  The input vertices of \(γ\) should exactly correspond to the input
  wires of \(g\), and the output vertex of \(γ\) should correspond to the
  output wire of \(g\).

  %\todo{not done, need some induction to make precise}

\end{proof}

\begin{example}{}{}
  Conversion of an example circuit (e.g., XOR gate?  or something simpler, like
  \(x₁∧¬x₂\)) to graph.

  TODO
\end{example}

\subsection{Graph-colorability}

\begin{definition}{The graph 3-colorability problem (\Problem{3col})}{}

  The \Term{graph 3-colorability problem} is stated as the following yes/no
  question: given a graph, is it (properly) 3-colorable?

\end{definition}

\begin{theorem}{}{}
  \Problem{3col} is \NP-complete.
\end{theorem}

\begin{proof}
  It is straightforward to see that \(\Problem{3col}∈\NP\), since it is
  solvable by polynomial-time guess-and-check: guess a color assignment for
  each vertex, then verify that, for each edge \(e\), the two vertices joined
  by \(e\) have distinct colors.  The ``check'' procedure takes
  \(\O(\Abs{\Edges[G]})\) time, which is polynomial with respect to the size of
  the graph.

  To show that \(\Problem{3col}\) is \NP-hard, we show
  \(\CircSat≤\Problem{3col}\) by reducing \CircSat{} to \Problem{3col}.
  Specifically, given a circuit \(C\) with inputs \(x₁,\dotsc,xₙ\), we wish to
  construct a corresponding graph \(G\) such that \(G\) is 3-colorable if and
  only if \(C\) is satisfiable.

  Let such a circuit \(C\) be given.  Apply the algorithm from
  \cref{th:circuit-to-graph} to produce a graph \(G\) that computes the boolean
  function \(C\), such that the inputs \(x₁,\dotsc,xₙ\) of \(C\) correspond
  respectively to input vertices \(i₁,\dotsc,iₙ\), and the output of \(C\)
  corresponds to the output vertex \(o\).

  Force the output vertex of \(G\) to take on the boolean value \(\True\) by
  ``merging'' vertices \(o\) and \(s_\True\) into one vertex, keeping all of
  their connections to other vertices.  (Equivalently, if we wish to avoid such
  a ``merging'' operation, we may construct \(G\) assuming \emph{a priori} that
  \(o=s_\True\).  Yet another alternative is to create edges joining \(o\) with
  \(s_\False\) and \(s_\Aux\), thereby forcing it to share a color with
  \(s_\True\).)

  We claim that this resulting graph \(G\) is 3-colorable if and only if \(C\)
  is satisfiable.
  \begin{itemize}
    \item[(\(⟹\))] Suppose that \(G\) is 3-colorable.  Then let \(κ\) be a
      proper 3-coloring of \(G\).  Let \(x₁,\dotsc,xₙ\) be the respective
      boolean values assigned by \(κ\) to the input vertices \(i₁,\dotsc,iₙ\)
      of \(G\). Since \(G\) computes \(C\)
      (\cref{def:boolean-graph-functions}), we know that
      \(κ(o)=C(x₁,\dotsc,xₙ)\).  At the same time, we have also ensured by
      construction that \(κ(o)=\True\).  Thus \(x₁,\dotsc,xₙ\) is a satisfying
      assignment for \(C\).  Thus \(C\) is satisfiable.

    \item[(\(⟸\))] Suppose that \(C\) is satisfiable.  Then let
      \(x₁,\dotsc,xₙ\) be a satisfying assignment for \(C\).  We construct a
      3-coloring for \(G\) as follows:
      \begin{itemize}
        \item Arbitrarily color the special vertices
          \(s_\True,s_\False,s_\Aux\).  Call those colors \(\True,\False,\Aux\)
          respectively.
        \item For each input TODO unfinished
      \end{itemize}


  \end{itemize}


\end{proof}



\subsection{Graph-coloring games}



\section{Set covering games}

\chapter{Conclusion}

future work: general framework to convert NP-complete problems to
sigmap-complete games.  basic idea--there is a consistency/properness condition
enforced at each step of the players' moves, and a completion condition (all
things have been filled in).  if the consistency condition is violated at any
step, the violating player loses.  otherwise, if the completion condition is
attained, the last player wins.  finally, map two conditions onto circuits.

other future work: explore more games/problems


%\chapter{Introduction}

The basic question of computational complexity---``how hard is this problem for
a computer to solve?''---is central to nearly every topic in computer science.
And yet the formalisms of complexity theory often seem, in my own experience,
intimidatingly abstract, phrased in terms of intangible models of computation
such as non-deterministic Turing machines and oracles.

The remedy, I believe, lies in studying complexity theory through the lens of
\emph{puzzles} and \emph{games}.  Not only do they provide a concrete grounding
for the abstractions, they also offer a particularly insightful, accessible,
and most importantly fun approach to understanding complexity theory.  In fact,
many of the most popularly known and appreciated results in complexity theory
are those about so-called ``\NP-complete puzzles'', such as Sudoku, and
``\PSPACE-complete games'', such as Checkers and Go.

This thesis emphasizes that approach in its exploration of a particularly
foundational, yet often overlooked, ladder of complexity classes known as the
\emph{polynomial hierarchy}.  \NP{} is the class of (one-player) ``puzzles'',
and \PSPACE{} is the class of (two-player) ``games'' of polynomial length; the
polynomial hierarchy, then, lies in the middle, encompassing games of
\emph{fixed} length.  Through this lens, the (in)famous \P-vs-\NP{} question is
but the first in a ladder of questions that are, arguably, just as crucial and
impactful.

%The polynomial hierarchy is as central to
%complexity theory as the \P-vs-\NP{} problem is well-known.

\section{Overview}

This document is structured as follows.  First, \cref{ch:background,ch:boolean}
establish preliminary background concepts and conventions adopted throughout
this thesis.  Next, \cref{ch:circuit} lays the central theoretical groundwork,
defining the \emph{polynomial hierarchy} through a fundamental family of
problems known as the \Problem{Circuit Satisfiability} games.  Next,
\cref{ch:misc} explores a novel family of games generalized from the
\Problem{Graph 3-Colorability} puzzle and establishes \emph{hardness} bounds on
each of those games.  Finally, \cref{ch:conclusion} concludes by discussing the
future directions of this work and its broader implications.

\section{Prior work and inspirations}

Much of the background exposition on complexity theory referenced in this thesis
is reproduced from Christos Papadimitriou's textbook,
\citet{papadimitriou.cc} (though many of the foundational ideas were
originally introduced/proven elsewhere, e.g.
\citet{cook.np,levin.np,stockmeyer.ph}), reframed through the
puzzles-and-games perspective and supplemented with a few comments on intuition.

The main family of games explored in this thesis, fixed-turn
\Problem{3-Colorability} games (\cref{ch:misc}), is a generalization of
(one-turn) \Problem{3-Colorability}, a well-known \NP-complete puzzle originally
proven \NP-complete by \citet{karp.np}.  Others have studied (multi-turn)
game generalizations of \Problem{3-Colorability}, but all versions that I've
encountered are \PSPACE-complete, in which the number of turns played during the
game scales proportionally with the size of the graph
\citep{bodlaender.coloring,bh.placement,kbd.impartial,cpss.coloring,schaefer.games}. As far as I'm aware, the variations I explore here—with fixed
numbers of turns regardless of the size of the graph—is unexplored, and the main
theorem about its \SigmaP k-completeness (\cref{th:yayay}) is novel.  The basic
idea underlying my proof is the composition of two well-known results:
\begin{itemize}[nosep]
  \item \citet{karp.np}'s classic proof of the \NP-hardness of the \Problem{3-Colorability} puzzle, via a reduction from \Problem{3CNF-Satisfiability};
  \item 's transformation from boolean circuits to equivalent 3CNF-clauses.
\end{itemize}

Without further ado, let's begin.


%TODO: outline/overview of chapters, after those chapters are written

%TODO: also give general citations here, e.g. papadimitriou for many
%foundational background info, etc.
%
%TODO: notation table also belongs in this chapter i think

%p-vs-np well known, polynomial-hierarchy central

%puzzles and games; hierarchy lies in the interstices.  we examine a few
%interesting (by no means exhaustive, or even close to comprehensive)
%np-complete puzzles with pspace-complete analogues, and we









%Famously central to the theory of computational complexity is the \P-vs-\NP{}
%question, and essential to our understanding of that question is the study of
%\NP-complete problems such as the Boolean Satisfiability puzzle, the Graph
%Colorability puzzle, and countless more.  Puzzles like these, which nearly any
%layperson can appreciate, offer a particularly insightful, intuitive, and
%\emph{fun} lens through which to study computational complexity. Explorations
%of more complex problem-classes such as \PSPACE{} can be similarly approached
%through the study of strategic decision \emph{games} such as Othello, Checkers,
%and Go.
%
%What lies in the interstices between \emph{puzzles} and \emph{games}?  How do
%we take a puzzle and generalize it into a game, and what are the puzzle-games
%we encounter along the way?  And how hard exactly are these puzzle-games to
%decide?  These questions are the focus of my thesis.

%So far, I have explored these questions from three angles:
%\begin{enumerate}
%
%  \item \label{itm:intro.q.generation} Puzzle generation.  If I wish to solve a
%    puzzle, you can play a game with me by constructing puzzle \emph{instances}
%    for me to solve.  For instance, \emph{solving} Sudoku is an \NP-complete
%    problem; your task is to \emph{generate} (partially-filled) Sudoku boards
%    for me to solve.
%
%    How hard is it to do so?  Moreover, how hard is it to generate \emph{good}
%    puzzle instances, for various definitions of \emph{good} (sufficiently
%    challenging to solve, or having unique solutions, or solvable/unsolvable by
%    certain strategies)?
%
%    % lauren sanchis
%
%  \item \label{itm:intro.q.pspace} \PSPACE-complete games derived from
%    \NP-complete puzzles.  A canonical \NP-complete puzzle is the \Problem{sat}
%    (Boolean Satisfiability) puzzle: given a Boolean formula \(\phi(x_1, \dots,
%    x_n)\), does there exist an assignment to its inputs \(x_1, \dots, x_n\)
%    such that \(\phi(\dots) = 1\)? In an analogous game, two players alternate
%    turns assigning \(x_1, \dots, x_n\); player 1 wins if \(\phi(\dots)=1\),
%    and player 2 wins if \(\phi(\dots)=0\).  Does either player have a
%    (guaranteed) winning strategy?  This game, known as \Problem{qsat}
%    (Quantified Satisfiability), is a canonical example of a \PSPACE-complete
%    game.
%
%    Can other \NP-complete puzzles be similarly generalized into games?  Will
%    those games also be \PSPACE-complete?
%
%    % schaefer
%
%  \item \label{itm:intro.q.ph} Fixed-turn games and the polynomial hierarchy.
%    In between the complexity classes \NP{} and \PSPACE{} lies a chain of
%    increasingly-complex problem-classes known as the \emph{polynomial
%    hierarchy}.  In some cases, problems in the polynomial hierarchy may be
%    thought of as game generalizations of \NP-complete puzzles with a
%    \emph{fixed} number of turns.  For instance, in a two-turn version of
%    \Problem{sat}, inputs are partitioned into two (disjoint) groups \(X_1\)
%    and \(X_2\); on turn 1, player 1 assigns \(X_1\), and on turn 2, player 2
%    assigns \(X_2\).  As before, player 1 (respectively 2) wins if
%    \(\phi(\dots) = 1\) (respectively \(0\)).  Determining whether player 1 has
%    a winning strategy is complete for a complexity class known as
%    \(\SigmaP2\), which lies just above \NP{} in the hierarchy, and analogous
%    games with \(k\) turns are \(\SigmaP k\)-complete.
%
%    Do polynomial-hierarchy generalizations of other \NP-complete puzzles
%    exist?
%
%\end{enumerate}
%
%\Cref{ch:progress} discusses my progress so far in each of these areas.
%Questions \ref{itm:intro.q.generation} and \ref{itm:intro.q.pspace} have been
%explored in-depth by others, while question \ref{itm:intro.q.ph} appears to be
%scarcely explored.  As such, I provide only brief summaries of/reflections on
%the existing work pertaining to \ref{itm:intro.q.generation} and
%\ref{itm:intro.q.pspace}.  Meanwhile, I describe in greater detail question
%\ref{itm:intro.q.ph}, which is the focus of my explorations so far.
%
%\Cref{ch:future} summarizes the primary questions \& goals that will guide
%my exploration next semester.
%
%Finally, \cref{ch:bib} contains an annotated bibliography of existing work
%pertaining to each of these topics.



%\chapter{Current progress}

\label{ch:progress}

\section{\NP{} as puzzles, or one-move games}

Recall that \NP{} is the class of problems solvable by guess-and-check, with a
\emph{check} problem in \P{} (\cref{defn:np}):
\[
  \NP = \SetBuilder{L}{
    \exists \underbrace{\mathstrut L' \in \P}_{\mathclap{\text{the ``check'' problem}}} \;
    \forall x \quad
    x \in L \iff \underbrace{\exists g \; (x, g) \in L'}_{\mathclap{\text{guess-and-check}}}
  }.
\]
(In the above, it is \emph{implicitly} required that \(\Abs g\) be
polynomially-bounded with respect to \(\Abs x\), but we have omitted it in
notation for readability.)

Another famous example of a problem in \NP{} is \Problem{sudoku}, framed as the
following decision problem:
\begin{definition}[\Problem{sudoku}]%
  We are given a square grid with dimensions \(n^2\times n^2\), some of whose
  cells are filled in with numbers in \(\{1,\dotsc,n^2\}\).  The grid is
  evenly partitioned into \(n\) chunks along each axis, resulting in \(n^2\)
  \emph{blocks} each with dimensions \(n\times n\).

  Does there exist a way to fill in the rest of the cells so that each row,
  column, and block on the grid contains each number in \(\{1,\dotsc,n^2\}\)
  exactly once?
\end{definition}

\todo[inline]{TODO add illustration of Sudoku board}

For this problem, a ``guess'' \(g\) consists of a list of numbers in
\(\{1,\dotsc,n^2\}\) specifying the values with which to fill in the empty
cells in the given grid.  The ``check'' problem, then, is stated as follows:
\begin{nested}
  Given a fully-filled-in Sudoku board, does each row, column, and block on the
  grid contain each of \(\{1,\dotsc,n^2\}\) exactly once?
\end{nested}




%Also recall an example of an \NP{} problem, \Problem{hamiltonian-path}
%(\cref{def:hamiltonian-path}), which asks: given a graph, does it have a
%Hamiltonian path?  Here, the ``check'' problem \(L'\) can be stated as follows:
%\begin{nested}
%  Given a graph \(x = \Gamma\) with vertices \(v_1,\dotsc,v_n\), along with a
%  permutation \(g = \phi(1),\dotsc,\phi(n)\), does the sequence
%  \(v_{\phi(1)},\dotsc,v_{\phi(n)}\) specify a valid path on \(\Gamma\)?
%\end{nested}
%
%We can now intuitively reframe \Problem{hamiltonian-path} as a one-player
%``game'', played on an input ``board'' in the form

%in which the player writes down some
%permutation \(\phi\).  They ``win'' if it meets the validity condition \((x, g)
%\in L'\) and ``lose'' if it doesn't.  Under this framing, the decision problem
%becomes the following question: does the player have a winning \emph{strategy}?

\section{Multi-turn games and the polynomial hierarchy}

Consider, now, a two player, a ``solver'' and an
``adversary'', in two turns:
\begin{itemize}
  \item First, the solver
  \item Next, the adversary
\end{itemize}

This ``guess-and-check'' extension of \P{} may be continued

to define higher
and higher complexity classes, known as the \emph{polynomial hierarchy}:
\begin{definition}[polynomial hierarchy]
  \begin{align*}
    \SigmaP1 &= \NP = \SetBuilder{L}{\exists L' \in \P \; \forall x \quad
      x \in L \iff \exists g \; (x, g) \in L'
    }, \\
    \SigmaP2 &= \NP = \SetBuilder{L}{\exists L' \in \P \; \forall x \quad
      x \in L \iff \exists g \; (x, g) \in L'
    }, \\
  \end{align*}
\end{definition}




%In general, any complexity class has a guess-and-check equivalent, called its
%\emph{projection}:
%\begin{definition}[projection]
%  Let \(\C\) be a complexity class.  Its \emph{projection} is the class of
%  problems
%  \[
%    \pro\C = \SetBuilder*{L}{
%      \exists L' \in \C, \text{polynomial \(p\)}; \; \forall x \quad
%      x \in L \iff
%      \exists g \; \text{\(\Abs g \le p(\Abs x)\) and \((x, g) \in L'\)}
%    }.
%  \]
%
%  (Observe, then, that \(\NP = \pro\P\).)
%\end{definition}








\section{Boolean Satisfiability}

\subsection{The Satisfiability puzzle}

\todo{context on what booleans are?}

Our puzzles-and-games characterization of the polynomial hierarchy begins with
a well-known family of problems generally referred to as Boolean Satisfiability
problems.  Here is perhaps the simplest, most well-known Satisfiability puzzle:

\begin{definition}[\Problem{sat}]%
  Given a Boolean formula \(\phi(x_1, \dots, x_n)\), does there exist an
  assignment of Boolean values to inputs \(x_1, \dots, x_n\) such that
  \(\phi(x_1, \dots, x_n) = 1\)?  \Problem{sat} consists of the formula
  instances for which the answer is \emph{yes}.

  Formally:
  \[
    \Problem{sat} = \SetBuilder \phi {
      \exists (x_1, \dots, x_n) \in \Set{0,1}^n \quad \phi(x_1, \dots, x_n) = 1
    }.
  \]
\end{definition}

The \SAT{} puzzle is particularly useful and worth studying because of its
generality.  Booleans form the foundation of mathematical logic: every logical
statement can be encoded, in some manner, as a Boolean formula.  Consequently,
\SAT{} is, on an intuitive level, the most general possible puzzle---given any
other puzzle, encoding its rules in terms of Booleans reveals that it is merely
a special case of \SAT.  This idea is expressed formally as the Cook-Levin
theorem:

\begin{theorem}[Cook-Levin]
  \SAT{} is \NP-complete.
\end{theorem}

%\begin{proof}
%  \todo[inline]{put proof.  the most important reason to have the proof here is
%  to illustrate}
%\end{proof}

\subsection{Satisfiability games}

\begin{definition}[The two-turn \SAT{} games]%
  The two-turn \SAT{} game is played on a Boolean formula \(\phi(x_1, \dots,
  x_n, y_1, \dots, y_n)\) with inputs partitioned into two groups \(X =
  \Set{x_i}\) and \(Y = \Set{y_i}\).  The two turns proceed as follows:
  \begin{enumerate}
    \item Player 1 assigns values to \(X\).
    \item Player 2 assigns values to \(Y\).
  \end{enumerate}
  Player 1 wins if \(\phi\) is satisfied (\(\phi(\dots) = 1\)), and player 2
  wins if \(\phi\) is falsified.

  Who wins?  Two decision problems arise from this game:
  \begin{itemize}
    \item Does player 1 have a winning strategy?  That is, can player 1 make
      some first move so that no matter what player 2 does, player 1 always
      wins?

    \item Does player 2 have a winning strategy?  That is, no matter what
      player 1 plays, can player 2 respond with some move guaranteeing a win?
  \end{itemize}

\end{definition}


\section{Graph coloring}

\begin{definition}[\Problem{3col}]%
  Given a graph \(\Gamma\), is there a way to color each vertex in  \(\Gamma\)
  with one of three colors so that every pair of adjacent vertices has distinct
  colors?
\end{definition}

\begin{theorem}
  \Problem{3col} is \NP-complete.
\end{theorem}

\subsection{Graph coloring games}

\begin{definition}[Two-turn \Problem{3col}]%
  The two-turn \Problem{3col} game is played on a graph \(\Gamma\) whose
  vertices are partitioned into two (disjoint) groups \(X\) and \(Y\).  Two
  players take turns assigning one of three colors to vertices.  First, player
  1 colors vertices in \(X\); second, player 2 colors  vertices in \(Y\).
  Player 1 wins if the resulting coloring is \emph{invalid}---that is, there
  exists a pair of vertices sharing the same color; player 2 wins if the
  resulting coloring is valid.

\end{definition}

\begin{conjecture}
  Two-turn \Problem{3col} is complete for \SigmaP2 (or \PiP2, depending on
  which player's winning strategy we examine).
\end{conjecture}










%\section{Puzzle generation}
%
%\label{sec:progress.generation}
%
%The topic of puzzle generation difficulty has been explored in detail by Laura
%Sanchis
%\parencite{language-instances,test-gen-complexity,hard-diverse-graph-tests}.
%
%\todo[inline]{incomplete; i'm focusing my effort on the fixed-turn section first
%because that's more novel/interesting}
%
%\subsection{Puzzles with unique solutions}
%
%In many popularly-known puzzle games, one criterion for ``good'' puzzle
%generation is that the generated puzzle instance should have a unique solution.
%For example, given a \(9\times9\) Sudoku grid (partially pre-filled with
%numbers \(1,\dotsc,9\)), there should be \emph{exactly one} way to complete the
%grid---no more, no less.
%
%This formulation is incompatible with our
%
%\section{\PSPACE-complete games}
%
%\label{sec:progress.pspace}
%
%\todo[inline]{incomplete}
%
%\section{Fixed-turn games}
%
%\label{sec:progress.ph}
%
%

%\chapter{Future work}

\label{ch:future}

Since I did most thinking about the polynomial hierarchy question, most of the
``future work'' (basically, next semester) will be about that.  Concrete open
questions and proof TODOs go here, puzzles I want to investigate immediately
following graph-coloring, etc.  As time permits, the two other areas.

\section{Goals?  Timelines?}

For clinic, we had to propose timelines for deliverables/assessments of
``success''.  I don't know if the same sort of thing applies to thesis, but
that's why this section exists.



%\chapter{Annotated bibliography}
\label{ch:bib}

\section{Puzzle generation}

\todo{fix formatting/combine with automatically-generated bibliography}

\todo{this is basically my full annotated bibliography from before on the topic
  of puzzle generation.  the focus of my topic has shifted somewhat since then
  away from generation and towards \PH{} and general games; is it still worth
keeping all these sources here in detail?}

\begin{itemize}

  \item \fullcite{language-instances}

    \begin{annotation}
      This paper introduces the simplest formal model for efficient puzzle
      generators.  A \emph{polynomial time constructor} (PTC) for a language
      \(L\) is a deterministic program that, on input \(1^n\), runs in
      polynomial-time and returns a string in \(L\) with length \(n\) iff one
      exists.  \emph{Polynomial time generators} (PTGs) is the nondeterministic
      analog of PTCs, with the additional requirement that every string in
      \(L\) of length \(n\) must be reachable.  PTCs and PTGs could be thought
      of as programs that produce solvable puzzles of given sizes (e.g., given
      \(n\), generate a solvable \(n^2 \times n^2\) Sudoku board).

      The main question explored here is: which classes of languages
      (puzzles/problems) have PTCs and PTGs?  Relevant results include:
      \begin{itemize}
        \item Every language that has a PTG is in \NP.
        \item For any language \(L\) in \NP, \(L\) has a PTC iff it has a PTG.
        \item Every \P{} language has a PTG iff every \NP{} language has a PTG.
      \end{itemize}
      Surprisingly enough, that last result indicates that we don't know
      whether every \P{} language has a PTG.  This paper goes on to define
      various special types of PTGs (e.g., categorical, lexicographical, etc.)
      and establishes various connections between PTG-existence questions and
      polynomial-hierarchy relations.
    \end{annotation}

  \item \fullcite{test-gen-complexity}

    \begin{annotation}
      In this paper, Sanchis generalizes the notion of puzzle generators
      introduced in \textcite{language-instances}.  Define a \emph{Test
      Instance Construction Method} (TICM) with respect to some fixed problem
      \(\Pi\) as a non-deterministic, polynomial-time program that, given a
      \emph{desired} answer \(\alpha\) (along with some desired parameters on
      the input, e.g., length), attempts to return an instance/input of \(\Pi\)
      that has answer \(\alpha\) and which meets the target parameters.

      The key result from this paper is that, unless \(\NP = \co\NP\), most
      \NP-hard problems do \emph{not} have efficient TICMs that can generate
      all input instances (with given known answers).  This establishes
      theoretical upper bounds on how comprehensive we can reasonably expect a
      puzzle generator to be in its coverage of available/possible inputs.
    \end{annotation}

  \item \fullcite{hard-diverse-graph-tests}

    \begin{annotation}
      In this paper, Sanchis continues to tackle the question, which problems
      have efficient TICMs?  This time, she further relaxes the desired TICM
      criterion---instead of looking for TICMs that can generate \emph{all}
      matching input instances, simply look for TICMs that generate a
      representative, or \emph{diverse}, set of inputs.  For a given problem
      optimization problem \(P\) and with respect to parameters (computable
      functions on \((\text{input}, \text{solution})\) pairs) \(l_1, \dots,
      l_k\), a set \(S\) of \((\text{input}, \text{solution})\) pairs is
      \emph{diverse} if every optimal pair in \(P\) has a corresponding pair in
      \(S\) with the same parameter values.

      The parameters \(l_1, \dotsc, l_k\) capture the way in which we can
      control properties of the generated puzzle instance.  Taking Sudoku as an
      example, \(l_i\) may be used to control the size of the board, the number
      of pre-filled squares, etc.  As another example, graph-based problems
      (e.g., traveling salesman, Hamiltonian path) may set \(l_i\) to be the
      number of vertices, edges, weights, etc. in the input graph.  In this
      sense, a diverse set contains representative puzzles for each
      (attainable) property combination; a diverse \emph{generator} is one
      capable of \emph{producing} puzzles for each such combination.

      This paper takes the TICM question in a concrete and exciting direction
      by \emph{constructing} efficient, hard (technical definition that roughly
      means ``doesn't only output trivially-solvable puzzles''), and diverse
      puzzle generators for three graph problems: minimum vertex cover,
      domination number, and chromatic number.
    \end{annotation}

  \item \fullcite{maximum-clique-generators}

    \begin{annotation}
      This paper, like \textcite{hard-diverse-graph-tests}, offers another
      deep-dive into a specific \NP-hard problem---the maximum clique problem
      (given an arbitrary graph, what is its largest complete subgraph?).  This
      paper describes several approximation algorithms for the maximum clique
      problem, as well as a ``hard and diverse'' algorithm generating test
      cases for any given number of vertices, edges, and maximum clique size.
    \end{annotation}

  \item \fullcite{stable-marriage-generation}

    \begin{annotation}
      This paper explores puzzle generation for the Stable Matching (SM)
      problem allowing for ties and incomplete preference lists.  The SM
      problem is stated as follows: given \(n\) people and \(n\) pets, and
      given a strictly-ordered preference list for each person and each pet,
      pair up people with pets so that no person and no pets simultaneously
      prefer each other to their currently-assigned partners; such a matching
      is called \emph{stable}.  The SM problem is solvable in quadratic time;
      the SMT variation, in which preference orderings are non-strict (i.e.,
      allowing ties), as well as the SMI variation, in which preferences lists
      may be incomplete (unspecified preferences are unacceptable; matchings
      may be partial), are also both solvable within quadratic time.  However,
      the \emph{SMTI} variation, in which both ties and incomplete lists are
      allowed, is NP-complete.

      This paper explores several proposed methods for generating SMTI puzzles,
      evaluating them by the criterion laid out in
      \textcite{test-gen-complexity}, observes theoretical shortcomings of each
      approach, and discusses connections between limitations in these
      algorithms to relationships between complexity-classes, e.g. \(\NP = \co\NP\).
    \end{annotation}

  \item \fullcite{random-latin-squares-sudoku}

    \begin{annotation}
      This paper describes relatively simple algorithms that randomly generate
      Sudoku boards and Latin Squares (\(n \times n\) grids in which each row
      and column contains each number \(1, \dotsc, n\)) with supposedly uniform
      probability.  There is not much theoretical discussion on complexity
      topics, but this paper discusses connections/reductions between
      Sudoku/Latin-Squares and the maximum clique problem, which in turn is
      examined at length from both theoretical and experimental perspectives in
      \textcite{maximum-clique-generators}.
    \end{annotation}

  \item \fullcite{strategy-solvable-sudoku}

    \begin{annotation}
      Computers generally solve Sudoku by reducing to \Problem{sat} or other
      well-optimized \NP-complete problem solvers, brute-force/backtracking,
      etc.  These approaches are quite different than the approaches taken by
      humans, which tend to entail various ``reduction'' rules, or
      \emph{strategies}.  An example of such a strategy is the \emph{naked
      singles} rule: for each given cell, start by writing out all the possible
      values (\(1, \dots, 9\)); eliminate from these possibilities values
      already taken by other cells in the same row/column/block; when only one
      possibility remains, finalize that cell with that value.

      While strategy-based approaches better represent how humans solve puzzles
      and are more intuitive, they lack the generality that \NP-complete puzzles
      usually require.  In particular, not all Sudoku puzzles are
      strategy-solvable.  The empty starting board, for instance, is not
      strategy-solvable because \emph{strategic} approaches are predicated on
      the assumption that the final solution is unique---which is not the case
      for an empty starting board.

      This paper details an algorithm (along with a framework for generalizing)
      for generating \emph{strategy-solvable} Sudoku puzzles.  The focus of
      this paper is applied, rather than pure---there is minimal emphasis on
      theoretical complexity or absolute coverage/generalizability.  Instead,
      the emphasis here is on generating puzzles that are solvable by these
      ``human-oriented'' strategies.  The algorithm proposed performs well on
      most grids but hits bottlenecks as the number of starting clues
      (pre-filled cells) shrinks to around \(20\).
    \end{annotation}

  \item \fullcite{conp-approximation}

    \begin{annotation}
      An set \(S\) of inputs \emph{approximates} a particular problem/language
      \(L\) if the \(S \subseteq L\).  Naturally, for a given approximation,
      another one is \emph{better} if it can solve infinitely many more
      problems in \(L\) (being able to solve finitely many more problems is
      does not constitute better, because any algorithm can always be finitely
      ``improved'' by hard-coding in additional solutions).

      \textcite{language-instances} and \textcite{test-gen-complexity}
      established several connections between puzzle generation and complexity
      theory---in particular, the relevant result: if a puzzle has a
      polynomial-time generator, it is in \NP.  As such, if all \co\NP{}
      problems have polynomial-time generators (PTGs), then \(\co\NP = \NP\).

      Previous work leverages that connection the other way: instead of trying
      to find perfect \(\co\NP\) PTGs (a doubtful pursuit, since it implies
      \(\co\NP = \NP\)), we can search for \emph{approximate} PTGs---those
      that, for a given problem \(L \in \co\NP\), generate not necessarily all
      strings in \(L\) but nevertheless large (ideally maximal) subsets of
      \(L\).  By doing so, the corresponding \NP{} algorithms \emph{induced by}
      those approximate PTGs are, themselves, approximation algorithms for
      \(L\).

      This paper focuses on the question of \emph{optimality} for such
      approximations and proves, assuming \(\NP \ne \co\NP\), a general
      condition/test for the optimality of an \(\NP\)-complete approximaton for
      \(\co\NP\) languages.
    \end{annotation}

  \item \fullcite{unique-sudoku-poly}

    \begin{annotation}
      To guarantee \emph{uniqueness} of a Sudoku solution, many generation
      algorithms take a trial-and-error approach---e.g., generate a puzzle,
      attempt to find multiple solutions, pre-fill additional cells as
      necessary to narrow the possibilities.  While this approach is more
      generalizable, it is not very performant (solving must be attempted after
      each revision to the board).

      This paper takes an alternate approach, summarized as follows: start with
      a fully-finished Sudoku board, then cleverly apply various reduction
      rules \emph{in reverse} (i.e., deleting the value in a cell when the
      surroundings fit a certain condition) in a manner that ensures uniqueness
      is preserved.

      Like \textcite{strategy-solvable-sudoku}, this paper does not delve into
      theoretical complexity topics and is geared towards more applied,
      ``ergonomics''-focused contexts.
    \end{annotation}

  \item \fullcite{difficulty-driven-sudoku}

    \begin{annotation}
      This paper, authored under the advice of our very own Prof. Dodds, takes
      a strategy-based approach to generating puzzles, its intentions
      reminiscent of \textcite{strategy-solvable-sudoku}.  The
      ``difficulty-driven'' generation algorithm proposed by this paper is,
      however, different: it is more similar to traditional trial-and-error
      (generate, try to solve, add/remove clues) Sudoku generation methods,
      with the key difference being that the verify/check solution step is done
      via strategies (as opposed to backtracking/brute-force) to mirror human
      solving capabilities.  In addition, this paper introduces notions of
      \emph{difficulty metrics} for Sudoku (e.g., ``most difficult required
      technique'') that may be used in the generation process to control/tune
      the difficulty-level of the resulting Sudoku board.
    \end{annotation}

  \item \fullcite{sudoku-difficulty-oracle}

    \begin{annotation}
      This paper, authored under the advice of our \emph{very} very own Prof.
      Jakes, also examines the problem of generating Sudoku boards with
      controlled difficulty.  Similarly to \textcite{difficulty-driven-sudoku},
      this paper attempts to model the difficulty of a Sudoku board in terms of
      the \emph{strategies} it uses.  More interestingly, it describes a more
      generalizable difficulty metric that abstracts specific choices of
      strategies into an \emph{oracle}, assessing difficulty in relation to
      number of oracle invocations, amount of work done by each oracle
      invocation, etc.
    \end{annotation}

  \item \fullcite{steiner-graph-generator}

    \begin{annotation}
      The Steiner tree problem in graphs is stated as follows: given a
      (non-negatively) weighted, undirected graph \(G\) and a subset \(V\) of
      its vertices, find a minimum-weight subtree of \(G\) connecting all of
      \(V\). (When \(\Abs V = 2\), this is a shortest-path problem; when \(V\)
      contains all vertices of \(G\), this is the minimum-spanning-tree
      problem. Surprisingly enough, while both those problems are
      polynomial-time-solvable, this problem is \NP-hard.)

      This paper proposes a (purportedly linear-time) algorithm to generate
      test cases for the Steiner tree problem in graphs.  The key technique
      involved in this algorithm is applying the so-called
      ``Karush-Kuhn-Tucker'' optimality conditions (which appear to be a sort
      of generalization of the Lagrange-multipliers method for solving an
      optimization-with-constraints problem).

      This paper does not contain much discussion on theoretical complexity
      topics but is instead valuable as a ``deep dive'' into a particular
      \NP-hard problem (and, perhaps, the broader category of
      combinatorial-optimization problems).
    \end{annotation}

  \item \fullcite{sudoku-education}

    \begin{annotation}
      This paper examines the problem of Sudoku puzzle generation in
      \emph{educational} contexts---in particular, not just generation
      algorithms but also exercises, classroom activities, etc. for students to
      do to explore the Sudoku generation problem.  Additionally, the
      mathematical concepts in this paper are meant to be approachable to
      high-schoolers (or even younger).  Given its educational/elementary
      focus, I don't expect to gather very deep/technical results from this
      paper, but I included it anyway because it's nevertheless adjacent to my
      topic, and I personally have tremendous interest in exploring innovative,
      inquiry-based mathematics teaching methods.
    \end{annotation}

\end{itemize}

\section{\PSPACE-complete games}

\todo[inline]{incomplete}

\begin{itemize}
  \item \fullcite{schaefer.deriving}

    \begin{annotation}
      In this paper, Schaefer gives a catalog of several \PSPACE-complete
      games, with the goal of finding an extensive list of \PSPACE-complete
      problems useful for showing other problems to be \PSPACE-complete,
      analogous to Karp's 21 \NP-complete problems.

      Furthermore, Schaefer argues that many \NP-complete problems have natural
      \PSPACE-complete counterparts, demonstrating on the Hamiltonian-path
      puzzle as example how the proof of \NP-completeness is translated into a
      proof of \PSPACE-completeness of a Hamiltonian-path game.
    \end{annotation}

  \item \fullcite{schaefer.games}

    \begin{annotation}
      Here, Schaefer gives another catalog (overlapping in parts with
      \textcite{schaefer.deriving}) of combinatorial games and poses a few open
      questions on certain games' complexities.
    \end{annotation}

  \item \fullcite{demaine.acgt}
  \item \fullcite{bodlaender.coloring}
  \item \fullcite{bh.placement}
  \item \fullcite{kbd.impartial}
  \item \fullcite{cpss.coloring}
\end{itemize}

\section{Fixed-turn games}

\todo[inline]{incomplete}

\begin{itemize}
  \item \fullcite{ddls.sat-variants-ph}

    \begin{annotation}
      This paper studies a few \emph{quantifications} of several \SAT{}
      variants, such as not-all-equal-\Problem{3sat}, etc., in relation to the
      polynomial hierarchy.  This provides a useful catalog (and useful example
      proofs) of other problems characterizing the polynomial hierarchy beyond
      the standard \(\Problem{qsat}_i\) problems.
    \end{annotation}

  \item \fullcite{stockmeyer.ph}

    \begin{annotation}

    \end{annotation}

  \item \fullcite{papadimitriou.cc}

    \begin{annotation}

    \end{annotation}

\end{itemize}




%\bibliographystyle{hmcmath}
%\bibliography{newbib.bib}
%\bibliography{shitbib.bib}

\printbibliography[heading=bibintoc]

\end{document}
