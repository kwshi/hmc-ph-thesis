\documentclass{final-report}

\title{You can solve it, but can you play it?}

\subtitle{Puzzles, games, and the polynomial hierarchy}

\author{Kye Shi}
\advisor{Nicholas Pippenger}
\reader{Arthur T. Benjamin}
\date{2021 November}

\bibliography{bibliography.bib}
\bibliography{newbib.bib}

\definecolor{ks0}{HTML}{0050c0}
\definecolor{ks1}{HTML}{e00060}
\definecolor{ks2}{HTML}{f0b000}

\tikzset{
  gates/.style={
    row sep=1em/2,
    column sep=2em,
    matrix of nodes,
  },
  every circuit symbol/.style={
    fill,
    fill opacity=1/8,
    anchor=output,
  },
  vertex/.style={
    circle,
    draw,
    minimum size=1em/2,
    inner sep=0pt,
  },
  vertex label/.style={
    text height=2em/3,
    text depth=1em/3,
  },
  edge/.style={
  },
  wire/.style={
    rounded corners=1em/4,
    to path={
      -- ($ (\tikztostart)!1em!(\tikztostart -| \tikztotarget) $)
      -- ($ (\tikztotarget)!1em!(\tikztostart |- \tikztotarget) $)
      -- (\tikztotarget)
    },
  },
}

\begin{document}

\frontmatter
\maketitle
\tableofcontents

TODO:

\begin{itemize}
  \item Overview, maybe.
  \item Introduction: everybody knows P vs. NP, maybe PSPACE; this thesis
    explores the space in-between.  the author suspects that for every NP
    problem, there is a sigma1p analog, etc. blah blah; studying problems by
    thinking of them as games provides valuable intuition, etc.
  \item Abstract complexity background: P, NP, PSPACE, and polynomial hierarchy
    \begin{itemize}
      \item Circuit value, satisfiability games
    \end{itemize}
  \item SAT variation games
    \begin{itemize}
      \item Circuits could just as well be plain boolean formulae, or CNF
        formulae, etc.; list some prior work on other known SAT games in
        PSPACE, polynomial-hierarchy, etc.
    \end{itemize}
  \item Graph coloring games
    \begin{itemize}
      \item Intro---overview of existing results: NP-completeness of graph
        coloring, col and snort as PSPACE-complete games.
      \item Custom presentation of NP-completeness proof via circuit reduction
      \item Introduction of 2-turn game, 3-turn game, k-turn game, etc.
      \item Complexity claims and proofs
    \end{itemize}
  \item three-dimensional matching games
  \item Sudoku??
  \item Conclusion: only a few of countlessly many problems introduced/explored
\end{itemize}


\mainmatter

\chapter{Introduction}

The basic question of computational complexity---``how hard is this problem for
a computer to solve?''---is central to nearly every topic in computer science.
And yet the formalisms of complexity theory often seem, in my own experience,
intimidatingly abstract, phrased in terms of intangible models of computation
such as non-deterministic Turing machines and oracles.

The remedy, I believe, lies in studying complexity theory through the lens of
\emph{puzzles} and \emph{games}.  Not only do they provide a concrete grounding
for the abstractions, they also offer a particularly insightful, accessible,
and most importantly fun approach to understanding complexity theory.  In fact,
many of the most popularly known and appreciated results in complexity theory
are those about so-called ``\NP-complete puzzles'', such as Sudoku, and
``\PSPACE-complete games'', such as Checkers and Go.

This thesis emphasizes that approach in its exploration of a particularly
foundational, yet often overlooked, ladder of complexity classes known as the
\emph{polynomial hierarchy}.  \NP{} is the class of (one-player) ``puzzles'',
and \PSPACE{} is the class of (two-player) ``games'' of polynomial length; the
polynomial hierarchy, then, lies in the middle, encompassing games of
\emph{fixed} length.  Through this lens, the (in)famous \P-vs-\NP{} question is
but the first in a ladder of questions that are, arguably, just as crucial and
impactful.

%The polynomial hierarchy is as central to
%complexity theory as the \P-vs-\NP{} problem is well-known.

\section{Overview}

This document is structured as follows.  First, \cref{ch:background,ch:boolean}
establish preliminary background concepts and conventions adopted throughout
this thesis.  Next, \cref{ch:circuit} lays the central theoretical groundwork,
defining the \emph{polynomial hierarchy} through a fundamental family of
problems known as the \Problem{Circuit Satisfiability} games.  Next,
\cref{ch:misc} explores a novel family of games generalized from the
\Problem{Graph 3-Colorability} puzzle and establishes \emph{hardness} bounds on
each of those games.  Finally, \cref{ch:conclusion} concludes by discussing the
future directions of this work and its broader implications.

\section{Prior work and inspirations}

Much of the background exposition on complexity theory referenced in this thesis
is reproduced from Christos Papadimitriou's textbook,
\citet{papadimitriou.cc} (though many of the foundational ideas were
originally introduced/proven elsewhere, e.g.
\citet{cook.np,levin.np,stockmeyer.ph}), reframed through the
puzzles-and-games perspective and supplemented with a few comments on intuition.

The main family of games explored in this thesis, fixed-turn
\Problem{3-Colorability} games (\cref{ch:misc}), is a generalization of
(one-turn) \Problem{3-Colorability}, a well-known \NP-complete puzzle originally
proven \NP-complete by \citet{karp.np}.  Others have studied (multi-turn)
game generalizations of \Problem{3-Colorability}, but all versions that I've
encountered are \PSPACE-complete, in which the number of turns played during the
game scales proportionally with the size of the graph
\citep{bodlaender.coloring,bh.placement,kbd.impartial,cpss.coloring,schaefer.games}. As far as I'm aware, the variations I explore here—with fixed
numbers of turns regardless of the size of the graph—is unexplored, and the main
theorem about its \SigmaP k-completeness (\cref{th:yayay}) is novel.  The basic
idea underlying my proof is the composition of two well-known results:
\begin{itemize}[nosep]
  \item \citet{karp.np}'s classic proof of the \NP-hardness of the \Problem{3-Colorability} puzzle, via a reduction from \Problem{3CNF-Satisfiability};
  \item 's transformation from boolean circuits to equivalent 3CNF-clauses.
\end{itemize}

Without further ado, let's begin.


%TODO: outline/overview of chapters, after those chapters are written

%TODO: also give general citations here, e.g. papadimitriou for many
%foundational background info, etc.
%
%TODO: notation table also belongs in this chapter i think

%p-vs-np well known, polynomial-hierarchy central

%puzzles and games; hierarchy lies in the interstices.  we examine a few
%interesting (by no means exhaustive, or even close to comprehensive)
%np-complete puzzles with pspace-complete analogues, and we









%Famously central to the theory of computational complexity is the \P-vs-\NP{}
%question, and essential to our understanding of that question is the study of
%\NP-complete problems such as the Boolean Satisfiability puzzle, the Graph
%Colorability puzzle, and countless more.  Puzzles like these, which nearly any
%layperson can appreciate, offer a particularly insightful, intuitive, and
%\emph{fun} lens through which to study computational complexity. Explorations
%of more complex problem-classes such as \PSPACE{} can be similarly approached
%through the study of strategic decision \emph{games} such as Othello, Checkers,
%and Go.
%
%What lies in the interstices between \emph{puzzles} and \emph{games}?  How do
%we take a puzzle and generalize it into a game, and what are the puzzle-games
%we encounter along the way?  And how hard exactly are these puzzle-games to
%decide?  These questions are the focus of my thesis.

%So far, I have explored these questions from three angles:
%\begin{enumerate}
%
%  \item \label{itm:intro.q.generation} Puzzle generation.  If I wish to solve a
%    puzzle, you can play a game with me by constructing puzzle \emph{instances}
%    for me to solve.  For instance, \emph{solving} Sudoku is an \NP-complete
%    problem; your task is to \emph{generate} (partially-filled) Sudoku boards
%    for me to solve.
%
%    How hard is it to do so?  Moreover, how hard is it to generate \emph{good}
%    puzzle instances, for various definitions of \emph{good} (sufficiently
%    challenging to solve, or having unique solutions, or solvable/unsolvable by
%    certain strategies)?
%
%    % lauren sanchis
%
%  \item \label{itm:intro.q.pspace} \PSPACE-complete games derived from
%    \NP-complete puzzles.  A canonical \NP-complete puzzle is the \Problem{sat}
%    (Boolean Satisfiability) puzzle: given a Boolean formula \(\phi(x_1, \dots,
%    x_n)\), does there exist an assignment to its inputs \(x_1, \dots, x_n\)
%    such that \(\phi(\dots) = 1\)? In an analogous game, two players alternate
%    turns assigning \(x_1, \dots, x_n\); player 1 wins if \(\phi(\dots)=1\),
%    and player 2 wins if \(\phi(\dots)=0\).  Does either player have a
%    (guaranteed) winning strategy?  This game, known as \Problem{qsat}
%    (Quantified Satisfiability), is a canonical example of a \PSPACE-complete
%    game.
%
%    Can other \NP-complete puzzles be similarly generalized into games?  Will
%    those games also be \PSPACE-complete?
%
%    % schaefer
%
%  \item \label{itm:intro.q.ph} Fixed-turn games and the polynomial hierarchy.
%    In between the complexity classes \NP{} and \PSPACE{} lies a chain of
%    increasingly-complex problem-classes known as the \emph{polynomial
%    hierarchy}.  In some cases, problems in the polynomial hierarchy may be
%    thought of as game generalizations of \NP-complete puzzles with a
%    \emph{fixed} number of turns.  For instance, in a two-turn version of
%    \Problem{sat}, inputs are partitioned into two (disjoint) groups \(X_1\)
%    and \(X_2\); on turn 1, player 1 assigns \(X_1\), and on turn 2, player 2
%    assigns \(X_2\).  As before, player 1 (respectively 2) wins if
%    \(\phi(\dots) = 1\) (respectively \(0\)).  Determining whether player 1 has
%    a winning strategy is complete for a complexity class known as
%    \(\SigmaP2\), which lies just above \NP{} in the hierarchy, and analogous
%    games with \(k\) turns are \(\SigmaP k\)-complete.
%
%    Do polynomial-hierarchy generalizations of other \NP-complete puzzles
%    exist?
%
%\end{enumerate}
%
%\Cref{ch:progress} discusses my progress so far in each of these areas.
%Questions \ref{itm:intro.q.generation} and \ref{itm:intro.q.pspace} have been
%explored in-depth by others, while question \ref{itm:intro.q.ph} appears to be
%scarcely explored.  As such, I provide only brief summaries of/reflections on
%the existing work pertaining to \ref{itm:intro.q.generation} and
%\ref{itm:intro.q.pspace}.  Meanwhile, I describe in greater detail question
%\ref{itm:intro.q.ph}, which is the focus of my explorations so far.
%
%\Cref{ch:future} summarizes the primary questions \& goals that will guide
%my exploration next semester.
%
%Finally, \cref{ch:bib} contains an annotated bibliography of existing work
%pertaining to each of these topics.



\chapter{The basics}

The fundamental question driving the study of computational complexity theory
is, ``how difficult are certain problems for computers to solve?''  In order to
answer this question precisely, we must start by figuring out what exactly it
asks.  That is, formally, what do we mean by \emph{difficulty}?  For that
matter, what constitutes a \emph{problem}?  What counts as a \emph{computer}?

Conventionally, \emph{computers} are formalized as Turing machines, with
\emph{difficulty} being measured by the number of Turing machine execution
steps.  For the purposes of this thesis, we avoid delving into the formalism of
Turing machines.  Instead, we assume an informal notion of computers given by
any algorithm or procedure straightforwardly implementable in modern,
high-level programming languages such as C/C++, Python, Java, etc.  Detailed
treatment of the relevant formalisms may be found in \textcite[Chapter
2]{papadimitriou.cc}.  In particular, there are theorems \parencite[Theorem
2.5]{papadimitriou.cc} showing that modern CPU/RAM-based computer architectures
are, for our purposes, equivalent to Turing machines, thereby justifying the
informal approach we take here.

In the following sections, we discuss what exactly constitutes a
\emph{problem}, how we describe the complexity (i.e., difficulty) of problems,
and how we categorize problems by difficulty into \emph{complexity classes}.

%emphasize intuitive descriptions of
%algorithms in terms of modern, ``high-level'' programming concepts exhibited by
%programming languages such as Python.

%An
%alternative treatment uses ``random access machines'', which mimic modern
%CPU/RAM-based computer architectures.  In this thesis, we avoid delving into
%these formal details.

%in terms of modern, high-level programming concepts

%In the conventional formalism, computers are modeled as Turing machines.
%Difficulty, then, refers to the number of execution steps required by a Turing
%Machine to solve a problem.  Alternatively, computers could be modeled as
%``random access machines'' \parencite[Section 2.6]{papadimitriou.cc}, which
%mimic modern CPU/RAM-based computer architectures.  For our purposes, the two
%models of computation are equivalent \parencite[Theorem 2.5]{papadimitriou.cc}.

%Formal treatments of these
%definitions are found in \textcite[Chapter 2]{papadimitriou.cc}

%Of particular
%note, \textcite[Theorem 2.5]{papadimitriou.cc} shows that these two models of
%computation are, for our purposes, equivalent.  Therefore, for our purposes, we will
%assume the latter model, allowing us to think in terms of modern programming
%patterns and give

%By \emph{hardness}, what we really mean is: given a problem input
%encoded in \(n\) bits, how much computational time, asymptotically with respect
%to \(n\), is required to solve that problem?  In order to discuss this question
%precisely, we have to clearly define what we mean by ``computational time'',
%and, for that matter, what we mean by ``computer''.  A traditional approach
%takes ``computer'' to mean Turing Machines and ``time'' to be Turing Machine
%execution steps; another approach defines ``computer'' via modern-day,
%CPU/RAM-based architectures, with ``time'' given by CPU instruction cycles.

%relatively \emph{informal} descriptions of algorithms.

\section{Decision problems}

The simplest flavor of computational problem is a \emph{decision problem}, or a
yes/no question: given an input \(X\), does \(X\) satisfy certain conditions?
Here are some examples of decision problems:
\begin{itemize}[nosep]
  \item Given an integer \(K\), is \(K\) even?
  \item Given a string of letters \(S\), is \(S\) a palindrome?
  \item A silly decision problem, but nevertheless a valid one: given any input
    \(X\), always return ``yes''.
\end{itemize}

In order for a yes/no question to qualify as a decision problem, it must be
stated in terms of an arbitrary input.  For instance, consider the following
question:
\begin{itemize}[nosep]
  \item Is \(314159\) a prime number?  (Answer: yes.  Proof: see WolframAlpha.)
\end{itemize}
This is a yes/no question, but it takes no inputs (the value \(314159\) is not
an input; it is merely part of the question statement).  In this sense, it is
computationally uninteresting: in order to solve this question, an algorithm
only needs to return the fixed answer ``yes''.  In contrast, what we're really
interested in is the general problem of primality testing:
\begin{itemize}[nosep]
  \item Given an arbitrary positive integer \(K\), is \(K\) prime?
\end{itemize}

We formalize the definition of decision problems below.

\begin{definition}{(decision) problem}{}

  A \Term{decision problem} is a function
  \(Π\colon\Set{0,1}^*→\Set{\text{yes},\text{no}}\).  Equivalently, a
  \Term{decision problem} is the set \(Π⊆\Set{0,1}^*\) comprising exactly the
  inputs, a.k.a. \Term{instances}, that result in ``yes'' answers.

  That is, for any input \(X∈\Set{0,1}^*\), we say \(X∈Π\) (in the \emph{set}
  sense) if \(Π(X)=\text{yes}\) (in the \emph{function} sense), and \(X∉Π\) to
  mean \(Π(X)=\text{no}\).

  \begin{aside}
    Formally, inputs to decision problems are always encoded as binary strings.
    Essentially, this requirement follows from the fact that all modern
    computers encode data in binary anyway.  Furthermore, it allows us to
    rigorously discuss notions such as \emph{input size}.  This is an important
    formal detail, but for the most part, we avoid dealing with any binary
    encoding/decoding technicalities.  We mention this detail here only to
    clarify the role of \(\Set{0,1}^*\) in the definition above.
  \end{aside}

\end{definition}

There is another notion of \emph{problems}, called \Term{function problems}, as
arbitrary (binary-encoded) functions \(\Set{0,1}^*→\Set{0,1}^*\).  However,
function problems seem to generally receive less attention than decision
problems, perhaps because decision problems are conceptually simpler but still
versatile enough to capture the core ideas of complexity theory.  Whatever the
reason, this thesis abides by that tradition.  As such, the vast majority of
the problems examined in this thesis are decision problems, so for convenience
we will simply say ``problems'' to mean decision problems, unless otherwise
specified.

\section{Complexities and classes}

When we ask how difficult a problem is, we are essentially asking, how much
time (or other resources, such as memory) does a computer need to solve that
problem?  Of course, the answer depends on the input: some inputs are easy to
solve, and others are harder.  Certainly, we expect the difficulty to scale
with input size: the larger the input, the more work it generally takes an
algorithm to process it.  Thus, the complexity of a problem is given as a
function of the input size.  Specifically, we ask, if an algorithm is given an
input string (recall, encoded in binary) of length \(n\), how much time in the
worst case is required, as a function of \(n\)?

However, exact function bounds are unnecessarily sensitive to pedantic
technicalities, e.g., slight variations in implementations of the same
algorithm, or specific details in the formal models of ``computer''. Instead,
loosely speaking, we are mostly interested in how these costs asymptotically
\emph{scale} as the input size gets large.  Thus, we categorize problems with
``similar'' complexities into \emph{complexity classes}.

So then, what counts as \emph{similar}?  As a starting approximation, we assert
that \emph{polynomials are small}: any algorithm whose running time is bounded
by some polynomial function is considered relatively ``fast''; problems with
polynomial-time solutions are considered relatively ``easy''.  We formalize
this idea in the definition of the complexity class \P{} below.

\begin{definition}{Polynomial-time problems, \P}{}

  Let \(A\) be an algorithm computing some (decision) problem (i.e., it takes a
  binary string as input and returns ``yes'' or ``no'').  We say \(A\) runs in
  \Term{polynomial time} if there exists some polynomial \(p\) such that, on
  any input \(X∈\Set{0,1}^*\), the algorithm \(A\) terminates in \(≤p(\Abs X)\)
  steps.

  The complexity class \P{} is the set of (decision) problems correctly
  solvable in polynomial time.

  \begin{aside}
    For contrast, we say an algorithm is \Term{super-polynomial} if its running
    time cannot be bounded by some polynomial.  Examples of super-polynomial
    functions include \(n^{\log n}\), \(2ⁿ\), etc.
  \end{aside}

\end{definition}

To be clear, taking polynomials to mean ``easy'' is a very crude rule-of-thumb:
there are important practical subdivisions \emph{within} \P{} that this
categorization plainly ignores (e.g., linear-time vs quadratic-time); there are
also a few notable examples of super-polynomial-time algorithms that are, by
this rule, slow, but quite efficient \emph{in practice} (e.g., the simplex
algorithm for linear programming).  Nevertheless, this delineation remains an
extremely useful (and arguably elegant) starting point for the classification
of problems.

% TODO examples of 𝐏 problems, and perhaps discuss papadimitriou theorem 2.5
% again with more precision?

\section{Hard problems and reductions}

Above, we establish that a problem is considered \emph{easy} if it has a
polynomial-time solution.  Hard problems, then, are those without
polynomial-time solutions… right?  Sure.  But how do we go about showing that a
problem is actually hard?  And how hard, exactly?

For an easy problem, proving \emph{existence} of a polynomial-time algorithm is
straightforward---simply construct one.  On the other hand, for a problem that
appears to be hard, we would have to prove \emph{non-existence} of a
polynomial-time algorithm---that it is \emph{impossible} to find a
polynomial-time algorithm. In general, this is incredibly difficult to show;
this difficulty is a large part of why the infamous \P-vs-\NP{} question
remains unsolved.

Thus, we take a different approach to understanding hard problems: by comparing
them to each other.

TODO: simple examples of two problems that reduce to each other, then
definition of reduction.  then definition of completeness.


Suppose we have two problems, \(Π₁\) and \(Π₂\).


%\section{Puzzles, non-determinism, and \NP}

%Consider the following problem.
%\begin{itemize}
%  \item Given a graph \(Γ\) and a positive integer \(k\), does \(Γ\) contain a
%    clique of size \(k\)?
%\end{itemize}


%\section{Hard problems and completeness}

%Notationally, we say \(X\) is \emph{in} the problem \(L\), or \(X\in L\), if
%the answer is yes; otherwise, we say \(X\notin L\).
%
%An example of a decision problem is the graph reachability problem:
%\begin{definition}[\Problem{reachability}]%
%  Given a graph with \(n\) vertices \(v_1, \dots, v_n\), does there exist a
%  path connecting \(v_1\) to \(v_n\)?
%\end{definition}
%This problem may be solved using simple graph-search algorithms such as
%Breadth-First/Depth-First Search, whose asymptotic running time is \(\O(n)\)
%---that is, bounded by a linear function of \(n\).  As such, this problem is
%considered relatively ``easy'' to solve.
%
%More generally, \Problem{reachability} belongs to the class of decision
%problems known as \P:
%\begin{definition}[\P]%
%  The class of decision problems whose solution runtime is bounded by a
%  polynomial function of the input length.
%\end{definition}
%We consider problems in \P{} to be ``easy''---at least, from the standpoint of
%computational complexity.
%
%Another example of a decision problem is the Hamiltonian path problem:
%\begin{definition}[\Problem{hamiltonian-path}]%
%  \label{def:hamiltonian-path} Given a graph with \(n\) vertices, does it
%  contain a Hamiltonian path (i.e., a path that visits each vertex exactly
%  once)?
%\end{definition}
%This problem is not known to be in \P.  In fact, the best known algorithms
%solving \Problem{hamiltonian-path} are essentially brute-force guess-and-check:
%\emph{guess} a possible Hamiltonian path (e.g., by writing down some
%permutation of the vertices), then \emph{check} that it is valid (e.g., that
%each pair of adjacent vertices in the guessed path are actually connected by an
%edge in the graph).  In the worst case, if our guesses are really unlucky, we
%may have to repeat up to \(n!\) iterations, which is definitely not polynomial.
%However, setting aside the cost associated with brute-forcing guesses, note
%that individual \emph{checking} steps \emph{do} run in polynomial time.
%Problems like this, which are solvable via guess-and-check, where the ``check''
%problem is in \P, belong to a class of problems known as \NP:
%\begin{definition}[\NP]%
%  \label{def:np} A decision problem \(L\) is in \NP{} if\dots
%  \begin{nested}
%    there exists a corresponding decision problem \(L'\in\P\) (intuitively: the
%    ``check'' problem) and a polynomial \(p\) such that\dots
%    \begin{nested}
%      for all input strings \(x\)\dots
%      \begin{nested}
%        \(x \in \NP\) if and only if\dots
%        \begin{nested}
%          there exists a ``guess'' \(g\) with length \(\Abs g \le p(\Abs x)\)
%          such that \((x, g) \in L'\) (intuitively: \(g\) passes the
%          ``check'').
%        \end{nested}
%      \end{nested}
%    \end{nested}
%  \end{nested}
%
%  Note that the \(\Abs g \le p(\Abs x)\) requirement is present in order to
%  ensure that the guesses are not so obscenely long as to abuse the idea of
%  ``efficient'' checking.  This requirement is not central to understanding the
%  definition of \NP{} but is nevertheless an important technical subtlety.
%\end{definition}
%
%The infamous \P-vs-\NP{} open question asks: is \NP{} truly more difficult than
%\P?  Does there exist some problem in \NP{} that definitively cannot be solved
%within polynomial time?  I, a baby undergraduate, am not in the business of
%answering that question.
%
%As such, the best we can do to determine the difficulty of a given problem is
%to compare them to other problems, deriving a \emph{relative} ordering telling
%us which problems are easier/harder than other ones.  To this end, we must
%define what easier/harder means---intuitively, we think of a problem \(L_1\) as
%easier than another problem \(L_2\) if knowing how to solve \(L_2\)
%automatically also tells us how to solve \(L_1\), with minimal
%(polynomially-bounded) overhead.  More precisely:
%\begin{definition}[reductions]
%  \label{def:reduction}
%  Let \(L_1\) and \(L_2\) be decision problems.  We say \(L_1\) is
%  \emph{reducible to} \(L_2\), or that \(L_1\) is \emph{at least as easy as}
%  \(L_2\)'', denoted \(L_1 \le L_2\), if\dots
%  \begin{nested}
%    there exists a function \(f\), called a \emph{reduction}, converting input
%    strings for \(L_1\) to inputs for \(L_2\), such that \(f\) is computable
%    within polynomial time, and\dots
%    \begin{nested}
%      for any input \(x_1\)\dots
%      \begin{nested}
%        \(x_1 \in L_1\) if and only if \(x_2 \in L_2\).
%      \end{nested}
%    \end{nested}
%  \end{nested}
%
%  Note that this definition of reductions is slightly different than the one
%  given in \textcite{papadimitriou.cc}, whose requirement on \(f\) is that it
%  is computable in \emph{logarithmic-space} rather than polynomial-time.
%  However, for the purposes of this project, the distinction between the two is
%  unimportant.
%\end{definition}
%
%This notion of comparison also gives us a good way of comparing problems to
%entire classes:
%\begin{definition}[hardness and completeness]%
%  \label{def:hard-complete}
%  Let \(\mathbfit C\) be a complexity class.
%  \begin{itemize}[nosep]
%    \item A problem \(L\) is \emph{hard for \(\mathbfit C\)}, or
%      \emph{\(\mathbfit C\)-hard}, if \(L\ge K\) for every \(K\in\mathbfit C\).
%    \item A problem \(L\) is \emph{complete for \(\mathbfit C\)}, or
%      \emph{\(\mathbfit C\)-complete}, if \(L\) is \(\mathbfit C\)-hard
%      \emph{and} \(L\in\mathbfit C\).
%  \end{itemize}
%\end{definition}
%
%In particular, \emph{complete} problems for a class \(\mathbfit C\) are at
%least as hard as everything else in \(\mathbfit C\) and simultaneously
%themselves \emph{in} \(\mathbfit C\).  In this sense, for any complexity class,
%its complete problems are its \emph{hardest} problems, giving us an effective,
%``exact'' characterization of the class in terms of its problems.
%
%This approach to characterizing complexity classes is the driving motivation
%behind our exploration of puzzles and games.
%
%%\todo[inline]{unfinished.  formalism of turing machines, decision problems,
%%  oracles \& the definition of polynomial hierarchy, proofs of completeness of
%%  SAT \& QSAT for classes in the polynomial hierarchy.  I imagine this stuff
%%  will be needed in the final thesis; is it needed also for the midyear
%%report?}
%%
%%\begin{definition}[decision problem/language]%
%%  A \textbf{decision problem} is a yes/no question posed on binary input
%%  strings, or problem \textbf{instances}.  As such, we may think of a decision
%%  problem as a mapping
%%  \[
%%    L \colon \Set{0, 1}^* \to \Set{\text{yes}, \text{no}}.
%%  \]
%%
%%  More commonly, we associate a problem with its ``yes'' instances, the set of
%%  which is a \textbf{language}:
%%  \[
%%    L(L) = \SetBuilder* {x \in \Set{0, 1}^*} {L(x) = \text{yes}}.
%%  \]
%%  Here, for clarity, we are distinguishing notationally between \(L\) and
%%  \(L(L)\), but in general we conflate the two notions and refer to both as
%%  the problem \(L\).
%%\end{definition}
%%
%%\begin{definition}[\NP]
%%  \NP{} is the class of problems solvable by a \emph{non-deterministic} Turing
%%  machine in \emph{polynomial time}.
%%\end{definition}
%%
%%
%%
%%


%\chapter{Introduction}

The basic question of computational complexity---``how hard is this problem for
a computer to solve?''---is central to nearly every topic in computer science.
And yet the formalisms of complexity theory often seem, in my own experience,
intimidatingly abstract, phrased in terms of intangible models of computation
such as non-deterministic Turing machines and oracles.

The remedy, I believe, lies in studying complexity theory through the lens of
\emph{puzzles} and \emph{games}.  Not only do they provide a concrete grounding
for the abstractions, they also offer a particularly insightful, accessible,
and most importantly fun approach to understanding complexity theory.  In fact,
many of the most popularly known and appreciated results in complexity theory
are those about so-called ``\NP-complete puzzles'', such as Sudoku, and
``\PSPACE-complete games'', such as Checkers and Go.

This thesis emphasizes that approach in its exploration of a particularly
foundational, yet often overlooked, ladder of complexity classes known as the
\emph{polynomial hierarchy}.  \NP{} is the class of (one-player) ``puzzles'',
and \PSPACE{} is the class of (two-player) ``games'' of polynomial length; the
polynomial hierarchy, then, lies in the middle, encompassing games of
\emph{fixed} length.  Through this lens, the (in)famous \P-vs-\NP{} question is
but the first in a ladder of questions that are, arguably, just as crucial and
impactful.

%The polynomial hierarchy is as central to
%complexity theory as the \P-vs-\NP{} problem is well-known.

\section{Overview}

This document is structured as follows.  First, \cref{ch:background,ch:boolean}
establish preliminary background concepts and conventions adopted throughout
this thesis.  Next, \cref{ch:circuit} lays the central theoretical groundwork,
defining the \emph{polynomial hierarchy} through a fundamental family of
problems known as the \Problem{Circuit Satisfiability} games.  Next,
\cref{ch:misc} explores a novel family of games generalized from the
\Problem{Graph 3-Colorability} puzzle and establishes \emph{hardness} bounds on
each of those games.  Finally, \cref{ch:conclusion} concludes by discussing the
future directions of this work and its broader implications.

\section{Prior work and inspirations}

Much of the background exposition on complexity theory referenced in this thesis
is reproduced from Christos Papadimitriou's textbook,
\citet{papadimitriou.cc} (though many of the foundational ideas were
originally introduced/proven elsewhere, e.g.
\citet{cook.np,levin.np,stockmeyer.ph}), reframed through the
puzzles-and-games perspective and supplemented with a few comments on intuition.

The main family of games explored in this thesis, fixed-turn
\Problem{3-Colorability} games (\cref{ch:misc}), is a generalization of
(one-turn) \Problem{3-Colorability}, a well-known \NP-complete puzzle originally
proven \NP-complete by \citet{karp.np}.  Others have studied (multi-turn)
game generalizations of \Problem{3-Colorability}, but all versions that I've
encountered are \PSPACE-complete, in which the number of turns played during the
game scales proportionally with the size of the graph
\citep{bodlaender.coloring,bh.placement,kbd.impartial,cpss.coloring,schaefer.games}. As far as I'm aware, the variations I explore here—with fixed
numbers of turns regardless of the size of the graph—is unexplored, and the main
theorem about its \SigmaP k-completeness (\cref{th:yayay}) is novel.  The basic
idea underlying my proof is the composition of two well-known results:
\begin{itemize}[nosep]
  \item \citet{karp.np}'s classic proof of the \NP-hardness of the \Problem{3-Colorability} puzzle, via a reduction from \Problem{3CNF-Satisfiability};
  \item 's transformation from boolean circuits to equivalent 3CNF-clauses.
\end{itemize}

Without further ado, let's begin.


%TODO: outline/overview of chapters, after those chapters are written

%TODO: also give general citations here, e.g. papadimitriou for many
%foundational background info, etc.
%
%TODO: notation table also belongs in this chapter i think

%p-vs-np well known, polynomial-hierarchy central

%puzzles and games; hierarchy lies in the interstices.  we examine a few
%interesting (by no means exhaustive, or even close to comprehensive)
%np-complete puzzles with pspace-complete analogues, and we









%Famously central to the theory of computational complexity is the \P-vs-\NP{}
%question, and essential to our understanding of that question is the study of
%\NP-complete problems such as the Boolean Satisfiability puzzle, the Graph
%Colorability puzzle, and countless more.  Puzzles like these, which nearly any
%layperson can appreciate, offer a particularly insightful, intuitive, and
%\emph{fun} lens through which to study computational complexity. Explorations
%of more complex problem-classes such as \PSPACE{} can be similarly approached
%through the study of strategic decision \emph{games} such as Othello, Checkers,
%and Go.
%
%What lies in the interstices between \emph{puzzles} and \emph{games}?  How do
%we take a puzzle and generalize it into a game, and what are the puzzle-games
%we encounter along the way?  And how hard exactly are these puzzle-games to
%decide?  These questions are the focus of my thesis.

%So far, I have explored these questions from three angles:
%\begin{enumerate}
%
%  \item \label{itm:intro.q.generation} Puzzle generation.  If I wish to solve a
%    puzzle, you can play a game with me by constructing puzzle \emph{instances}
%    for me to solve.  For instance, \emph{solving} Sudoku is an \NP-complete
%    problem; your task is to \emph{generate} (partially-filled) Sudoku boards
%    for me to solve.
%
%    How hard is it to do so?  Moreover, how hard is it to generate \emph{good}
%    puzzle instances, for various definitions of \emph{good} (sufficiently
%    challenging to solve, or having unique solutions, or solvable/unsolvable by
%    certain strategies)?
%
%    % lauren sanchis
%
%  \item \label{itm:intro.q.pspace} \PSPACE-complete games derived from
%    \NP-complete puzzles.  A canonical \NP-complete puzzle is the \Problem{sat}
%    (Boolean Satisfiability) puzzle: given a Boolean formula \(\phi(x_1, \dots,
%    x_n)\), does there exist an assignment to its inputs \(x_1, \dots, x_n\)
%    such that \(\phi(\dots) = 1\)? In an analogous game, two players alternate
%    turns assigning \(x_1, \dots, x_n\); player 1 wins if \(\phi(\dots)=1\),
%    and player 2 wins if \(\phi(\dots)=0\).  Does either player have a
%    (guaranteed) winning strategy?  This game, known as \Problem{qsat}
%    (Quantified Satisfiability), is a canonical example of a \PSPACE-complete
%    game.
%
%    Can other \NP-complete puzzles be similarly generalized into games?  Will
%    those games also be \PSPACE-complete?
%
%    % schaefer
%
%  \item \label{itm:intro.q.ph} Fixed-turn games and the polynomial hierarchy.
%    In between the complexity classes \NP{} and \PSPACE{} lies a chain of
%    increasingly-complex problem-classes known as the \emph{polynomial
%    hierarchy}.  In some cases, problems in the polynomial hierarchy may be
%    thought of as game generalizations of \NP-complete puzzles with a
%    \emph{fixed} number of turns.  For instance, in a two-turn version of
%    \Problem{sat}, inputs are partitioned into two (disjoint) groups \(X_1\)
%    and \(X_2\); on turn 1, player 1 assigns \(X_1\), and on turn 2, player 2
%    assigns \(X_2\).  As before, player 1 (respectively 2) wins if
%    \(\phi(\dots) = 1\) (respectively \(0\)).  Determining whether player 1 has
%    a winning strategy is complete for a complexity class known as
%    \(\SigmaP2\), which lies just above \NP{} in the hierarchy, and analogous
%    games with \(k\) turns are \(\SigmaP k\)-complete.
%
%    Do polynomial-hierarchy generalizations of other \NP-complete puzzles
%    exist?
%
%\end{enumerate}
%
%\Cref{ch:progress} discusses my progress so far in each of these areas.
%Questions \ref{itm:intro.q.generation} and \ref{itm:intro.q.pspace} have been
%explored in-depth by others, while question \ref{itm:intro.q.ph} appears to be
%scarcely explored.  As such, I provide only brief summaries of/reflections on
%the existing work pertaining to \ref{itm:intro.q.generation} and
%\ref{itm:intro.q.pspace}.  Meanwhile, I describe in greater detail question
%\ref{itm:intro.q.ph}, which is the focus of my explorations so far.
%
%\Cref{ch:future} summarizes the primary questions \& goals that will guide
%my exploration next semester.
%
%Finally, \cref{ch:bib} contains an annotated bibliography of existing work
%pertaining to each of these topics.



%\chapter{Current progress}

\label{ch:progress}

\section{\NP{} as puzzles, or one-move games}

Recall that \NP{} is the class of problems solvable by guess-and-check, with a
\emph{check} problem in \P{} (\cref{defn:np}):
\[
  \NP = \SetBuilder{L}{
    \exists \underbrace{\mathstrut L' \in \P}_{\mathclap{\text{the ``check'' problem}}} \;
    \forall x \quad
    x \in L \iff \underbrace{\exists g \; (x, g) \in L'}_{\mathclap{\text{guess-and-check}}}
  }.
\]
(In the above, it is \emph{implicitly} required that \(\Abs g\) be
polynomially-bounded with respect to \(\Abs x\), but we have omitted it in
notation for readability.)

Another famous example of a problem in \NP{} is \Problem{sudoku}, framed as the
following decision problem:
\begin{definition}[\Problem{sudoku}]%
  We are given a square grid with dimensions \(n^2\times n^2\), some of whose
  cells are filled in with numbers in \(\{1,\dotsc,n^2\}\).  The grid is
  evenly partitioned into \(n\) chunks along each axis, resulting in \(n^2\)
  \emph{blocks} each with dimensions \(n\times n\).

  Does there exist a way to fill in the rest of the cells so that each row,
  column, and block on the grid contains each number in \(\{1,\dotsc,n^2\}\)
  exactly once?
\end{definition}

\todo[inline]{TODO add illustration of Sudoku board}

For this problem, a ``guess'' \(g\) consists of a list of numbers in
\(\{1,\dotsc,n^2\}\) specifying the values with which to fill in the empty
cells in the given grid.  The ``check'' problem, then, is stated as follows:
\begin{nested}
  Given a fully-filled-in Sudoku board, does each row, column, and block on the
  grid contain each of \(\{1,\dotsc,n^2\}\) exactly once?
\end{nested}




%Also recall an example of an \NP{} problem, \Problem{hamiltonian-path}
%(\cref{def:hamiltonian-path}), which asks: given a graph, does it have a
%Hamiltonian path?  Here, the ``check'' problem \(L'\) can be stated as follows:
%\begin{nested}
%  Given a graph \(x = \Gamma\) with vertices \(v_1,\dotsc,v_n\), along with a
%  permutation \(g = \phi(1),\dotsc,\phi(n)\), does the sequence
%  \(v_{\phi(1)},\dotsc,v_{\phi(n)}\) specify a valid path on \(\Gamma\)?
%\end{nested}
%
%We can now intuitively reframe \Problem{hamiltonian-path} as a one-player
%``game'', played on an input ``board'' in the form

%in which the player writes down some
%permutation \(\phi\).  They ``win'' if it meets the validity condition \((x, g)
%\in L'\) and ``lose'' if it doesn't.  Under this framing, the decision problem
%becomes the following question: does the player have a winning \emph{strategy}?

\section{Multi-turn games and the polynomial hierarchy}

Consider, now, a two player, a ``solver'' and an
``adversary'', in two turns:
\begin{itemize}
  \item First, the solver
  \item Next, the adversary
\end{itemize}

This ``guess-and-check'' extension of \P{} may be continued

to define higher
and higher complexity classes, known as the \emph{polynomial hierarchy}:
\begin{definition}[polynomial hierarchy]
  \begin{align*}
    \SigmaP1 &= \NP = \SetBuilder{L}{\exists L' \in \P \; \forall x \quad
      x \in L \iff \exists g \; (x, g) \in L'
    }, \\
    \SigmaP2 &= \NP = \SetBuilder{L}{\exists L' \in \P \; \forall x \quad
      x \in L \iff \exists g \; (x, g) \in L'
    }, \\
  \end{align*}
\end{definition}




%In general, any complexity class has a guess-and-check equivalent, called its
%\emph{projection}:
%\begin{definition}[projection]
%  Let \(\C\) be a complexity class.  Its \emph{projection} is the class of
%  problems
%  \[
%    \pro\C = \SetBuilder*{L}{
%      \exists L' \in \C, \text{polynomial \(p\)}; \; \forall x \quad
%      x \in L \iff
%      \exists g \; \text{\(\Abs g \le p(\Abs x)\) and \((x, g) \in L'\)}
%    }.
%  \]
%
%  (Observe, then, that \(\NP = \pro\P\).)
%\end{definition}








\section{Boolean Satisfiability}

\subsection{The Satisfiability puzzle}

\todo{context on what booleans are?}

Our puzzles-and-games characterization of the polynomial hierarchy begins with
a well-known family of problems generally referred to as Boolean Satisfiability
problems.  Here is perhaps the simplest, most well-known Satisfiability puzzle:

\begin{definition}[\Problem{sat}]%
  Given a Boolean formula \(\phi(x_1, \dots, x_n)\), does there exist an
  assignment of Boolean values to inputs \(x_1, \dots, x_n\) such that
  \(\phi(x_1, \dots, x_n) = 1\)?  \Problem{sat} consists of the formula
  instances for which the answer is \emph{yes}.

  Formally:
  \[
    \Problem{sat} = \SetBuilder \phi {
      \exists (x_1, \dots, x_n) \in \Set{0,1}^n \quad \phi(x_1, \dots, x_n) = 1
    }.
  \]
\end{definition}

The \SAT{} puzzle is particularly useful and worth studying because of its
generality.  Booleans form the foundation of mathematical logic: every logical
statement can be encoded, in some manner, as a Boolean formula.  Consequently,
\SAT{} is, on an intuitive level, the most general possible puzzle---given any
other puzzle, encoding its rules in terms of Booleans reveals that it is merely
a special case of \SAT.  This idea is expressed formally as the Cook-Levin
theorem:

\begin{theorem}[Cook-Levin]
  \SAT{} is \NP-complete.
\end{theorem}

%\begin{proof}
%  \todo[inline]{put proof.  the most important reason to have the proof here is
%  to illustrate}
%\end{proof}

\subsection{Satisfiability games}

\begin{definition}[The two-turn \SAT{} games]%
  The two-turn \SAT{} game is played on a Boolean formula \(\phi(x_1, \dots,
  x_n, y_1, \dots, y_n)\) with inputs partitioned into two groups \(X =
  \Set{x_i}\) and \(Y = \Set{y_i}\).  The two turns proceed as follows:
  \begin{enumerate}
    \item Player 1 assigns values to \(X\).
    \item Player 2 assigns values to \(Y\).
  \end{enumerate}
  Player 1 wins if \(\phi\) is satisfied (\(\phi(\dots) = 1\)), and player 2
  wins if \(\phi\) is falsified.

  Who wins?  Two decision problems arise from this game:
  \begin{itemize}
    \item Does player 1 have a winning strategy?  That is, can player 1 make
      some first move so that no matter what player 2 does, player 1 always
      wins?

    \item Does player 2 have a winning strategy?  That is, no matter what
      player 1 plays, can player 2 respond with some move guaranteeing a win?
  \end{itemize}

\end{definition}


\section{Graph coloring}

\begin{definition}[\Problem{3col}]%
  Given a graph \(\Gamma\), is there a way to color each vertex in  \(\Gamma\)
  with one of three colors so that every pair of adjacent vertices has distinct
  colors?
\end{definition}

\begin{theorem}
  \Problem{3col} is \NP-complete.
\end{theorem}

\subsection{Graph coloring games}

\begin{definition}[Two-turn \Problem{3col}]%
  The two-turn \Problem{3col} game is played on a graph \(\Gamma\) whose
  vertices are partitioned into two (disjoint) groups \(X\) and \(Y\).  Two
  players take turns assigning one of three colors to vertices.  First, player
  1 colors vertices in \(X\); second, player 2 colors  vertices in \(Y\).
  Player 1 wins if the resulting coloring is \emph{invalid}---that is, there
  exists a pair of vertices sharing the same color; player 2 wins if the
  resulting coloring is valid.

\end{definition}

\begin{conjecture}
  Two-turn \Problem{3col} is complete for \SigmaP2 (or \PiP2, depending on
  which player's winning strategy we examine).
\end{conjecture}










%\section{Puzzle generation}
%
%\label{sec:progress.generation}
%
%The topic of puzzle generation difficulty has been explored in detail by Laura
%Sanchis
%\parencite{language-instances,test-gen-complexity,hard-diverse-graph-tests}.
%
%\todo[inline]{incomplete; i'm focusing my effort on the fixed-turn section first
%because that's more novel/interesting}
%
%\subsection{Puzzles with unique solutions}
%
%In many popularly-known puzzle games, one criterion for ``good'' puzzle
%generation is that the generated puzzle instance should have a unique solution.
%For example, given a \(9\times9\) Sudoku grid (partially pre-filled with
%numbers \(1,\dotsc,9\)), there should be \emph{exactly one} way to complete the
%grid---no more, no less.
%
%This formulation is incompatible with our
%
%\section{\PSPACE-complete games}
%
%\label{sec:progress.pspace}
%
%\todo[inline]{incomplete}
%
%\section{Fixed-turn games}
%
%\label{sec:progress.ph}
%
%

%\chapter{Future work}

\label{ch:future}

Since I did most thinking about the polynomial hierarchy question, most of the
``future work'' (basically, next semester) will be about that.  Concrete open
questions and proof TODOs go here, puzzles I want to investigate immediately
following graph-coloring, etc.  As time permits, the two other areas.

\section{Goals?  Timelines?}

For clinic, we had to propose timelines for deliverables/assessments of
``success''.  I don't know if the same sort of thing applies to thesis, but
that's why this section exists.



%\chapter{Annotated bibliography}
\label{ch:bib}

\section{Puzzle generation}

\todo{fix formatting/combine with automatically-generated bibliography}

\todo{this is basically my full annotated bibliography from before on the topic
  of puzzle generation.  the focus of my topic has shifted somewhat since then
  away from generation and towards \PH{} and general games; is it still worth
keeping all these sources here in detail?}

\begin{itemize}

  \item \fullcite{language-instances}

    \begin{annotation}
      This paper introduces the simplest formal model for efficient puzzle
      generators.  A \emph{polynomial time constructor} (PTC) for a language
      \(L\) is a deterministic program that, on input \(1^n\), runs in
      polynomial-time and returns a string in \(L\) with length \(n\) iff one
      exists.  \emph{Polynomial time generators} (PTGs) is the nondeterministic
      analog of PTCs, with the additional requirement that every string in
      \(L\) of length \(n\) must be reachable.  PTCs and PTGs could be thought
      of as programs that produce solvable puzzles of given sizes (e.g., given
      \(n\), generate a solvable \(n^2 \times n^2\) Sudoku board).

      The main question explored here is: which classes of languages
      (puzzles/problems) have PTCs and PTGs?  Relevant results include:
      \begin{itemize}
        \item Every language that has a PTG is in \NP.
        \item For any language \(L\) in \NP, \(L\) has a PTC iff it has a PTG.
        \item Every \P{} language has a PTG iff every \NP{} language has a PTG.
      \end{itemize}
      Surprisingly enough, that last result indicates that we don't know
      whether every \P{} language has a PTG.  This paper goes on to define
      various special types of PTGs (e.g., categorical, lexicographical, etc.)
      and establishes various connections between PTG-existence questions and
      polynomial-hierarchy relations.
    \end{annotation}

  \item \fullcite{test-gen-complexity}

    \begin{annotation}
      In this paper, Sanchis generalizes the notion of puzzle generators
      introduced in \textcite{language-instances}.  Define a \emph{Test
      Instance Construction Method} (TICM) with respect to some fixed problem
      \(\Pi\) as a non-deterministic, polynomial-time program that, given a
      \emph{desired} answer \(\alpha\) (along with some desired parameters on
      the input, e.g., length), attempts to return an instance/input of \(\Pi\)
      that has answer \(\alpha\) and which meets the target parameters.

      The key result from this paper is that, unless \(\NP = \co\NP\), most
      \NP-hard problems do \emph{not} have efficient TICMs that can generate
      all input instances (with given known answers).  This establishes
      theoretical upper bounds on how comprehensive we can reasonably expect a
      puzzle generator to be in its coverage of available/possible inputs.
    \end{annotation}

  \item \fullcite{hard-diverse-graph-tests}

    \begin{annotation}
      In this paper, Sanchis continues to tackle the question, which problems
      have efficient TICMs?  This time, she further relaxes the desired TICM
      criterion---instead of looking for TICMs that can generate \emph{all}
      matching input instances, simply look for TICMs that generate a
      representative, or \emph{diverse}, set of inputs.  For a given problem
      optimization problem \(P\) and with respect to parameters (computable
      functions on \((\text{input}, \text{solution})\) pairs) \(l_1, \dots,
      l_k\), a set \(S\) of \((\text{input}, \text{solution})\) pairs is
      \emph{diverse} if every optimal pair in \(P\) has a corresponding pair in
      \(S\) with the same parameter values.

      The parameters \(l_1, \dotsc, l_k\) capture the way in which we can
      control properties of the generated puzzle instance.  Taking Sudoku as an
      example, \(l_i\) may be used to control the size of the board, the number
      of pre-filled squares, etc.  As another example, graph-based problems
      (e.g., traveling salesman, Hamiltonian path) may set \(l_i\) to be the
      number of vertices, edges, weights, etc. in the input graph.  In this
      sense, a diverse set contains representative puzzles for each
      (attainable) property combination; a diverse \emph{generator} is one
      capable of \emph{producing} puzzles for each such combination.

      This paper takes the TICM question in a concrete and exciting direction
      by \emph{constructing} efficient, hard (technical definition that roughly
      means ``doesn't only output trivially-solvable puzzles''), and diverse
      puzzle generators for three graph problems: minimum vertex cover,
      domination number, and chromatic number.
    \end{annotation}

  \item \fullcite{maximum-clique-generators}

    \begin{annotation}
      This paper, like \textcite{hard-diverse-graph-tests}, offers another
      deep-dive into a specific \NP-hard problem---the maximum clique problem
      (given an arbitrary graph, what is its largest complete subgraph?).  This
      paper describes several approximation algorithms for the maximum clique
      problem, as well as a ``hard and diverse'' algorithm generating test
      cases for any given number of vertices, edges, and maximum clique size.
    \end{annotation}

  \item \fullcite{stable-marriage-generation}

    \begin{annotation}
      This paper explores puzzle generation for the Stable Matching (SM)
      problem allowing for ties and incomplete preference lists.  The SM
      problem is stated as follows: given \(n\) people and \(n\) pets, and
      given a strictly-ordered preference list for each person and each pet,
      pair up people with pets so that no person and no pets simultaneously
      prefer each other to their currently-assigned partners; such a matching
      is called \emph{stable}.  The SM problem is solvable in quadratic time;
      the SMT variation, in which preference orderings are non-strict (i.e.,
      allowing ties), as well as the SMI variation, in which preferences lists
      may be incomplete (unspecified preferences are unacceptable; matchings
      may be partial), are also both solvable within quadratic time.  However,
      the \emph{SMTI} variation, in which both ties and incomplete lists are
      allowed, is NP-complete.

      This paper explores several proposed methods for generating SMTI puzzles,
      evaluating them by the criterion laid out in
      \textcite{test-gen-complexity}, observes theoretical shortcomings of each
      approach, and discusses connections between limitations in these
      algorithms to relationships between complexity-classes, e.g. \(\NP = \co\NP\).
    \end{annotation}

  \item \fullcite{random-latin-squares-sudoku}

    \begin{annotation}
      This paper describes relatively simple algorithms that randomly generate
      Sudoku boards and Latin Squares (\(n \times n\) grids in which each row
      and column contains each number \(1, \dotsc, n\)) with supposedly uniform
      probability.  There is not much theoretical discussion on complexity
      topics, but this paper discusses connections/reductions between
      Sudoku/Latin-Squares and the maximum clique problem, which in turn is
      examined at length from both theoretical and experimental perspectives in
      \textcite{maximum-clique-generators}.
    \end{annotation}

  \item \fullcite{strategy-solvable-sudoku}

    \begin{annotation}
      Computers generally solve Sudoku by reducing to \Problem{sat} or other
      well-optimized \NP-complete problem solvers, brute-force/backtracking,
      etc.  These approaches are quite different than the approaches taken by
      humans, which tend to entail various ``reduction'' rules, or
      \emph{strategies}.  An example of such a strategy is the \emph{naked
      singles} rule: for each given cell, start by writing out all the possible
      values (\(1, \dots, 9\)); eliminate from these possibilities values
      already taken by other cells in the same row/column/block; when only one
      possibility remains, finalize that cell with that value.

      While strategy-based approaches better represent how humans solve puzzles
      and are more intuitive, they lack the generality that \NP-complete puzzles
      usually require.  In particular, not all Sudoku puzzles are
      strategy-solvable.  The empty starting board, for instance, is not
      strategy-solvable because \emph{strategic} approaches are predicated on
      the assumption that the final solution is unique---which is not the case
      for an empty starting board.

      This paper details an algorithm (along with a framework for generalizing)
      for generating \emph{strategy-solvable} Sudoku puzzles.  The focus of
      this paper is applied, rather than pure---there is minimal emphasis on
      theoretical complexity or absolute coverage/generalizability.  Instead,
      the emphasis here is on generating puzzles that are solvable by these
      ``human-oriented'' strategies.  The algorithm proposed performs well on
      most grids but hits bottlenecks as the number of starting clues
      (pre-filled cells) shrinks to around \(20\).
    \end{annotation}

  \item \fullcite{conp-approximation}

    \begin{annotation}
      An set \(S\) of inputs \emph{approximates} a particular problem/language
      \(L\) if the \(S \subseteq L\).  Naturally, for a given approximation,
      another one is \emph{better} if it can solve infinitely many more
      problems in \(L\) (being able to solve finitely many more problems is
      does not constitute better, because any algorithm can always be finitely
      ``improved'' by hard-coding in additional solutions).

      \textcite{language-instances} and \textcite{test-gen-complexity}
      established several connections between puzzle generation and complexity
      theory---in particular, the relevant result: if a puzzle has a
      polynomial-time generator, it is in \NP.  As such, if all \co\NP{}
      problems have polynomial-time generators (PTGs), then \(\co\NP = \NP\).

      Previous work leverages that connection the other way: instead of trying
      to find perfect \(\co\NP\) PTGs (a doubtful pursuit, since it implies
      \(\co\NP = \NP\)), we can search for \emph{approximate} PTGs---those
      that, for a given problem \(L \in \co\NP\), generate not necessarily all
      strings in \(L\) but nevertheless large (ideally maximal) subsets of
      \(L\).  By doing so, the corresponding \NP{} algorithms \emph{induced by}
      those approximate PTGs are, themselves, approximation algorithms for
      \(L\).

      This paper focuses on the question of \emph{optimality} for such
      approximations and proves, assuming \(\NP \ne \co\NP\), a general
      condition/test for the optimality of an \(\NP\)-complete approximaton for
      \(\co\NP\) languages.
    \end{annotation}

  \item \fullcite{unique-sudoku-poly}

    \begin{annotation}
      To guarantee \emph{uniqueness} of a Sudoku solution, many generation
      algorithms take a trial-and-error approach---e.g., generate a puzzle,
      attempt to find multiple solutions, pre-fill additional cells as
      necessary to narrow the possibilities.  While this approach is more
      generalizable, it is not very performant (solving must be attempted after
      each revision to the board).

      This paper takes an alternate approach, summarized as follows: start with
      a fully-finished Sudoku board, then cleverly apply various reduction
      rules \emph{in reverse} (i.e., deleting the value in a cell when the
      surroundings fit a certain condition) in a manner that ensures uniqueness
      is preserved.

      Like \textcite{strategy-solvable-sudoku}, this paper does not delve into
      theoretical complexity topics and is geared towards more applied,
      ``ergonomics''-focused contexts.
    \end{annotation}

  \item \fullcite{difficulty-driven-sudoku}

    \begin{annotation}
      This paper, authored under the advice of our very own Prof. Dodds, takes
      a strategy-based approach to generating puzzles, its intentions
      reminiscent of \textcite{strategy-solvable-sudoku}.  The
      ``difficulty-driven'' generation algorithm proposed by this paper is,
      however, different: it is more similar to traditional trial-and-error
      (generate, try to solve, add/remove clues) Sudoku generation methods,
      with the key difference being that the verify/check solution step is done
      via strategies (as opposed to backtracking/brute-force) to mirror human
      solving capabilities.  In addition, this paper introduces notions of
      \emph{difficulty metrics} for Sudoku (e.g., ``most difficult required
      technique'') that may be used in the generation process to control/tune
      the difficulty-level of the resulting Sudoku board.
    \end{annotation}

  \item \fullcite{sudoku-difficulty-oracle}

    \begin{annotation}
      This paper, authored under the advice of our \emph{very} very own Prof.
      Jakes, also examines the problem of generating Sudoku boards with
      controlled difficulty.  Similarly to \textcite{difficulty-driven-sudoku},
      this paper attempts to model the difficulty of a Sudoku board in terms of
      the \emph{strategies} it uses.  More interestingly, it describes a more
      generalizable difficulty metric that abstracts specific choices of
      strategies into an \emph{oracle}, assessing difficulty in relation to
      number of oracle invocations, amount of work done by each oracle
      invocation, etc.
    \end{annotation}

  \item \fullcite{steiner-graph-generator}

    \begin{annotation}
      The Steiner tree problem in graphs is stated as follows: given a
      (non-negatively) weighted, undirected graph \(G\) and a subset \(V\) of
      its vertices, find a minimum-weight subtree of \(G\) connecting all of
      \(V\). (When \(\Abs V = 2\), this is a shortest-path problem; when \(V\)
      contains all vertices of \(G\), this is the minimum-spanning-tree
      problem. Surprisingly enough, while both those problems are
      polynomial-time-solvable, this problem is \NP-hard.)

      This paper proposes a (purportedly linear-time) algorithm to generate
      test cases for the Steiner tree problem in graphs.  The key technique
      involved in this algorithm is applying the so-called
      ``Karush-Kuhn-Tucker'' optimality conditions (which appear to be a sort
      of generalization of the Lagrange-multipliers method for solving an
      optimization-with-constraints problem).

      This paper does not contain much discussion on theoretical complexity
      topics but is instead valuable as a ``deep dive'' into a particular
      \NP-hard problem (and, perhaps, the broader category of
      combinatorial-optimization problems).
    \end{annotation}

  \item \fullcite{sudoku-education}

    \begin{annotation}
      This paper examines the problem of Sudoku puzzle generation in
      \emph{educational} contexts---in particular, not just generation
      algorithms but also exercises, classroom activities, etc. for students to
      do to explore the Sudoku generation problem.  Additionally, the
      mathematical concepts in this paper are meant to be approachable to
      high-schoolers (or even younger).  Given its educational/elementary
      focus, I don't expect to gather very deep/technical results from this
      paper, but I included it anyway because it's nevertheless adjacent to my
      topic, and I personally have tremendous interest in exploring innovative,
      inquiry-based mathematics teaching methods.
    \end{annotation}

\end{itemize}

\section{\PSPACE-complete games}

\todo[inline]{incomplete}

\begin{itemize}
  \item \fullcite{schaefer.deriving}

    \begin{annotation}
      In this paper, Schaefer gives a catalog of several \PSPACE-complete
      games, with the goal of finding an extensive list of \PSPACE-complete
      problems useful for showing other problems to be \PSPACE-complete,
      analogous to Karp's 21 \NP-complete problems.

      Furthermore, Schaefer argues that many \NP-complete problems have natural
      \PSPACE-complete counterparts, demonstrating on the Hamiltonian-path
      puzzle as example how the proof of \NP-completeness is translated into a
      proof of \PSPACE-completeness of a Hamiltonian-path game.
    \end{annotation}

  \item \fullcite{schaefer.games}

    \begin{annotation}
      Here, Schaefer gives another catalog (overlapping in parts with
      \textcite{schaefer.deriving}) of combinatorial games and poses a few open
      questions on certain games' complexities.
    \end{annotation}

  \item \fullcite{demaine.acgt}
  \item \fullcite{bodlaender.coloring}
  \item \fullcite{bh.placement}
  \item \fullcite{kbd.impartial}
  \item \fullcite{cpss.coloring}
\end{itemize}

\section{Fixed-turn games}

\todo[inline]{incomplete}

\begin{itemize}
  \item \fullcite{ddls.sat-variants-ph}

    \begin{annotation}
      This paper studies a few \emph{quantifications} of several \SAT{}
      variants, such as not-all-equal-\Problem{3sat}, etc., in relation to the
      polynomial hierarchy.  This provides a useful catalog (and useful example
      proofs) of other problems characterizing the polynomial hierarchy beyond
      the standard \(\Problem{qsat}_i\) problems.
    \end{annotation}

  \item \fullcite{stockmeyer.ph}

    \begin{annotation}

    \end{annotation}

  \item \fullcite{papadimitriou.cc}

    \begin{annotation}

    \end{annotation}

\end{itemize}




\printbibliography[heading=bibnumbered]

\end{document}
