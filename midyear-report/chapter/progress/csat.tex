\section{Boolean circuit games}

\todo[inline]{this section onward is untouched since our last meeting.  what
  remains to be filled out are: concrete examples of \SigmaP k-complete
  problems, i.e., satisfiability games; bits and pieces of our discussions on
  graph coloring games; a brief discussion of future work, in which I will
  gently mention the other branches I read about near the start of the
  semester, and summarize the main direction of open questions continuing the
current ``main'' thread of ideas.}

A Boolean circuit consists of a set of \emph{inputs}, which each take on a
value \(0\) or \(1\), and wired into a network of \emph{logic gates}, which
each take in one or two Boolean values (\(0\) or \(1\)) and output another
according to certain fixed rules.  Together, the logic gates compute some
Boolean function \(C\colon\Set{0,1}^m\to\Set{0,1}^n\).

\todo[inline]{example, also note that circuit should be acyclic}

A simple decision problem arising from Boolean circuits is known as the
\Problem{circuit-value} problem:
\begin{definition}[\Problem{circuit-value}]%
  Given an \(n\)-input circuit \(C\) together with fixed inputs \(v_1, v_2,
  \dotsc, v_n\), does the output of the circuit \(C(v_1,\dotsc,v_n)\) evaluate
  to \(1\)?
  \[
    \Problem{circuit-value} = \SetBuilder{
      (\text{Boolean circuit \(C\)}, \text{inputs \((v_1,\dotsc,v_n)\)})
      }{
      C(v_1, \dotsc, v_n) = 1
    }.
  \]

\end{definition}
This problem is fairly straightforward.  To solve it, start by attempting to
evaluate the right-most gate, whose output is the overall output of \(C\).  If
its inputs are not yet evaluated, recursively do so, \emph{marking} each gate
with its output value once it is computed, so that on repeated visits to the
same gate its output doesn't need to be recomputed.  Thus each gate in \(C\) is
evaluated at most once, taking \(\O(1)\) time, so evaluating the overall
circuit value takes \(\O(n)\) time.  Thus \(\Problem{circuit-value}\in\P\).

\subsection{The circuit satisfiability puzzle}

Given that \(\Problem{circuit-value}\in\P\), we can use it as a ``check''
problem from which to derive puzzles and games whose decision problems lie in
the polynomial hierarchy.  An input to the \Problem{circuit-value} decision
problem consists of two parts:
\begin{itemize}[nosep]
  \item a Boolean circuit \(C\), and
  \item assignments to \(C\)'s input variables, \(v_1, v_2, \dotsc, v_n\).
\end{itemize}
Now consider a puzzle in which the player is given only \(C\) (the game
``board''); their task is to \emph{guess} assignments to its inputs so that
\(C(\dotsc)\) evaluates to \(1\).  This puzzle is called the \emph{Circuit
Satisfiability} puzzle, or \CSAT{} for short:
\begin{definition}[\CSAT]%
  \label{def:csat}
  Given a Boolean circuit \(C\), does there exist an assignment to \(C\)'s
  inputs so that \(C\) evaluates to \(1\)?
  \begin{align*}
    \CSAT
    &= \SetBuilder{\text{Boolean circuit \(C\)}}{
      ∃(v_1,\dotsc,v_n) \quad C(v_1,\dotsc,v_n) = 1
    } \\
    &= \SetBuilder{C}{∃V \; (C,V)∈\Problem{circuit-value}}.
  \end{align*}
\end{definition}
It is straightforward to see, immediately following from the definition of
\NP{} (\cref{def:np}) and the fact that \(\Problem{circuit-value}∈\P\), that
\(\CSAT∈\NP\).

What makes \CSAT{} an especially interesting puzzle is its generality.  Boolean
circuits literally underpin the design of all modern computer architectures,
and consequently, any algorithm we can conceivably implement in some
programming language can, theoretically, be encoded as a Boolean circuit.  What
this fact implies for puzzles is that any \P{} problem can be reduced to the
Boolean \Problem{circuit-value} problem, and, in a similar manner, any \NP{}
problem (``puzzle'') \CSAT.  Consequently, \CSAT{} is \NP-complete
\parencite{cook.np}.  This fact is commonly known as the \emph{Cook--Levin}
theorem.
\begin{theorem}[Cook--Levin]
  \CSAT{} is \NP-complete \parencite{cook.np}.

  Technically, the actual Cook--Levin theorem concerns Boolean \emph{formula}
  satisfiability rather than \emph{circuit} satisfiability.  Nevertheless,
  since any formula can be represented as a circuit (every formula is a circuit
  where each gate's output is used only once), \NP-completeness of \CSAT{} is a
  straightforward corollary of the ``real'' Cook--Levin theorem.

  Alternatively, \textcite{ladner.cval} demonstrates a process of converting
  any algorithm (formalized as a Turing Machine) directly to a Boolean circuit.
  Consequently, I suspect (but haven't made sure) that a proof of \CSAT's
  \NP-completeness can be obtained via a simple generalization of this process.
\end{theorem}

\subsection{The circuit satisfiability games}

We now define the two-player circuit satisfiability \emph{games}.
\begin{definition}[\(\CSAT_k\)]%
  The game is played on a given circuit \(C\).  The inputs of \(C\) are
  partitioned into \(k\) (disjoint) groups \(V_1, \dotsc, V_k\).  The players
  alternate turns, assigning values to \emph{groups} of inputs at a time:
  \begin{enumerate}[nosep]
    \item Player 1 assigns values to all inputs in \(V_1\).
    \item Player 2 assigns values to all inputs in \(V_2\).
    \item Player 1 assigns values to all inputs in \(V_3\).
    \item[] \dots
    \item[{[\(k\)]}] Player 1 if \(k\) is odd, or player 2 if \(k\) is
      even, assigns values to inputs in \(V_k\).
  \end{enumerate}
  After all inputs have been assigned, the final output of the circuit
  determines the victor of the game.  In particular, player \(1\) wins if and
  only if \(C(V_1, \dotsc, V_k) = 1\).

  Define the decision problem \(\CSAT_k\) specifically as the question: does
  \emph{player 1} have a winning strategy?
  \begin{align*}
    \CSAT_k
    &= \SetBuilder* {(C, V_1, V_2, \dotsc, V_k)} {
      ∃V_1 \; ∀V_2 \; ∃V_3 \; \dotso \; \sfrac∃∀ \, V_k \quad
      C(V_1, V_2, \dotsc, V_k) = 1
    } \\
    &= \SetBuilder* {(C, V_1, V_2, \dotsc, V_k)} {
      ∃V_1 \; ∀V_2 \; ∃V_3 \; \dotso \; \sfrac∃∀ \, V_k \quad
      (C, (V_1, V_2, \dotsc, V_k)) ∈ \Problem{circuit-value}
    }.
  \end{align*}

  Note that \(\CSAT_1 = \CSAT\) is just the (one-player, one-turn) circuit
  satisfiability \emph{puzzle} (\cref{def:csat}).
\end{definition}

How difficult is \(\CSAT_k\)?  Again, it follows straightforwardly from its
definition that \(\CSAT_k ∈ \SigmaP k\).  More interesting is that, due to the
generality of Boolean circuits, each \(\CSAT_k\) is also \(\SigmaP k\)
complete:

\begin{theorem}
  For each \(k = 1, 2, 3, \dotsc\),
  \begin{itemize}[nosep]
    \item \(\CSAT_k\) is complete for \(\SigmaP k\), and
    \item (equivalently) \(\CSAT_k^c\) is complete for \(\PiP k\).
  \end{itemize}

  The version of this theorem in terms of identical games played on Boolean
  \emph{formulas} is proved in \textcite{wrathall.qsat}, but one can easily
  show the same holds for \emph{circuits} by a simple reduction converting
  formulas to circuits.
\end{theorem}


%\subsection{The Circuit Satisfiability puzzle}
%
%Our puzzles-and-games characterization of the polynomial hierarchy begins with
%a well-known family of problems generally referred to as Boolean Satisfiability
%problems.  Here is perhaps the simplest, most well-known Satisfiability puzzle:
%
%\begin{definition}[\SAT]%
%  Given a Boolean formula \(\phi(x_1, \dots, x_n)\), does there exist an
%  assignment of Boolean values to inputs \(x_1, \dots, x_n\) such that
%  \(\phi(x_1, \dots, x_n) = 1\)?  \Problem{sat} consists of the formula
%  instances for which the answer is \emph{yes}.
%
%  Formally:
%  \[
%    \Problem{sat} = \SetBuilder \phi {
%      \exists (x_1, \dots, x_n) \in \Set{0,1}^n \quad \phi(x_1, \dots, x_n) = 1
%    }.
%  \]
%\end{definition}
%
%The \SAT{} puzzle is particularly useful and worth studying because of its
%generality.  Booleans form the foundation of mathematical logic: every logical
%statement can be encoded, in some manner, as a Boolean formula.  Consequently,
%\SAT{} is, on an intuitive level, the most general possible puzzle---given any
%other puzzle, encoding its rules in terms of Booleans reveals that it is merely
%a special case of \SAT.  This idea is expressed formally as the Cook-Levin
%theorem:
%
%\begin{theorem}[Cook-Levin]
%  \SAT{} is \NP-complete.
%\end{theorem}
%
%%\begin{proof}
%%  \todo[inline]{put proof.  the most important reason to have the proof here is
%%  to illustrate}
%%\end{proof}
%
%\subsection{Satisfiability games}
%
%\begin{definition}[The two-turn \SAT{} games]%
%  The two-turn \SAT{} game is played on a Boolean formula \(\phi(x_1, \dots,
%  x_n, y_1, \dots, y_n)\) with inputs partitioned into two groups \(X =
%  \Set{x_i}\) and \(Y = \Set{y_i}\).  The two turns proceed as follows:
%  \begin{enumerate}
%    \item Player 1 assigns values to \(X\).
%    \item Player 2 assigns values to \(Y\).
%  \end{enumerate}
%  Player 1 wins if \(\phi\) is satisfied (\(\phi(\dots) = 1\)), and player 2
%  wins if \(\phi\) is falsified.
%
%  Who wins?  Two decision problems arise from this game:
%  \begin{itemize}
%    \item Does player 1 have a winning strategy?  That is, can player 1 make
%      some first move so that no matter what player 2 does, player 1 always
%      wins?
%
%    \item Does player 2 have a winning strategy?  That is, no matter what
%      player 1 plays, can player 2 respond with some move guaranteeing a win?
%  \end{itemize}
%
%\end{definition}
%
