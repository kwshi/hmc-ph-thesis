\section{Graph \(3\)-coloring games}

Having established the \SigmaP k/\PiP k-completeness of the circuit
satisfiability games, we now move on to explore another collection of games,
called the \emph{graph coloring games}.

Consider an undirected graph \(G\), in which each vertex is assigned a color
among \(\{0, 1, 2\}\).  The assignment of colors to the vertices of \(G\) is
called a \emph{(vertex) \(3\)-coloring} of \(G\).  We say a coloring is
\emph{proper} if every two adjacent vertices have different colors.

Given a graph \(G\) along with a \(3\)-coloring on \(G\), is the coloring
proper?  We can solve this problem by simply checking, for each edge, whether
the two vertices on that edge have different colors.  The run-time of this
solution is \(\O(e)\) and therefore polynomially-bounded in the size of \(G\).
Thus the problem of \emph{checking} whether a given \(3\)-coloring is proper is
in \P.

\subsection{The \(3\)-coloring puzzle}

The puzzle-ification of this problem comes in the following form:
\begin{definition}[\Problem{3col}]%
  Given a graph \(G\), is there a way to properly \(3\)-color the vertices of
  \(G\)?
  \[
    \Problem{3col} = \SetBuilder* G {
      ∃\,\text{coloring \(C = (c_1,\dotsc,c_n) ∈ \Set{0,1,2}^n\)} \quad \text{\(C\) is proper}
    }
  \]
\end{definition}
It is straightforward to see from its definition and the fact that
properness-checking is in \P{} that \(\Problem{3col} ∈ \NP\).

The natural question to ask is: is it also \NP-complete?  After all, earlier,
we could confidently expect that \NP-completeness from \emph{Boolean} \CSAT{}
because of the universality of Boolean logic, but, at a glance, it isn't
obvious that graphs and proper colorings are somehow ``fundamental'' to
computation as Booleans are.  But, in fact, that is exactly the case:
\begin{theorem}
  \Problem{3col} is \NP-complete.
\end{theorem}

An early proof of this result is given in \textcite{karp.np}, via a reduction
from a variant of the Boolean satisfiability puzzle known as
\Problem{cnf-3sat}.  Previously, I had come up with a more direct proof of this
result by directly reducing to \Problem{3col} from \CSAT, inspired by
\textcite{potapov.3col}, but I just realized a flaw in the proof that makes it
incorrect.  I will sketch my proof idea below anyway and discuss where it
fails.

%Here, we will
%sketch a slightly different proof that directly proves this result, leveraging
%\CSAT's \NP-completeness.

\begin{aside}
\begin{proof}
  We will show that \(\CSAT \le \Problem{3col}\) by giving a reduction
  (\cref{def:reduction}) from \CSAT{} to \Problem{3col}.  That is, given a
  \CSAT{} input, which is a circuit \(C\), we convert it into a graph \(G\) so
  that \(G\) is colorable if and only if \(C\) is satisfiable.

  First, construct a triangle \(t_0, t_1, t_2\) in \(G\).  For each \emph{wire}
  \(w\) (including the input wires) in \(C\), construct a corresponding vertex
  \(v(w)\) in \(G\), and connect that vertex to \(t_2\).  For each logic gate
  in \(C\), we implement the gate as a subgraph, which we call a \emph{gadget},
  in \(G\), as follows:
  \begin{description}

    \item[\NOT{} gate] Assume a \NOT{} gate takes as input a wire \(w\) and
      produces its output on a wire \(w'\).  Then, in \(G\), introduce an edge
      \(\Set{w, w'}\).

      \begin{center}
        \begin{tikzpicture}
          \matrix[column sep=1em, row sep=2em]{
            \coordinate[vertex](w); && \coordinate[vertex](w'); \\
            & \coordinate[vertex, opacity=1/3](t2); \\
          };
          \draw[edge] (w) -- (w');
          \draw[edge, opacity=1/3] (w) -- (t2) -- (w');
          \draw[edge, opacity=1/3] foreach \i in {1,...,4} {
            (w) -- +({180+15*(\i-5/2)}:2em)
            (w') -- +({15*(\i-5/2)}:2em)
          };
          \node[left, opacity=1/3] at ($ (w)+(-2em,0) $) {\(\dots\)};
          \node[right, opacity=1/3] at ($ (w')+(2em,0) $) {\(\dots\)};

          \node[vertex label, above] at (w.north) {\(v(w)\)};
          \node[vertex label, above] at (w'.north) {\(v(w')\)};
          \node[vertex label, below, opacity=1/3] at (t2.south) {\(t_2\)};
        \end{tikzpicture}
      \end{center}

    \item[\OR{} gate] Assume an \OR{} gate takes as input wires \(w_1, w_2\)
      and produces its output on a wire \(w'\).  Then, in \(G\), introduce two
      additional vertices \(v_1, v_2\) (note: \emph{not} \(v(w_1), v(w_2)\)),
      and the following edges:
      \[
        \Set{v(w_1), v_1}, \quad
        \Set{v(w_2), v_2}, \quad
        \Set{v_1, v_2}, \quad
        \Set{v_1, v'}, \quad
        \Set{v_2, v'}.
      \]

      \begin{center}
        \begin{tikzpicture}
          \matrix[column sep=3em, row sep=1em]{
            \coordinate[vertex](w1); & \coordinate[vertex](v1); \\
            && \coordinate[vertex](w'); \\
            \coordinate[vertex](w2); & \coordinate[vertex](v2); \\\\\\
            & \coordinate[vertex, opacity=1/3](t2); \\
          };
          \draw[edge] (w1) -- (v1) -- (w') -- (v2) -- (w2) (v1) -- (v2);
          \draw[edge, opacity=1/3] (w1) -- (t2) (w2) -- (t2) (w') -- (t2);
          \draw[edge, opacity=1/3] foreach \i in {1,...,4} {
            (w1) -- +({180+15*(\i-5/2)}:2em)
            (w2) -- +({180+15*(\i-5/2)}:2em)
            (w') -- +({15*(\i-5/2)}:2em)
          };
          \node[left, opacity=1/3] at ($ (w1)+(-2em,0) $) {\(\dots\)};
          \node[left, opacity=1/3] at ($ (w2)+(-2em,0) $) {\(\dots\)};
          \node[right, opacity=1/3] at ($ (w')+(2em,0) $) {\(\dots\)};

          \node[vertex label, above] at (w1.north) {\(v(w_1)\)};
          \node[vertex label, above] at (v1.north) {\(v_1\)};
          \node[vertex label, below] at (w2.south) {\(v(w_2)\)};
          \node[vertex label, below] at (v2.south) {\(v_2\)};
          \node[vertex label, above] at (w'.north) {\(v(w')\)};
          \node[vertex label, below, opacity=1/3] at (t2.south) {\(t_2\)};

        \end{tikzpicture}
      \end{center}

    \item[\AND{} gate] Assume an \AND{} gate takes as input wires \(w_1, w_2\)
      and produces its output on \(w'\).  We appeal to the fact that
      \[
        x \AND y = \NOT \Brack*{(\NOT x) \OR (\NOT y)},
      \]
      treating this \AND{} gate as though it were replaced by an equivalent
      network of \NOT{} and \OR{} gates, and therefore generate components in
      \(G\) accordingly.

  \end{description}
  Finally, connect the final output wire's corresponding vertex to \(t_0\).

  We claim that this construction ensures \(G\) is colorable if and only if
  \(C\) is satisfiable.
  \begin{itemize}

    \item[(\(⟹\))] \label{iff.3col.gc} First, suppose there exists a proper
      coloring on \(G\).  We wish to show that \(C\) is satisfiable.

      Without loss of generality, assume that the triangle \(t_0, t_1, t_2\) is
      colored as
      \[
        t_0 ↦ 0, \quad t_1 ↦ 1, \quad t_2 ↦ 2.
      \]
      By construction, each wire \(w\)'s corresponding vertex \(v(w)\) is
      connected to \(t_2\).  Since \(t_2\) has color \(2\), this ensures
      \(v(w)∈\Set{0,1}\) for each wire \(w\) in the circuit.  For each
      \emph{input wire} \(w\), assign the \emph{Boolean} value \(\Set{0,1}\)
      matching the color of \(v(w)\).  We claim that the resulting Boolean
      values on all other wires match their corresponding colorings.
      Consequently, because

      We show that this assignment of input values satisfies \(C\) by examining
      the colorings for each gate's ``gadget'':
      \begin{description}
      \item[\NOT] Consider an arbitrary \NOT{} gate with input \(w\) and output
        \(w'\).  As we noted above, both \(v(w)\) and \(v(w')\) must be colored
        either \(0\) or \(1\).  Then, since they neighbor each other, they must
        have opposite colors.  Thus \(v(w') = \NOT v(w)\), as desired.

      \item[\OR] Consider an arbitrary \OR{} gate with input \(w_1, w_2\) and
        output \(w'\).  Depending on the colors assigned to \(v(w_1), v(w_2)\),
        the following colorings on the \OR{} gadget are possible:
        \[
          \begin{array}{cc|c|c}
            v(w_1) & v(w_2) & (v_1, v_2) & v(w') \\ \midrule
            0 & 0 & \text{\((1, 2)\) or \((2, 1)\)} & 0 \\
            1 & 1 & \text{\((0, 2)\) or \((2, 0)\)} & 1 \\
            0 & 1 & \text{\((1, 2)\) or \((2, 0)\) } & \text{\(0\) or \(1\)} \\
            1 & 0 & \text{\((0, 2)\) or \((2, 1)\) } & \text{\(1\) or \(0\)}
          \end{array}
        \]
        We see that this doesn't imply that \(v(w')\) is \emph{necessarily}
        assigned the color corresponding to \(w_1 \OR w_2\), but rather that
        there \emph{exists} an assignment to \(v(w')\) equal to \(w_1 \OR
        w_2\).

        It turns out that this distinction is the Achilles' heel of this
        argument.  I'll explain why in detail at the end of the proof and
        continue for now in the interest of finishing up the proof sketch.
      \end{description}
      Finally, since the output wire's vertex neighbors both \(t_0\) and
      \(t_2\), it cannot be colored with either \(0\) or \(2\) and therefore
      must be colored \(1\).  Thus the circuit is also satisfied.

    \item[(\(⟸\))] \label{pf:3col.cg} Conversely, suppose \(C\) is satisfiable.
      We wish to show that \(G\) is colorable.

      Take a satisfying assignment on the inputs of \(C\), and color the
      corresponding vertices \(v(\dots)\) with matching colors in
      \(\Set{0,1}\).  As we demonstrated above, for each wire \(w\), the
      coloring \(v(w)\) with the color matching the Boolean value of \(w\)
      (filling in the rest appropriately) gives a proper coloring.  Thus \(G\)
      is colorable.  \qedhere

  \end{itemize}

  Back to the Achilles' heel.  The fact that the \OR{} gadget allows either
  color on the output when the two inputs differ implies that the coloring has
  ``extra freedoms'' that the corresponding circuit does not.  That is, certain
  proper colorings don't correspond to valid circuit states.  Per se, this
  isn't the issue: we are only concerned with that \emph{existence} of
  colorings/assignments is preserved, not that colorings correspond one-to-one
  with assignments.  Hence I was convinced of this proof anyway, until I
  realized, per the following example, that even \emph{existence} is not
  preserved.  Consider the following circuit \(C\):
  \begin{center}
    \begin{tikzpicture}[circuit logic US]

      \matrix[gates]{
        |(x)| && |[or gate](tx)| & |[not gate](f)| & %|[or gate](o)| & |[not gate](n)| &
        |(z)| \\
        & |[not gate](nx)| \\
        %|(y)| && |[or gate](ty)| \\
        %& |[not gate](ny)| \\
      };
      \draw[wire]
      (x) to (tx.input 1) (x) to (nx.input) (nx.output) to (tx.input 2)
      %(y) to (ty.input 1) (y) to (ny.input) (ny.output) to (ty.input 2)
      (tx.output) to (f.input)
      %(f.output) to (o.input 1) (ty.output) to (o.input 2)
      %(o.output) to (n.input)
      %(n.output) to (z)
      (f.output) to (z)
      ;
      \node[left] at (x) {\(x\)};
      %\node[left] at (y) {\(y\)};
      \node[right] at (z) {\(C(x,y)\) (output)};

    \end{tikzpicture}
  \end{center}
  This circuit computes the Boolean function
  \[
    C(x) = x \OR (\NOT x) = 1
  \]
  regardless of the value of \(x\).  Thus \(C\) is unsatisfiable.  However, now
  consider the graph \(G\) derived from \(C\), according to our construction
  above.  The excess freedom to color each \OR{} gate's output with either
  \(0\) or \(1\) enables a proper coloring, even when \(C\) is unsatisfiable:
  \begin{center}
    \begin{tikzpicture}

      \tikzset{
        pics/gadget/.style n args={3}{
          code={
            \coordinate(nw) at ($ (#2) + (-1em/2,1em/2) $);
            \coordinate(se) at ($ (#3) + (1em/2,-1em/2) $);
            \fill[fill opacity=1/8, rounded corners=1em/4]
            (nw) rectangle (se);
            \node[above, opacity=1/2] at ($ (nw)!1/2!(nw -| se) $) {#1};
          },
        },
      }

      \matrix[row sep=1em, column sep=3em, matrix of nodes]{
        && |[vertex, fill=ks1](x')| \\
        |[vertex, fill=ks0](x)| &&& |[vertex, fill=ks0](tx)| & |[vertex, fill=ks1](fx)| \\
        & |[vertex, fill=ks1](nx)| & |[vertex, fill=ks2](nx')| \\%&&& |[vertex](fx')| \\
        %&&&&&& |[vertex](o)| & |[vertex](n)| \\
        %&& |[vertex](y')| &&& |[vertex](ty')| \\
        %|[vertex, fill=ks0](y)| &&& |[vertex](ty)| \\
        %& |[vertex, fill=ks1](ny)| & |[vertex](ny')| \\[6em]
        %&&&& |[vertex, fill=ks2](t2)| && |[vertex, fill=ks0](t0)| \\
        %&&&&& |[vertex, fill=ks1](t1)| \\
        \\
        && |[vertex, fill=ks2](t2)| && |[vertex, fill=ks0](t0)| \\
        &&& |[vertex, fill=ks1](t1)| \\
      };

      \draw[edge]
      (x) -- (x') -- (tx) -- (fx) %-- (fx') -- (o) -- (n)
      (x) -- (nx) -- (nx') -- (tx)
      %(y) -- (y') -- (ty) -- (ty') -- (o)
      %(y) -- (ny) -- (ny') -- (ty)
      (x') -- (nx') %(y') -- (ny') (fx') -- (ty')
      ;

      \draw[edge, opacity=1/2, in=south, out=north, out looseness=0]
      %\draw[edge, opacity=1/4]
      %foreach \t in {x,y,nx,ny,tx,ty,fx,o,n} { (t2) to (\t) }
      foreach \t in {x,nx,tx,fx} { (t2) to (\t) }
      %(t0) to (n)
      (t0) to (fx)
      ;

      \draw[edge, opacity=1/2] (t0) -- (t2) -- (t1) -- (t0);

      \node[left] at (x.west) {\(x\)};
      %\node[left] at (y.west) {\(y\)};
      \node[below] at (t2.south) {\(t_2\)};
      \node[below] at (t1.south) {\(t_1\)};
      \node[below] at (t0.south) {\(t_0\)};
      %\node[right] at (n.east) {\(C(x)\) (output)};
      \node[right] at (fx.east) {\(C(x)\) (output)};

      \pic{gadget={\NOT}{nx}{nx}};
      %\pic{gadget={\NOT}{ny}{ny}};
      \pic{gadget={\OR}{x'}{nx'-|tx}};
      %\pic{gadget={\OR}{y'}{ny'-|ty}};
      \pic{gadget={\NOT}{fx}{fx}};
      %\pic{gadget={\OR}{fx'}{ty'-|o}};
      %\pic{gadget={\NOT}{n}{n}};

    \end{tikzpicture}
  \end{center}

  Thus our proof fails.  Yet an almost identical construction is used in the
  original proof \parencite{karp.np,potapov.3col}---why does that \emph{not}
  fail?  That's because the original proof uses \Problem{cnfsat}
  (conjunctive-normal-form Boolean formula satisfiability), rather than \CSAT{}
  as we attempt here.  In particular, the ``shallow'' structure of
  \Problem{cnfsat} formulas mitigates the ``\OR-gates have too much freedom''
  problem we encounter here, thereby ensuring that coloring values \emph{do}
  exactly match the actual \OR-gate output.

\end{proof}
\end{aside}

So, unfortunately, while it \emph{is} true that \Problem{3col} is \NP-complete,
we don't have a direct reduction from \CSAT{} demonstrating this result.  That
will have to wait until next semester, it seems\dots

\subsection{Two-player \(3\)-coloring games}

Next, we attempt to define \emph{game} generalizations of the graph
\(3\)-coloring problem.  So far, several alternatives have been considered:
\begin{itemize}

  \item A first attempt at defining such a game might be: partition vertices
    into \(k\) groups \(V_1,\dotsc,V_k\); two players alternate turns assigning
    colors to groups at a time.  Victory is decided by whether the finalized
    coloring is proper.

    Unfortunately, this game might be trivial for the ``adversarial'' player,
    whose goal is to falsify/improper-ify the coloring.  On the \(i\)-th turn,
    as long as any vertex \(v∈V_i\) neighbors any other vertex in
    \(v'∈V_{≤i}\), this player can easily sabotage the coloring by matching the
    color of \(v\) to \(v'\).  The only way to ensure that this doesn't occur,
    then, is to ensure, in each game instance, that each of the adversary's
    vertex groups do not neighbor any previous groups' vertices \emph{or each
    other}. But if the adversary's vertices are totally disjoint from the rest
    of the graph (and each other), then, once again, it is trivially impossible
    for the adversary to influence the coloring's properness at all, and the
    ``solver'' trivially wins.

    In order to avoid both of these trivial situations, then, the following
    must hold:
    \begin{itemize}[nosep]
      \item Each of the adversary's vertices must not neighbor each other or
        any vertex from previous groups.  At the same time, the adversary's
        vertices must be \emph{connected} to other vertices in the graph.
        (Those that aren't can be trivially ignored.)
      \item Because the adversary's vertices are connected to other vertices,
        and yet cannot neighbor any of them from the same or previous turns,
        the last player must be the \emph{solver}, who has control over
        vertices in the ``boundaries'' between adversary's vertices and other
        previously-colored vertices.
    \end{itemize}
    These conditions seem to significantly narrow the the kinds of graphs (and
    vertex partitionings) whose games are interesting.  Additionally, still
    more ``trivial'' attacks by the adversary might be possible---e.g.,
    coloring three vertices who share a common neighbor with three different
    colors.  So far, my attempts to convert the \CSAT{} games to this have
    suffered from these problems, and I am not sure whether these games have
    the desired completeness properties at all.

  \item Instead of restricting the graphs on which interesting games can be
    played, we can also try imposing a more aggressive restriction on the
    adversary's abilities.  For instance, we can require that adversary never
    makes any ``trivial'' falsifying-assignments, i.e., ones that immediately
    contradict previously-colored vertices.  If they do so, they automatically
    lose, and the solver automatically wins.  We can build this condition into
    the ``check'' problem while still preserving its membership in \P, thereby
    ensuring this game remains within the polynomial hierarchy.

    However, the additional special treatment of the adversary here throws off
    the apparent \emph{symmetry} between \SigmaP k and \PiP k (or, ordering of
    \(∃\) vs. \(∀\) quantifiers in the winning-strategy conditions), making
    this game somewhat less theoretically elegant.

    Given the flaw I discovered in my \(\CSAT\le\Problem{3col}\) proof earlier,
    I haven't yet been able to verify that this version of the game has the
    desired completeness properties either, though it seems more promising, if
    the \(\CSAT\le\Problem{3col}\) flaw is fixed, that the proof will be
    straightforwardly adaptible to this game.

  \item A third option is to also restrict the colors available to the
    adversary.  I haven't thought very deeply about this option; based on my
    attempts to reduce from \CSAT, this also seems plausible/promising, but
    because of its special treatment against the adversary, is also an
    aesthetic compromise.

\end{itemize}

We will have to see, next semester, whether ``nice'', theoretically simple,
polynomial-hierarchy versions of the graph \(3\)-coloring game exist.  And, of
course, we will have to prove that they are, indeed, \emph{interesting}.



%\begin{definition}[Two-turn \Problem{3col}]%
%  The two-turn \Problem{3col} game is played on a graph \(\Gamma\) whose
%  vertices are partitioned into two (disjoint) groups \(X\) and \(Y\).  Two
%  players take turns assigning one of three colors to vertices.  First, player
%  1 colors vertices in \(X\); second, player 2 colors vertices in \(Y\).
%  Player 1 wins if the resulting coloring is \emph{invalid}---that is, there
%  exists a pair of vertices sharing the same color; player 2 wins if the
%  resulting coloring is valid.
%
%\end{definition}
%
%\begin{conjecture}
%  Two-turn \Problem{3col} is complete for \SigmaP2 (or \PiP2, depending on
%  which player's winning strategy we examine).
%\end{conjecture}


