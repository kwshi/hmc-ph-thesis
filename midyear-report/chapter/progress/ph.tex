\section{Multi-turn games and the polynomial hierarchy}

We think of \NP{} problems as one-turn games (``puzzles'') and \SigmaP2
problems as two-turn games.  If we go backwards by a step, and imagine games
with \emph{zero} turns (i.e., neither player moves; the input automatically
determines who wins), we simply obtain \P.

It is natural for us to now extend this reasoning to \(k\)-turn games.  If two
players alternate turns making moves \(g_1, g_2, \dotsc, g_k\), with victory
decided by \((x, g_1, g_2, \dotsc, g_k) \overset{?}{\in} L'\) (with
\(\L'\in\P\)), does the starting player have a winning strategy?  For each
\(k\), this question and its complement (whether the second player has a
winning strategy) define classes \SigmaP{k} and \PiP{k}.  This chain of
complexity classes is known as the \emph{polynomial hierarchy}:

\begin{definition}[polynomial hierarchy]%
  The \emph{polynomial hierarchy} refers to the collection of complexity
  classes
  \begin{align*}
    \SigmaP0 = \P, \\
    \SigmaP1 = \NP &= \SetBuilder{L}{
      ∃\mathstrut L'∈\P \; ∀x \quad
      x∈L ⟺ ∃g \; (x, g)∈L'
    }, \\
    \SigmaP2 &= \SetBuilder* L {
      ∃ L'∈\P \; ∀x \quad
      x∈L ⟺ ∃g_1 \; ∀g_2 \; (x, g_1, g_2)∈L'
    }, \\
    \SigmaP3 &= \SetBuilder* L {
      ∃ L'∈\P \; ∀x \quad
      x∈L ⟺ ∃g_1 \; ∀g_2 \; ∃ g_3 \; (x, g_1, g_2, g_3)∈L'
    }, \\
    &\vdotswithin{=} \\
    \SigmaP k &= \SetBuilder* L {
      ∃ L'∈\P \; ∀x \quad
      x∈L ⟺ ∃g_1 \; ∀g_2 \; \dotsb \sfrac∃∀\, g_k
      \; (x, g_1, g_2, \dotsc, g_k)∈L'
    },
  \end{align*}
  along with their complement classes
  \begin{align*}
    \PiP0 = \co\SigmaP0 = \co\P &= \P, \\
    \PiP1 = \co\SigmaP1 = \co\NP &= \SetBuilder*{L}{
      ∃ \mathstrut L'∈\P \; ∀x \quad
      x∈L ⟺ ∀g \; (x, g) ∉ L'
    }, \\
    \PiP2 = \co\SigmaP2 &= \SetBuilder*{L}{
      ∃ \mathstrut L'∈\P \; ∀x \quad
      x∈L ⟺ ∀g_1 \; ∃g_2 \; (x, g_1, g_2) ∉ L'
    }, \\
    &\vdotswithin{=} \\
    \PiP k = \co\SigmaP k &= \SetBuilder*{L}{
      ∃ \mathstrut L'∈\P \; ∀x \quad
      x∈L ⟺ ∀g_1 \; ∃g_2 \dotsb \sfrac∀∃ \, g_k
      \; (x, g_1, g_2, \dotsc, g_k) ∉ L'
    }.
  \end{align*}
  Also note, just as in \cref{def:np}, that each of the guesses \(g_1, \dotsc,
  g_k\) must be bounded in length by polynomial functions of \(\Abs x\); we
  have omitted these requirements from the definitions stated above only for
  sake of brevity.  Again, these conditions are not central to the intuition of
  these definitions but an important technicality.

  As an aside, observe that \(∉ L'\) means the same thing as \(∈ L'^c\)
  and that \(L'\in\P\) if and only if \(L'^c\in\P\) as well (since, in general,
  \(\co\P=\P\)).  Thus we can equivalently define \(\PiP k\) without negation,
  as
  \[
    \PiP k = \SetBuilder*{L}{
      ∃\mathstrut L' ∈ \P \; ∀x \quad
      x ∈ L ⟺ ∀g_1 \; ∃g_2 \dotsb \sfrac∀∃ \, g_k
      \; (x, g_1, g_2, \dotsc, g_k)
      \smash[b]{\underset{\substack{\uparrow\\\mathclap{\text{not negated}}}}{{}∈{}}} L'
    }.
    \vphantom{\underset{\substack{\uparrow\\\text{not negated}}}{∈}}
  \]

  Note that this definition differs in presentation from
  \textcite{papadimitriou.cc}.  Here, we formulate these classes explicitly
  from the perspective of games.  \textcite[Definition 17.2]{papadimitriou.cc}
  defines an additional chain of classes named \DeltaP{k} and does so
  recursively via a concept called \emph{oracles}.  While also insightful and
  interesting, those ideas are somewhat less pertinent to our games-and-puzzles
  treatment, so for now we will set them aside.
\end{definition}

How complex is each class in the polynomial hierarchy?

First, we can observe that any game with \(k\) turns can also be played as a
\(k+1\)-turn game, in which the last (or first) player's move is ignored in
deciding the outcome of the game.  In other words, any \(k\)-turn game is
\emph{reducible} (\cref{def:reduction}) to a \(k+1\)-turn game, and therefore
\[
  \SigmaP k, \PiP k ⊆ \SigmaP{k+1}, \PiP{k+1}.
\]
Thus, in general, \(k\)-turn games are at least as easy as \(k+1\)-turn games.
But are they strictly easier?  Similar to \P-vs-\NP, this is an open question:
it isn't known whether any of these containments are strict, though many
suspect they are.  Relatedly, it is also not known, but suspected to be the
case, whether \(\SigmaP k \ne \PiP k\) at all levels \(k\ge1\).  The
containment relations in the polynomial hierarchy may be visualized as follows:

\begin{center}
  \begin{tikzpicture}
    \tikzset{
      subset/.style={
        ->,
        draw opacity=1/2,
        out looseness=.5,
      },
      unknown/.style={
        draw opacity=1/2,
        text opacity=1/2,
        densely dotted,
      },
      hierarchy/.style={
        row sep=1em, column sep=4em, matrix of math nodes,
        nodes={
          anchor=base,
          text height=2em/3,
          text depth=1em/3,
        },
      },
    }

    \matrix[hierarchy]{
      & |(Σ1)|\SigmaP1 = \NP & |(Σ2)|\SigmaP2 & |(Σ3)|\SigmaP3 & |(Σ)|\dotso \\
      |(0)| \SigmaP0 = \PiP0 = \P \\
      & |(Π1)|\PiP1 = \co\NP & |(Π2)|\PiP2 & |(Π3)|\PiP3 & |(Π)|\dotso \\
    };

    \draw[subset, out=+45, in=west] (0) to (Σ1);
    \draw[subset, out=-45, in=west] (0) to (Π1);
    \foreach \i/\j in {1/2,2/3,3/} {
      \draw[subset] (Σ\i) -- (Σ\j);
      \draw[subset] (Π\i) -- (Π\j);
      \draw[subset] (Σ\i) -- (Π\j);
      \draw[subset] (Π\i) -- (Σ\j);
      \draw[unknown] (Σ\i) -- (Π\i) node[fill=white, midway]{?};
    }
  \end{tikzpicture}
\end{center}

To concretely develop a sense of how difficult these classes are, we shall
search for good ``representative'' problems from each class.  In particular,
what are some \emph{complete} (\cref{def:hard-complete}) problems for each
class in the polynomial hierarchy?


