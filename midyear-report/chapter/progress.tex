\chapter{Current progress}

\label{ch:progress}

\section{Puzzle generation}

\label{sec:progress.generation}

The topic of puzzle generation difficulty has been explored in detail by Laura
Sanchis
\parencite{language-instances,test-gen-complexity,hard-diverse-graph-tests}.

\todo[inline]{incomplete; i'm focusing my effort on the fixed-turn section first
because that's more novel/interesting}

\subsection{Puzzles with unique solutions}

In many popularly-known puzzle games, one criterion for ``good'' puzzle
generation is that the generated puzzle instance should have a unique solution.
For example, given a \(9\times9\) Sudoku grid (partially pre-filled with
numbers \(1,\dotsc,9\)), there should be \emph{exactly one} way to complete the
grid---no more, no less.

This formulation is incompatible with our

\section{\PSPACE-complete games}

\label{sec:progress.pspace}

\todo[inline]{incomplete}

\section{Fixed-turn games}

\label{sec:progress.ph}

\subsection{Oracles, and the polynomial hierarchy}

\todo[inline]{expository definitions on oracles and the polynomial hierarchy}

\subsection{Boolean Satisfiability}

\subsubsection{The Satisfiability puzzle}

\todo{context on what booleans are?}

Our puzzles-and-games characterization of the polynomial hierarchy begins with
a well-known family of problems generally referred to as Boolean Satisfiability
problems.  Here is perhaps the simplest, most well-known Satisfiability puzzle:

\begin{definition}[\Problem{sat}]%
  Given a Boolean formula \(\phi(x_1, \dots, x_n)\), does there exist an
  assignment of Boolean values to inputs \(x_1, \dots, x_n\) such that
  \(\phi(x_1, \dots, x_n) = 1\)?  \Problem{sat} consists of the formula
  instances for which the answer is \emph{yes}.

  Formally:
  \[
    \Problem{sat} = \SetBuilder \phi {
      \exists (x_1, \dots, x_n) \in \Set{0,1}^n \quad \phi(x_1, \dots, x_n) = 1
    }.
  \]
\end{definition}

The \SAT{} puzzle is particularly useful and worth studying because of its
generality.  Booleans form the foundation of mathematical logic: every logical
statement can be encoded, in some manner, as a Boolean formula.  Consequently,
\SAT{} is, on an intuitive level, the most general possible puzzle---given any
other puzzle, encoding its rules in terms of Booleans reveals that it is merely
a special case of \SAT.  This idea is expressed formally as the Cook-Levin
theorem:

\begin{theorem}[Cook-Levin]
  \SAT{} is \NP-complete.
\end{theorem}

%\begin{proof}
%  \todo[inline]{put proof.  the most important reason to have the proof here is
%  to illustrate}
%\end{proof}

\subsubsection{Satisfiability games}

\begin{definition}[The two-turn \SAT{} games]%
  The two-turn \SAT{} game is played on a Boolean formula \(\phi(x_1, \dots,
  x_n, y_1, \dots, y_n)\) with inputs partitioned into two groups \(X =
  \Set{x_i}\) and \(Y = \Set{y_i}\).  The two turns proceed as follows:
  \begin{enumerate}
    \item Player 1 assigns values to \(X\).
    \item Player 2 assigns values to \(Y\).
  \end{enumerate}
  Player 1 wins if \(\phi\) is satisfied (\(\phi(\dots) = 1\)), and player 2
  wins if \(\phi\) is falsified.

  Who wins?  Two decision problems arise from this game:
  \begin{itemize}
    \item Does player 1 have a winning strategy?  That is, can player 1 make
      some first move so that no matter what player 2 does, player 1 always
      wins?

    \item Does player 2 have a winning strategy?  That is, no matter what
      player 1 plays, can player 2 respond with some move guaranteeing a win?
  \end{itemize}

\end{definition}


\subsection{Graph coloring}

\begin{definition}[\Problem{3col}]%
  Given a graph \(\Gamma\), is there a way to color each vertex in  \(\Gamma\)
  with one of three colors so that every pair of adjacent vertices has distinct
  colors?
\end{definition}

\begin{theorem}
  \Problem{3col} is \NP-complete.
\end{theorem}

\subsubsection{Graph coloring games}

\begin{definition}[Two-turn \Problem{3col}]%
  The two-turn \Problem{3col} game is played on a graph \(\Gamma\) whose
  vertices are partitioned into two (disjoint) groups \(X\) and \(Y\).  Two
  players take turns assigning one of three colors to vertices.  First, player
  1 colors vertices in \(X\); second, player 2 colors  vertices in \(Y\).
  Player 1 wins if the resulting coloring is \emph{invalid}---that is, there
  exists a pair of vertices sharing the same color; player 2 wins if the
  resulting coloring is valid.

\end{definition}

\begin{conjecture}
  Two-turn \Problem{3col} is complete for \SigmaP2 (or \PiP2, depending on
  which player's winning strategy we examine).
\end{conjecture}
