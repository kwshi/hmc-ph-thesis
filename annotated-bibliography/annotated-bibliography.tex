\documentclass{extarticle}

\usepackage{geometry}
\usepackage{fancyhdr}
\usepackage{titling}
\usepackage{lastpage}
\geometry{
  top=.25in,
  bottom=.25in,
  left=1in,
  right=1in,
  includeheadfoot,
  headsep=1em,
  headheight=1em,
}
\fancypagestyle{annotated-bibliography}{
  \fancyhf{}
  \lhead{\bfseries Math Thesis: Annotated Bibliography}
  \rhead{Kye Shi}
  \rfoot{\thepage{} / \pageref{LastPage}}
}
\pagestyle{annotated-bibliography}

\usepackage{fontspec}
\usepackage{mathtools}
\usepackage{libertinus}
\usepackage{microtype}
\usepackage{parskip}
\renewcommand\familydefault\sfdefault

\setmonofont{Fira Code}[Scale=MatchLowercase]

\usepackage{hyperref}
\usepackage{url}

\usepackage[style=authortitle]{biblatex}
\addbibresource{bibliography.bib}

\usepackage{tcolorbox}
\tcbuselibrary{breakable}
\newtcolorbox{annotation}{
  breakable,
  sharp corners,
  opacityframe=0,
  opacityback=0,
  boxsep=0pt,
  right=0em,
  left=2em,
  parbox=false,
}

\begin{document}

The topic of my thesis is \emph{puzzle generation}.  When we talk about
complexity problems \& algorithms, we mostly focus on \emph{solving} problems
(i.e., ``given so-and-so setup, find so-and-so satisfying some conditions'').
But what does it take to \emph{make} a good puzzle?  How hard is it to generate
a good Sudoku puzzle?  What does \emph{good} even mean?  Are there general ways
to generate arbitrary, NP-complete/NP-hard ``flavored'' puzzles?

\paragraph{Annotated bibliography}

\begin{itemize}

  \item \fullcite{language-instances}

    \begin{annotation}
      This paper introduces the simplest formal model for efficient puzzle
      generators.  A \emph{polynomial time constructor} (PTC) for a language
      \(L\) is a deterministic program that, on input \(1^n\), runs in
      polynomial-time and returns a string in \(L\) with length \(n\) iff one
      exists.  \emph{Polynomial time generators} (PTGs) is the nondeterministic
      analog of PTCs, with the additional requirement that every string in
      \(L\) of length \(n\) must be reachable.  PTCs and PTGs could be thought
      of as programs that produce solvable puzzles of given sizes (e.g., given
      \(n\), generate a solvable \(n^2 \times n^2\) Sudoku board).

      The main question explored here is: which classes of languages
      (puzzles/problems) have PTCs and PTGs?  Relevant results include:
      \begin{itemize}
        \item Every language that has a PTG is in NP.
        \item For any language \(L\) in NP, \(L\) has a PTC iff it has a PTG.
        \item Every P language has a PTG iff every NP language has a PTG.
      \end{itemize}
      Surprisingly enough, that last result indicates that we don't know
      whether every P language has a PTG.  This paper goes on to define various
      special types of PTGs (e.g., categorical, lexicographical, etc.) and
      establishes various connections between PTG-existence questions and
      polynomial-hierarchy relations.
    \end{annotation}

  \item \fullcite{test-gen-complexity}

    \begin{annotation}
      In this paper, Sanchis generalizes the notion of puzzle generators
      introduced in \textcite{language-instances}.  Define a \emph{Test
      Instance Construction Method} (TICM) with respect to some fixed problem
      \(\Pi\) as a non-deterministic, polynomial-time program that, given a
      \emph{desired} answer \(\alpha\) (along with some desired parameters on
      the input, e.g., length), attempts to return an instance/input of \(\Pi\)
      that has answer \(\alpha\) and which meets the target parameters.

      The key result from this paper is that, unless NP = co-NP, most NP-hard
      problems do \emph{not} have efficient TICMs that can generate all input
      instances (with given known answers).  This establishes theoretical upper
      bounds on how comprehensive we can reasonably expect a puzzle generator
      to be in its coverage of available/possible inputs.
    \end{annotation}

  \item \fullcite{hard-diverse-graph-tests}

    \begin{annotation}
      In this paper, Sanchis continues to tackle the question, which problems
      have efficient TICMs?  This time, she further relaxes the desired TICM
      criterion---instead of looking for TICMs that can generate \emph{all}
      matching input instances, simply look for TICMs that generate a
      representative, or \emph{diverse}, set of inputs.  For a given problem
      optimization problem \(P\) and with respect to parameters (computable
      functions on \((\text{input}, \text{solution})\) pairs) \(l_1, \dots,
      l_k\), a set \(S\) of \((\text{input}, \text{solution})\) pairs is
      \emph{diverse} if every optimal pair in \(P\) has a corresponding pair in
      \(S\) with the same parameter values.

      The parameters \(l_1, \dotsc, l_k\) capture the way in which we can
      control properties of the generated puzzle instance.  Taking Sudoku as an
      example, \(l_i\) may be used to control the size of the board, the number
      of pre-filled squares, etc.  As another example, graph-based problems
      (e.g., traveling salesman, Hamiltonian path) may set \(l_i\) to be the
      number of vertices, edges, weights, etc. in the input graph.  In this
      sense, a diverse set contains representative puzzles for each
      (attainable) property combination; a diverse \emph{generator} is one
      capable of \emph{producing} puzzles for each such combination.

      This paper takes the TICM question in a concrete and exciting direction
      by \emph{constructing} efficient, hard (technical definition that roughly
      means ``doesn't only output trivially-solvable puzzles''), and diverse
      puzzle generators for three graph problems: minimum vertex cover,
      domination number, and chromatic number.
    \end{annotation}

  \item \fullcite{stable-marriage-generation}

    \begin{annotation}
      This paper explores puzzle generation for the Stable Matching (SM)
      problem allowing for ties and incomplete preference lists.  The SM
      problem is stated as follows: given \(n\) people and \(n\) pets, and
      given a strictly-ordered preference list for each person and each pet,
      pair up people with pets so that no person and no pets simultaneously
      prefer each other to their currently-assigned partners; such a matching
      is called \emph{stable}.  The SM problem is solvable in quadratic time;
      the SMT variation, in which preference orderings are non-strict (i.e.,
      allowing ties), as well as the SMI variation, in which preferences lists
      may be incomplete (unspecified preferences are unacceptable; matchings
      may be partial), are also both solvable within quadratic time.  However,
      the \emph{SMTI} variation, in which both ties and incomplete lists are
      allowed, is NP-complete.

      This paper explores several proposed methods for generating SMTI puzzles,
      evaluating them by the criterion laid out in
      \textcite{test-gen-complexity}, observes theoretical shortcomings of each
      approach, and discusses connections between limitations in these
      algorithms to relationships between complexity-classes, e.g. NP = co-NP.
    \end{annotation}

  \item \fullcite{random-latin-squares-sudoku}

    \begin{annotation}
      This paper describes relatively simple algorithms that randomly generate
      Sudoku boards and Latin Squares (\(n \times n\) grids in which each row
      and column contains each number \(1, \dotsc, n\)) with supposedly uniform
      probability.  There is not much theoretical discussion on complexity
      topics, but this paper discusses connections/reductions between
      Sudoku/Latin-Squares and the maximum clique problem, which in turn is
      examined at length from both theoretical and experimental perspectives in
      \textcite{maximum-clique-generators}.
    \end{annotation}

  \item \fullcite{strategy-solvable-sudoku}

    \begin{annotation}
      Computers generally solve Sudoku by reducing to SAT or other
      well-optimized NP-complete problem solvers, brute-force/backtracking,
      etc.  These approaches are quite different than the approaches taken by
      humans, which tend to entail various ``reduction'' rules, or
      \emph{strategies}.  An example of such a strategy is the \emph{naked
      singles} rule: for each given cell, start by writing out all the possible
      values (\(1, \dots, 9\)); eliminate from these possibilities values
      already taken by other cells in the same row/column/block; when only one
      possibility remains, finalize that cell with that value.

      While strategy-based approaches better represent how humans solve puzzles
      and are more intuitive, they lack the generality that NP-complete puzzles
      usually require.  In particular, not all Sudoku puzzles are
      strategy-solvable.  The empty starting board, for instance, is not
      strategy-solvable because \emph{strategic} approaches are predicated on
      the assumption that the final solution is unique---which is not the case
      for an empty starting board.

      This paper details an algorithm (along with a framework for generalizing)
      for generating \emph{strategy-solvable} Sudoku puzzles.  The focus of
      this paper is applied, rather than pure---there is minimal emphasis on
      theoretical complexity or absolute coverage/generalizability.  Instead,
      the emphasis here is on generating puzzles that are solvable by these
      ``human-oriented'' strategies.  The algorithm proposed performs well on
      most grids but hits bottlenecks as the number of starting clues
      (pre-filled cells) shrinks to around \(20\).
    \end{annotation}

  \item \fullcite{conp-approximation}

    \begin{annotation}
      An set \(S\) of inputs \emph{approximates} a particular problem/language
      \(L\) if the \(S \subseteq L\).  Naturally, for a given approximation,
      another one is \emph{better} if it can solve infinitely many more
      problems in \(L\) (being able to solve finitely many more problems is
      does not constitute better, because any algorithm can always be finitely
      ``improved'' by hard-coding in additional solutions).

      \textcite{language-instances} and \textcite{test-gen-complexity}
      established several connections between puzzle generation and complexity
      theory---in particular, the relevant result: if a puzzle has a
      polynomial-time generator, it is in NP.  As such, if all co-NP problems
      have polynomial-time generators (PTGs), then co-NP = NP.

      Previous work leverages that connection the other way: instead of trying
      to find perfect co-NP PTGs (a doubtful pursuit, since it implies co-NP =
      NP), we can search for \emph{approximate} PTGs---those that, for a given
      problem \(L\) in co-NP, generate not necessarily all strings in \(L\) but
      nevertheless large (ideally maximal) subsets of \(L\).  By doing so, the
      corresponding NP algorithms \emph{induced by} those approximate PTGs are,
      themselves, approximation algorithms for \(L\).

      This paper focuses on the question of \emph{optimality} for such
      approximations and proves, assuming NP \(\ne\) co-NP, a general
      condition/test for the optimality of an NP-complete approximaton for
      co-NP languages.
    \end{annotation}

  \item \fullcite{unique-sudoku-poly}

    \begin{annotation}
      To guarantee \emph{uniqueness} of a Sudoku solution, many generation
      algorithms take a trial-and-error approach---e.g., generate a puzzle,
      attempt to find multiple solutions, pre-fill additional cells as
      necessary to narrow the possibilities.  While this approach is more
      generalizable, it is not very performant (solving must be attempted after
      each revision to the board).

      This paper takes an alternate approach, summarized as follows: start with
      a fully-finished Sudoku board, then cleverly apply various reduction
      rules \emph{in reverse} (i.e., deleting the value in a cell when the
      surroundings fit a certain condition) in a manner that ensures uniqueness
      is preserved.

      Like \textcite{strategy-solvable-sudoku}, this paper does not delve into
      theoretical complexity topics and is geared towards more applied,
      ``ergonomics''-focused contexts.
    \end{annotation}

  \item \fullcite{difficulty-driven-sudoku}



  \item \fullcite{sudoku-difficulty-oracle}

  \item \fullcite{steiner-graph-generator}

    \begin{annotation}
      The Steiner tree problem in graphs is stated as follows: given a
      (non-negatively) weighted, undirected graph \(G\) and a subset \(V\) of
      its vertices, find a minimum-weight subtree of \(G\) connecting all of
      \(V\). (When \(\lvert V \rvert = 2\), this is a shortest-path problem;
      when \(V\) contains all vertices of \(G\), this is the
      minimum-spanning-tree problem. Surprisingly enough, while both those
      problems are polynomial-time-solvable, this problem is NP-hard.)

      This paper proposes a (purportedly linear-time) algorithm to generate
      test cases for the Steiner tree problem in graphs.  The key technique
      involved in this algorithm is applying the so-called
      ``Karush-Kuhn-Tucker'' optimality conditions (which appear to be a sort
      of generalization of the Lagrange-multipliers method for solving an
      optimization-with-constraints problem).

      This paper does not contain much discussion on theoretical complexity
      topics but is instead valuable as a ``deep dive'' into a particular
      NP-hard problem (and, perhaps, the broader category of
      combinatorial-optimization problems).
    \end{annotation}

  \item \fullcite{maximum-clique-generators}

  \item \fullcite{sudoku-education}

\end{itemize}


\end{document}
