%% Use the hmcposter class with the clinic document-class option.
\documentclass[clinic]{hmcposter}
\usepackage{graphicx}
\usepackage{natbib}
\usepackage{booktabs}
\usepackage{subfig}
\usepackage{amsmath}
\usepackage{textcomp}
\usepackage{url}

%% For a Clinic poster, there is no author (the author is the team).

%% Project Year.
%% The year is provided using the \year command.
\posteryear{2016}

%% Project Title.
%% The title of the poster should probably be the name of your
%% Clinic project.
\title{Putting Together a Poster\\for Presentation Days}


%% Sponsor's Name.
%% Your sponsoring institution's name.
\sponsor{Harvey Mudd College}

%% Sponsor's Logo.
%% The name (base name only; no extension) of an image file with
%% the sponsor's logo.  Ideally, you'll have a PDF version that is
%% resolution-independent.  If not, you'll need a high-resolution
%% PNG file to allow us to print it at a large size without the
%% image becoming blurred.
\sponsorlogo{hmcmath-hexen-logo}

%% Optional -- if your sponsor's logo looks too small or too big,
%% you can adjust its width with the \sponsorlogowidth command.
%% (The height of the logo image is automatically adjusted to
%% preserve the image's aspect ratio.)
%%
%% Note that the argument must be a TeX length; for example, 3in,
%% 5cm, 120pt, etc.  The default width is 2in.
\sponsorlogowidth{4in}


%% Optional -- if your sponsor is hot about their intellectual
%% property and insists on having a copyright statement on the
%% poster, you can use the \copyrightholder command to supply a
%% name for a copyright holder for your poster.
% \copyrightholder{Sponsoring Corporation, Inc.}


%% Define the \BibTeX command, used in our example document.
\providecommand{\bibtex}{{\rmfamily B\kern-.05em%
    \textsc{i\kern-.025em b}\kern-.08em%
    T\kern-.1667em\lower.7ex\hbox{E}\kern-.125emX}}


\pagestyle{fancy}

\begin{document}

\begin{poster}

\section{Results}

Summarize the results of your project here---what have you
learned, and what does what you've learned mean for your reader,
the world at large, and your future research?

It's important to give your reader a reason to keep reading your
poster---why should they care about your project?  Tell them!

Diagrams and images---charts, graphs, photographs---are all good
things to include here.

Integer aliquam auctor erat. Duis velit nulla, nonummy nec,
elementum vel, congue sed, mauris:

\begin{itemize}
\item Fusce bibendum ipsum nec leo. 
\item Mauris ac odio. 
\item Nulla facilisi. 
\item Suspendisse vel lorem.
\end{itemize}
Vestibulum non ante a mi consequat porta. Aliquam sapien
purus, rhoncus ac, suscipit quis, bibendum at, justo. Proin sed
lacus. Sed laoreet scelerisque ipsum. Nulla velit mauris, sagittis
a, pretium sed, posuere id, mauris. Phasellus ligula. Vivamus eu
felis. Nam nunc.


\section{Materials and Methods}%

Some basic information about how you went about doing your
research.  You might want to have some images here that show some of
your equipment, your lab setup, or some other relevant things.  (For
example, for a project on image processing, you might have a
``before'' and ``after'' image.)

\begin{itemize}
\item Use itemized lists rather than full paragraphs of text
\item Remember, people are supposed to be intrigued by your poster
\item The details should be in your full report or thesis
\item Tell them how to get those (if they can) at the end of the poster
\end{itemize}


\section{Results}%

Here we have the real meat of the poster.  Talk about what you did,
how it worked out, and how it could have gone better.

Diagrams and images---charts, graphs, photographs---are all good
things to include here.


\section{Conclusions}

Rather than just a summary of your findings (which you presented in
the previous section), write your \emph{conclusions} based on those
results---what have you learned, and what does what you've learned
mean for your reader, the world at large, and your future research?


\section{For Further Information}

Possibly the most important section of your poster!  Tell people
how they can find out more about your research.  Be sure to
include
\begin{itemize}
\item Your e-mail address.  Mine's \url{cmc@math.hmc.edu}.
\item A URL for a website with more information.  \url{http://www.math.hmc.edu/computing/support/printing/posters/}.
\end{itemize}



\vfill
\columnbreak

\subsection{Figures in Small Multiples}

Sometimes you need (or want) to include more than one image in a
figure, such as when you have several close variations on a single
image, as shown in Figure~\ref{fig:small-multiples}, which has
subfigures \subref*{fig:small-mults-orig} or
\subref{fig:small-mults-45}.  You could also refer to the subfigures
as Figure~\ref{fig:small-mults-90} or
Figure~\ref{fig:small-multiples}\subref{fig:small-mults-135}.

\begin{figure}
  \centering
        \subfloat[][Original image.]{\scalebox{.75}{\includegraphics{shapes}}%
                \label{fig:small-mults-orig}%
        }\qquad\qquad
        \subfloat[][Turned 45\textdegree.]{\includegraphics[scale=.75,origin=c,angle=45]{shapes}%
                \label{fig:small-mults-45}%
        }\\
        \subfloat[][Turned 90\textdegree.]{\scalebox{.75}{\includegraphics[origin=c,angle=90]{shapes}}%
                \label{fig:small-mults-90}%
        }\qquad\qquad
        \subfloat[][Turned 135\textdegree.]{\scalebox{.75}{\includegraphics[origin=c,angle=135]{shapes}}%
                \label{fig:small-mults-135}%
        }
  \caption[Small multiples]{Small multiples.}%
  \label{fig:small-multiples}
\end{figure}






\section{Formatting References}

As usual, you want to cite anything that you've taken from other
sources, and provide the details here.  If you don't want to use
\bibtex, you can just put an itemized list here instead.


%% References.
%% Note that BibTeX will add its own section-level header, ``References''.

\bibliographystyle{hmcmath}
\bibliography{sampleposter}


\section{Acknowledgments}

If there are people or institutions that were particularly helpful
to you during your research, thank them here.  It's especially
important to mention anyone who gave you money.

I want to express my appreciation to the Department of Biology at
Swarthmore College, who provided an excellent sample poster
\citep{swarthmore-poster} that helped inspire this version of our
sample poster.  Colin Purrington also maintains an excellent page
with information about designing scientific
posters. \citeyearpar{purrington-sciposters}


\subsection{Team Members}

\begin{multicols}{2}
\setlength{\columnseprule}{0pt}
\begin{itemize}
\item Esmerelda Weatherwax (Project Manager)
\item Gytha Ogg
\item Magrat Garlick
\item Agnes Nitt
\item \textbf{Faculty Advisor} Mustrum Ridcully
\item \textbf{Liaison} Rincewind
\end{itemize}
\end{multicols}


\vfill


\end{poster}

\end{document}

 
