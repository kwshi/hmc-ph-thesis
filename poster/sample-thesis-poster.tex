%% Use the hmcposter class with the thesis document-class option.
\documentclass[thesis, landscape]{hmcposter}

\usepackage{mathtools}
\usepackage{amsmath}
%\usepackage{lmodern}
%\usepackage{unicode-math}
\usepackage{libertinus}
\renewcommand\familydefault\sfdefault

\setlength\parindent{0pt}
%\usepackage{parskip}

\usepackage[compact, medium]{titlesec}
\titlespacing\section{0pt}{.25em}{.25em}
\titlespacing\paragraph{0pt}{.25em}{1em}


\usepackage[inline]{enumitem}
\setlist[itemize]{
  left=1em..2em,
  label={\color{gray}\textbullet},
}

\usepackage{graphicx}
\usepackage{natbib}
\usepackage{booktabs}
\usepackage{subfig}
\usepackage{textcomp}
\usepackage{url}

\author{Kye Shi}
\posteryear{2022}
\title{Games for one, games for two!}
\subtitle{Computationally complex fun for polynomial-hierarchical families}

\class{Math 197: Senior Thesis}

\advisor{Nicholas J. Pippenger}
\reader{Arthur Benjamin}

\usepackage{colortbl}
\setlength\arraycolsep{.5em}
\setlength\tabcolsep{.5em}
%\setlength\tabrowsep{2em}
\setlength\arrayrulewidth{2pt}
\arrayrulecolor{gray}

\usepackage{complexity}

\pagestyle{fancy}


\tikzset{
  gate/.style={
    draw,
    line width=2pt,
    rounded corners=1em/8,
    inner sep=1em/4,
  },
  every path/.style={line width=2pt},
  pipe/.style={
    line width=3pt,
    %rounded corners=1em/2,
    %to path={
    %  (\tikztostart)
    %  -- ($ (\tikztostart -| \tikztotarget)!1/2!(\tikztostart) $)
    %  %-- ($ (\tikztotarget)!1em!(\tikztostart |- \tikztotarget) $)
    %  -- (\tikztotarget)
    %},
    -{Stealth[length=1em/2]},
  },
  common baseline/.style={text depth=1em/3, text height=2em/3},
  over/.style={
    preaction={
      draw=white,
      line width=7pt,
      -,
    },
  },
  gates/.style={
    row sep=1em, column sep=2em, matrix of math nodes, inner sep=0pt,
  },
  input/.style={
    circle,
    inner sep=0pt,
    minimum size=1em,
    draw,
    line width=2pt,
    text opacity=1,
  },
  vertex/.append style={
    line width=2pt,
  },
  pics/circuit example/.style n args={8}{
    code={
      \matrix[gates, ampersand replacement=\&]{
        |[input, #1](x)|x \& |[gate, #4](xy)|∧ \\
        |[input, #2](y)|y \& |[gate, #5](yz)|∧ \&\& |[gate, #8](or')|∨ \\
        |[input, #3](z)|z \& |[gate, #6](zx)|∧ \\
      };

      \node[gate, #7](or) at ($ (xy)!1/2!(or') $){\(∨\)};

      \draw[pipe, #2] (y) to (xy);
      \draw[pipe, #2] (y) to (yz);
      \draw[pipe, #3] (z) to (yz);
      \draw[pipe, #3] (z) to (zx);
      \draw[pipe, #1, over] (x) to (zx.north west);
      \draw[pipe, #1] (x) to (xy);

      \draw[pipe, #4] (xy) to (or);
      \draw[pipe, #5] (yz) to (or);
      \draw[pipe, #7] (or) to (or');
      \draw[pipe, #6] (zx) to (or');

      \draw[pipe, #8] (or'.east) -- +(2em,0) coordinate(out);
    },
  },
  group/.style={
    draw,
    rounded corners=1em/4,
    densely dashed,
    line width=2pt,
    opacity=1/4,
    node contents=,
  },
  pics/circuit game/.style n args={8}{
    code={
      \pic{circuit example={#1}{#2}{#3}{#4}{#5}{#6}{#7}{#8}};
      \node[fit=(x), group, "\(I₁\)" {left, opacity=1/2}];
      \node[fit=(y), group, "\(I₂\)" {left, opacity=1/2}];
      \node[fit=(z), group, "\(I₃\)" {left, opacity=1/2}];
    },
  },
  blank/.style={
    draw opacity=1/2,
  },
  done/.style={
    fill=ks#1,
    draw=ks#1,
    text=white,
  },
  pics/fake or/.style n args={5}{code={

      \coordinate[vertex, #4](#2) at ($ (#1) + (0,-2em) $);
      \coordinate[vertex, fill=ks2](aux) at ($ (#1) + (1em,-1em) $);
      \coordinate[vertex](#1') at ($ (#1) + (2em,0) $);
      \coordinate[vertex](#2') at ($ (#1) + (2em,-2em) $);
      \coordinate[vertex, #5](#3) at ($ (#1) + (3em,-1em) $);
      \draw (#1') -- (#3) -- (#2') -- (#2) -- (aux) -- (#1) -- (#1') -- (#2');

  }},
  vertex label/.style={
    common baseline,
    inner xsep=1em/4,
    left,
  },
  pics/and or/.style 2 args={code={
      \matrix[ampersand replacement=\&, inner sep=0pt]{
        \coordinate[vertex, "\(#1x\)" vertex label](x);
        \pic{fake or={x}{y}{t}{"\(#1y\)" vertex label}{}};
        \pic{fake or={t}{nz}{o}{"\(#2z\)" vertex label}{fill=ks1}}; \&
        \coordinate[vertex, "\(#2x\)" vertex label](nx) at (0,-1em);
        \pic{fake or={nx}{z}{o}{"\(#1z\)" vertex label}{fill=ks1}}; \&
        \coordinate[vertex, "\(#2y\)" vertex label](ny) at (0,-1em);
        \pic{fake or={ny}{z}{o}{"\(#1z\)" vertex label}{fill=ks1}}; \\
      };
  }},
}

\begin{document}

\begin{poster}

\section{Puzzles, games, and hierarchies}

Many well-known one-player games, such as \emph{Sudoku}, Minesweeper, and
Solitaire, are ``difficult'' for computers to solve---formally: they are
\emph{\NP-complete}. What is \NP?  What is \P?  How are they related, and how
does that relationship generalize?

My thesis takes the following perspective:
\begin{itemize}
  \item Problems in \P{} represent games with ``\(0\) turns'' (e.g., given a
    fully-filled-in Sudoku grid, decide whether it's valid).
  \item Problems in \NP{} represent games with \(1\) turn (e.g., given an
    incomplete grid, decide whether a winning move exists).
\end{itemize}
There are infinitely many more classes of problems along this \emph{hierarchy}:
\SigmaP k, representing decision problems for games with \(k\) turns, for each
\(k∈\Set{0,1,2,3,\dotsc}\).  These classes form the \Term{polynomial hierarchy}.

We explore two collections of polynomial hierarchy games: boolean
circuit satisfiability games, and graph 3-coloring games.

\section{Boolean circuits}

\Term{Boolean circuits} consist of \Term{wires}, which carry \True/\False{}
values, sent through various \Term{logic gates}:
\[
  \begin{array}{c|cl}
    \NOT & ¬x & \text{flips the value of \(x\)} \\
    \AND & x∧y & \text{\True{} iff both of \(x,y\) are \True} \\
    \OR & x∨y & \text{\True{} iff at least one of \(x,y\) are \True}
  \end{array}
\]

A circuit is \Term{satisfied} if its output is \True.  For example, the circuit
below is satisfied if and only if at least \(2/3\) of its inputs are \True:

\begin{center}
  \begin{tikzpicture}
    \pic{circuit example={}{}{}{}{}{}{}{}};
    \node[right] at (out) {\(xy∨yz∨zx\)};
  \end{tikzpicture}
\end{center}

Here is a circuit that is \emph{unsatisfiable}:

\begin{center}
  \begin{tikzpicture}
    \matrix[gates, ampersand replacement=\&]{
      |[input](x)|x \& \& |[gate, done=0](and)|∧ \\
      \& |[gate](not)|¬ \\
    };

    %\draw[pipe] (x) to (buf);
    \draw[pipe] (x) -- (not);
    \draw[pipe] (x) to (and);
    \draw[pipe] (not) to (and);

    \draw[pipe, draw=ks0] (and) -- +(2em,0) node[right]{\(x∧¬x=\False\)};

  \end{tikzpicture}
\end{center}

\columnbreak

\section{Circuit satisfiability games}

The objective: \emph{given a circuit, can you satisfy it?}

A \(k\)-turn circuit game proceeds as follows.
\begin{itemize}
  \item The game starts with a circuit whose inputs are partitioned into \(k\)
    disjoint groups labeled \(I₁,I₂,\dotsc,Iₖ\).
  \item Two players play against each other: on turn \(i\), player \((i\bmod2)\)
    picks values for all inputs in \(Iᵢ\).
\end{itemize}
Player \(1\) wins if the circuit ends up satisfied; player \(2\) wins if the
circuit is falsified.  Who has the winning strategy?

Shown below is an example game in which player \(1\) wins.

\begin{center}
  \tikzset{every picture/.style={baseline=(y.base)}}
  \begin{tabular}{c|c}
    (start) & \tikz{\pic{circuit game={blank}{blank}{blank}{blank}{blank}{blank}{blank}{blank}}} \\\midrule
    P\(1\): \(x=\True\) &
    \tikz{\pic{circuit game={done=1}{blank}{blank}{blank}{blank}{blank}{blank}{blank}}} \\\midrule
    P\(2\): \(y=\False\) &
    \tikz{\pic{circuit game={done=1}{done=0}{blank}{done=0}{done=0}{blank}{done=0}{blank}}} \\\midrule
    P\(1\): \(z=\True\) &
    \tikz{\pic{circuit game={done=1}{done=0}{done=1}{done=0}{done=0}{done=1}{done=0}{done=1}}}
  \end{tabular}
\end{center}

\emph{Any algorithm can be represented as a sequence of Boolean circuits.}  (In
a sense, this is literally how real computers work!)

Consequently, any \SigmaP k game can be converted to an equivalent \(k\)-turn
\emph{circuit} game.  Thus, \(k\)-turn circuit games are \SigmaP k-complete.

\section{Graph-coloring games}

Give each vertex in a graph one of three colors \(\Colors\).  The coloring is
\Term{proper} if every two neighboring vertices have different colors:

\begin{center}
  \tikzset{
    pics/four/.style n args={4}{
      code={

        \coordinate[vertex, fill=ks#1](a) at (1em,0);
        \coordinate[vertex, fill=ks#2](b) at ($ (a) + (120:2em) $);
        \coordinate[vertex, fill=ks#3](c) at ($ (a) + (180:2em) $);
        \coordinate[vertex, fill=ks#4](d) at ($ (a) + (60:2em) $);

        \draw[line width=2pt] (a) -- (b) -- (c) -- (a) -- (d);

      },
    },
  }
  \begin{tikzpicture}

    \matrix[column sep=2em, inner sep=0pt] {
      \pic{four=2100}; & \pic{four=2120}; &
      \coordinate[vertex](o) at (0,1em);
      \foreach \i in {0,...,4} {
        \coordinate[vertex](v\i) at ($ (o) + ({360/5*\i+90}:2em) $);
        \draw[line width=2pt] (o) -- (v\i);
      }
      \draw[line width=2pt] (v0) -- (v1) -- (v2) -- (v3) -- (v4) -- (v0); \\
      \node[common baseline]{proper}; & \node[common baseline]{improper}; &
      \node[common baseline]{not properly-colorable}; \\
    };

  \end{tikzpicture}
\end{center}

The objective: \emph{given a graph, can you find a proper coloring for it?}

A \(k\)-turn 3-coloring game proceeds as follows:
\begin{itemize}[nosep]
  \item Start with a graph with vertices partitioned as \(V₁,\dotsc,Vₖ\).
  \item On turn \(i\), player \((i\bmod 2)\) colors all vertices in \(Vᵢ\).
\end{itemize}
If, on any turn, a player creates an improper edge, they lose. Otherwise, if all
turns are played properly, the last player wins.

Any boolean circuit can be converted into an ``equivalent'' 3-coloring graph.
Associate \(\ColorId1≡\True\), \(\ColorId0≡\False\); convert logic gates to
graphs:

\begin{center}
  \tikzset{every picture/.style={baseline=(current bounding box.center)}}
  \begin{tabular}{c|c|c}
    \NOT & \(y=¬x\) &
    \begin{tikzpicture}
      \coordinate[vertex, "\(x\)" vertex label](x);
      \coordinate[vertex, "\(y\)" {vertex label, right}](y) at (2em,0);
      \coordinate[vertex, fill=ks2](aux) at (1em,-1em);
      \draw (x) -- (y) -- (aux) -- (x);
    \end{tikzpicture} \\ \midrule
    \AND & \(z=x∧y\) & \tikz{\pic{and or={¬}{}}} \\ \midrule
    \OR & \(z=x∨y\) & \tikz{\pic{and or={}{¬}}}
  \end{tabular}
\end{center}

To convert between \emph{games}, map \(I₁,\dotsc,Iₖ\) to \(V₁,\dotsc,Vₖ\), and
also include logic-gate-vertices in \(Vₖ\).  Pre-color the output vertex
\ColorId1; the last player wins iff the circuit is finally satisfied.  Thus
\(k\)-turn graph-coloring games are \SigmaP k-complete for odd \(k\), and
\((\SigmaP k)\Complement\)-complete for even \(k\).

\section{Conclusions}

Only two ``flavors'' of \(\SigmaP k\)/\((\SigmaP k)\Complement\)-complete games
are presented here.  However, there are certainly many more out there.  The
basic ideas explored here are readily generalized; almost every \NP-complete
problem, I suspect, gives rise to interesting \(k\)-turn generalization.

\end{poster}

\paragraph{Acknowledgments}

Many thanks to Prof. Nicholas Pippenger, my advisor, for patiently putting up
with me despite my slow pace of progress.  Also thanks to Prof. Jon Jacobsen,
Melissa Hernandez-Alvarez, and DruAnn Thomas, for running the well-oiled thesis
machine and providing a (generally) enjoyable senior thesis experience. And of
course, thanks to Sheri Hsueh, mom, for funding my exorbitant college education.

\paragraph{References}
\begin{enumerate*}[label=\textcolor{gray}{[\arabic*]}]
  \item Papadimitriou, Christos H. (1993). \textit{Computational Complexity}. University
    of California -- San Diego: Addison-Wesley.
  \item Cormen, Thomas H.; Leiserson, Charles E.; Rivest, Ronald L.; Stein, Clifford.
    (2001). \textit{Introduction to Algorithms}. The MIT Press.
\end{enumerate*}

%\begin{enumerate*}
%  \item 
%  \item Potapov, Igor. ``\textit{3-Coloring is NP-Complete}'' (2020).
%    \url{https://cgi.csc.liv.ac.uk/~igor/COMP309/3CP.pdf}.
%\end{enumerate*}




%\begin{itemize}[nosep]
%  \item Feel free to email me at \url{kwshi@hmc.edu}.
%  \item Check out the project website at \url{TODO}
%  \item download the full thesis report at \url{insert claremont thesis URL here}
%  \item GitHub page: \url{https://github.com/kwshi/hmc-math-thesis}
%\end{itemize}

%\bibliographystyle{hmcmath}
%\bibliography{sampleposter}



\end{document}
