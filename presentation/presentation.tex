\documentclass{presentation}

\tikzset{
  not demo/.pic={
    \matrix[gates] {
      \coordinate[input](x); \& \node[not gate](n){}; \& \coordinate[output](o); \\
    };
    \draw[wire] (x) to (n.input);
    \draw[wire] (n.output) to (o);
  },
}

\title{You can solve it, but can you play it?}
\author{Kye Shi}
\date{2021 November 30}

\begin{document}

\begin{frame}
  \maketitle
\end{frame}

\begin{frame}
  not gates and or gates
  \begin{center}
    \begin{tikzpicture}
    \end{tikzpicture}
  \end{center}
\end{frame}

\begin{frame}
  circuit satisfiability puzzle
  \begin{center}
    \begin{tikzpicture}
      \matrix[gates]{
        \coordinate[input](x); \\
        \& \node[not gate](nx){}; \& \node[or gate](o1){}; \& \node[not gate](n1){}; 
        \& \node[or gate](o4){}; \& \coordinate[output](out);
        \& \coordinate[output](y); \\
        \coordinate[input](y);
        \& \node[not gate](ny){}; \& \node[or gate](o2){}; \& \node[not gate](n2){}; \\
        \& \node[or gate](o3){}; \\
        \coordinate[input](z); \\
      };


      \draw[wire] (x) to (o1.input 1) (x) to (nx.input);
      \draw[wire] (y) to (ny.input) (y) to (o3.input 1);
      \draw[wire] (z) to (o3.input 2);

      \draw[wire] (nx.output) to (o2.input 1);
      \draw[wire] (ny.output) to (o1.input 2);

      \draw[wire] (o3.output) to (o2.input 2);

      \draw[wire] (o1.output) to (n1.input);
      \draw[wire] (o2.output) to (n2.input);

      \draw[wire] (n1.output) to (o4.input 1);
      \draw[wire] (n2.output) to (o4.input 2);

      \draw[wire] (o4.output) to (out);

      \node[left] at (x){\(x\)};
      \node[left] at (y){\(y\)};
      \node[left] at (z){\(z\)};

    \end{tikzpicture}
  \end{center}

  remarks: can compute any boolean function.  "complete", which means any
  puzzle (solution can be obtained by guess-and-check) can be reduced to it.

\end{frame}

\begin{frame}
  enter jon jacobsen
\end{frame}

\begin{frame}
  statement of general game version, hierarchy.  take as characterization,
  "definition"
\end{frame}

\begin{frame}
  another puzzle: graph coloring.

  how hard is it?
\end{frame}

\begin{frame}
  reduction to circuit sat.
\end{frame}

\begin{frame}
  conclusion.
\end{frame}

\end{document}
