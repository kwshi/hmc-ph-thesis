\documentclass{presentation}

\begin{document}
  
\begin{frame}
  title slide: you can solve it but can you play it?
\end{frame}

\begin{frame}
  what is it?

  sudoku---rules.
\end{frame}

\begin{frame}
  sudoku, example solution approach.  
\end{frame}

\begin{frame}
  arbitrary size generalization.  formal statement of problem, existence
  question.  (mention not-necessarily-unique? prob not.)
\end{frame}

%\begin{frame}
%  on hold!  graph coloring puzzle.  given a graph, a set of colors, a few
%  pre-colored nodes, does there exist?
%\end{frame}
%
%\begin{frame}
%  example puzzle and solution.
%\end{frame}
%
%\begin{frame}
%  pause.  why did I describe two puzzles?  are they related?  yes.  here is a
%  claim: if I know how to solve graph coloring, I can easily solve sudoku.
%
%  what does this mean?  how can I go about showing this?
%  suppose I know how to solve graph coloring.  then, given a sudoku puzzle...
%
%  construct a graph coloring puzzle out of it; solve the graph coloring puzzle;
%  trivially back-convert solution.
%\end{frame}
%
%\begin{frame}
%  given a sudoku puzzle (size n, nsquare board + pre-filled cells), how to
%  "convert" to graph coloring puzzle (n colors, graph + pre-colored nodes)?
%
%  natural translation: size -> colors, board -> graph, pre-filled cells -> nodes.
%\end{frame}
%
%\begin{frame}
%  demo of reduction.
%\end{frame}
%
%\begin{frame}
%  sudoku is "easier than" graph coloring.  no harder than.  less-equals.
%
%  also, turns out, by a far more convoluted argument, gcp le sudoku.
%\end{frame}

\begin{frame}
  ask the question: in general, how hard is it to solve it?
\end{frame}

\end{document}
